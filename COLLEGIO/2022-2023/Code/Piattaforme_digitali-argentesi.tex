\documentclass[10pt]{article}
\usepackage[utf8]{inputenc}
\usepackage{graphicx} % Required for inserting images
\usepackage{amsmath}
\usepackage{amsthm}
\usepackage{amssymb}
\usepackage[a4paper,left=2.1cm, right=2.1cm, top=2cm, bottom=2cm]{geometry}

\title{The power of Big Tech:\\competition in digital markets\\(prof. E.Argentesi)}
\author{Lecture notes by A.Z.}
\date{March 2023}

\begin{document}

\maketitle

\section{Lecture 1}
\paragraph{Who are big tech?} Dominant companies / firms / platforms in information technology industry in terms of:
\begin{enumerate}
    \item \textbf{Market value} combined $\approx$ 7 trillion \$
    \item \textbf{Market concentration} Dominant Position: e.g. Google market share (search engines) = 83\%, Microsoft (desktop OS) = 76\%, Android and iOS (duopoly in mobile OS) = 71.8 + 27.6 \%, Amazon 13 \% of Gross Merchandise Volume [\textit{total sales monetary value of merchandise sold through a marketplace in a certain time}]
\end{enumerate}

\paragraph{Benefits of digital markets}
\begin{itemize}
    \item New products and services
    \item Highly innovative prod.
    \item Extremely low prices (even zero)
    \item Low cost of starting (and operating) a business $\rightarrow$ lots of start-ups
\end{itemize}
But characterized by strong market power.

\paragraph{Pros and cons of market concentration}
\begin{description}
\item[Pros] \begin{itemize}
    \item  Firms may become dominant because they offer better
products at lower prices
\item Market concentration may be efficient (lower costs due to \textit{economy of scale} : reduced cost/unit for larger numbers)
\item Dominance may not last forever (\textit{turnover})
\end{itemize}
\item[Cons]
\begin{itemize}
    \item Higher prices : unregulated monopolist / oligopolists could exploit market power
    \item Concerns for non-economic sides : e.g. \textbf{Data exploitation} for advertisement (attention market) $\rightarrow$ 'shallow price'
    \item Lower quality + less innovation : no incentive for firms without challengers (no risk of losign customers)
\end{itemize}
\end{description}

\paragraph{Gateways} = few dominant companies / platforms : broad influence $\rightarrow$ economic, socio-political (can bias political outcomes, drive public opinion), on media pluralism (see Google News), on privacy
\paragraph{How did they get there?} Internal growth + \textbf{External g. (!)} $\rightarrow$ acquisition of competitors : actual or potential (not only diversification strategy...)

\paragraph{Benefits of competition}
\begin{itemize}
    \item Lower prices
    \item Wider variety of products (more choice)
    \item More incentives towards innovation (\textit{Competition breeds i.})
    \item Creates better competitors for global markets (exporters)
    \item Consumer / social welfare improvement : e.g. mobile telecom market $\rightarrow$ Iliad Italy (2018) $\approx$ -20\% prices, Free France (2012) total welfare gains $\approx$ 1.2 mln euro
\end{itemize}

\paragraph{Competition policy} = \textit{aims at
ensuring that competition in the marketplace
is not restricted in a way that is detrimental to
society}
\\Needed to avoid persistence of Dominant Positions (even in markets without \textit{natural anti-competitive features}) due to:
\begin{itemize}
    \item High sunk costs : high stakes for entry / survival (e.g. R\&D)
    \item Reluctance towards change by consumers (psychological factor) $\rightarrow$ \textit{lock-in effects}
    \item Network effects : tendency to 'follow the herd' / difficult for challengers to gain 'critical mass'
\end{itemize}

\paragraph{Harmful practices} for social welfare put in place by firms with DP:
\begin{itemize}
    \item Abuse of DP : predatory / exclusionary behaviour
    \item Collusion (Cartel, Trusts $\rightarrow$ etymology of anti-t. !)
    \item Mergers and acquisitions    
\end{itemize}

\paragraph{Regulation $\neq$ competition policy} : 
\begin{description}
    \item[R.] acts \textit{ex ante} in markets with unavoidable tendency towards monopolization / strategic assets, infrastructure. Has more extensive powers, can intervene on market structure, continuous intervention.
    \item[C.P.] only \textit{ex post} in \textit{open markets} against dangerous behaviours. Occasional intervention (triggered), longer time span needed $\leftarrow$ different information intensiveness: authorities need to gather industry-specific info on case-by-case base
\end{description}  

\paragraph{EU Competition Law : pillars} 
\begin{itemize}
\item From Treaty on the Functioning of EU (TFEU, one of the two forming the constitutional base of union) : \textit{art. 101} against anti-competitive agreements (horizontal and vertical) - similar to US Sherman Act, \textit{art. 102} against abuse of DP
\item Merger regulation
\item State aid control : limit distortions
\end{itemize}

\paragraph{Authorities} enforcing law: Directorate-General for Competition + Commissioner for Competition (Margrethe Vestager) + national competition agency (Italy = Autorità Garante della Concorrenza e del Mercato) - subsidiarity principle 

\section{Lecture 2}
\paragraph{Key features of Digital Markets / industries} Supply-side factors:
\begin{description}
        \item[Economies of scale] extremely low \textit{marginal cost} $\rightarrow$ favours concentration
        \item[Economies of scope] greater variety of products $\rightarrow$ lower costs for diversification. 
        \\In D.M. most products are oriented towards acquisition and usage of \textit{data} 
        \item[Data (key input!)]$\rightarrow$ consumers' habits tracking ($\rightarrow$ targeted product development, advertisement and pricing*) + data-based services (e.g. Search Engines) + Recommendation systems
        \\* Algorithmic price-setting : even led to algorithms colluding!
        \\Data endowments ($\approx$ patrimony) $\rightarrow$ competitive advantage
\end{description}
Demand-side:
\begin{description}
    \item[Strong network effects] = when benefits for users depend on how many others are relying on the same service
    \begin{itemize}
        \item Direct NE = purely social behaviour (e.g. social media) $\neq$ Indirect = involve \textit{cross NE} in \textit{multi-sided markets} (e.g. SE : users + advertisers) - positive-feedback loops
        \item \textit{Gatekeepers} largely control audience attention $\rightarrow$ \textit{market tipping} (high barriers to entry)
    \end{itemize}
    \item[Switching cost] (economic + psychological) : unexpectedly strong (\textit{power of default}) - consumers' behavioural bases
\end{description}
\paragraph{Large ecosystems} = large platforms / companies increasingly more efficient at offering large set of services. \\Arise because of Economies of scale, scope + Network effects
\\Have positive effects but become detrimental if companies voluntarily adopt anti-comp. practices : \textit{incumbency advantage}.
\\\textbf{Mitigation} : \textit{Multi homing} (easiness of switching or using multiple services - patronizing mult. products) + Product differentiation + Rapid innovation pace

\paragraph{Theories of Harm} = \textit{coherent stories that explain why an agreement between firms or a practice put in place by one may harm competition and affect customers - compared to a relevant counterfactual}
\\Considers market features + incentives, ability of firms, consistent with economic theory and empirical evidence. Consumers = main focus on users (not businesses). Measurement / quantification of consumer welfare (non-price effects e.g. quality, innovation) $\rightarrow$ econometric techniques - if impossible to apply, evaluation of welfare gains

\paragraph{Abuse of dominance ToH} \textit{effect-based approach} (+ behavioural case study) : needed to face new anti-comp practices.
\\\textit{Exclusionary} abuses (deterrence of entry or forced exit of potential / actual rivalss) + \textit{exploitative} ones (e.g. concerning data)
\\Art. 102 : dominant position not in itself illegal - but implies special responsibility. Abuses : exclusive purchasing, predatory pricing / predation (set prices at loss-making level), refusal to supply necessary input for ancillary markets, excessive pricing.
\\Peculiar practices (challenge for authorities to define and intervene):

\section{Lecture 3}

\paragraph{Tying and bundling} = linkages between digital products $\rightarrow$ firms may be incentivized.
\begin{description}
    \item[Bundling] more products sold together as a package: can be pure or mixed (still possible to buy / use separately)
    \item[Tying] purchase of a product conditioned to the p. of another
\end{description}
Benefit for customers : common interface, one-stop shopping (exploit complementarities of products) + price discrimination = mixed net effect on market welfare
\\Abuse of Dominance? Can cause \textit{foreclosure} / entry deterrence $\rightarrow$ \textit{Leverage theory} : monopoly in one market exploited to deter challengers' entry in another one somehow complementary with the first $\rightarrow$ \textbf{Bundling with complementarities} complem. products that can only be used together : impossible to focus on production of single one $\rightarrow$ difficult entry + competitors need to match bundle composition
\\Jean Tirole (Nobel 2014) : start-ups enter niches (not all segments altogether $\rightarrow$ bundling prevents efficient entrants (expecially in case of NE - reinforcement, amplification) $\rightarrow$ strenghtening / extension of market power
\\\textbf{Cases} Windows Media Player, Internet Explorer, Android (2018 - sued by US DoJ in 2020) $\rightarrow$ used mobile OS to cement dominance of SE (required manufacturers to install Chrome + Google app as condition for play store licensing, prevented selling of devices with forks)

\paragraph{Platform Envelopment} = combination of functionalities in platforms $\rightarrow$ analogous to bundling but new and peculiar of DM from technical standpoint. Leads to user base leveraging, foreclosure, benefits from economies of scope and NE.
\\Special case: \textit{Privacy Policy tying} = consumer required to grant consent to combination of data $\rightarrow$ effects: monetization of combined data (\textit{static incentive}) + monopolization of target market + reinforcement / entrenchment of DP in native market segment

\paragraph{Self-preferencing} = platform in which transaction take place is also a competitor in inner market $\rightarrow$ advantages own products
\\Debate on anti-comp. effects (leveraging) $\rightarrow$ Solutions : \textit{presumption} of anti-competitiveness in case of relevant intermediate infrastructure + Obligations for Gatekeepers (DMA)

\paragraph{Sherlocking} = free-riding (taking advantage of other's work) or cloning of innovative products by big firms (name from Apple's macOS file search tool Sherlock, which copied indipendent developer's Watson)

\paragraph{Other ToH} \textbf{Access to data} require public access? Negative trade-offs : authoritarian gov.ts, collapse of data market
\\\textbf{Combination of D.}, \textbf{Predation} difficulties in proving illicit and assessing \textit{recoupment} (reparation)

\paragraph{Limitations of current approach}
\begin{itemize}
    \item Lenght of intervention (critical in fast-changing markets!)
    \\ $\uparrow$
    \item New ToH (new practices put in place) and high burden of proof required (strong defence) $\rightarrow$ under-enforcement ?
    \item Limited effectiveness of remedies (hard to design effective ones)
    \item Impossibility to adress \textit{structural} problems
    \item Possibile necessity of a more regulatory approach
\end{itemize}

\paragraph{The DMA} approved 10/2022, will enter into force $\approx$ 05/2023 $\rightarrow$ definition of \textit{Gatekeepers} + list of obligations (e.g. No self-preferencing, no exclusionary practices against challenging platforms, no extended tracking for targeted advertisement outside core service)

\paragraph{The Digital Services Act} (11/2022) protects customers' rights, establishes \textit{transparency and accountability framework}, promotes / fosters common, integrated european market

\paragraph{Anti-competitive agreements} (prohibited by art.101 TFEU)
\begin{description}
    \item[Horizontal / collusive] generally harmful for comp.
    \item[Vertical] between platforms and providers of services (often in two-sided markets, e.g. Expedia Booking) $\rightarrow$ can have efficiency justification, lead to issues in case of large market power (foreclosure in distribution channel)
\end{description}

\section{Lecture 4}

\paragraph{Merger review} by competition authorities to prevent entrenchment of DP $\rightarrow$ EU \textit{Merger Guidelines} (2004) : more general case, horizontal m. prohibited if would \textit{significantly impede effective competition} (established \textit{threshold} - events below it are 'automatically' authorized)

\paragraph{Authorities} EU : Commission + national auth
\\US : Federal Trade Commission + Department of Justice

\paragraph{Type of mergers}
\begin{description}
    \item[Horizontal] between competitors (most concerning x authorities) : increased market power / risk of collusion between few dominant firms $\rightarrow$ harmful to consumers in case of \textit{unilateral effect} : arbitrary price setting
    \item[Vertical] = different levels of \textit{value / production chain}
    \item[Conglomerate] = different industries : can have financial reasons or happen between adjacent markets / complementary products
\end{description}
Last two are less concerning because of possible positive effects on efficiency
\\\textbf{Good mergers} = actual increase of ability of new firms to enter market + efficiency gains $\rightarrow$ cost savings can lead to lowered prices, but socio-economical and consumer welfare should be considered as a whole : quality ? 
\\Powerful buyers / retailers with relevant \textit{bargaining power} $\rightarrow$ limit ability of merged entity to exert market power
\\$\square$ 'Guess exercise' for authorities : need to predict effects before mergers happen in order to choose if authorize - block - impose remedies (concerning specific segments where negative eff. are confidently expected)
\\Mergers decisions in EU 1990 - 2022 : mostly authorized (7796), remedies much more frequent than block (490 vs 32)
\\~\\\textbf{Giphy - Facebook (UK)} : firms have to notify authorities in all country in which they operate $\rightarrow$ can op. as merged only where green-lighted 

\paragraph{Start-ups acquisitions / mergers} often go undetected ! (see thresholds) 
\\$\boxed{:)}$ financial stability, innovation scaled-up, products integrated in richer, better functioning platforms, new functionalities, highly skilled stuff
\\$\boxed{:(}$ for SU: less freedom, less incentive towards innovation // for market : \textit{direct} prevention of potential threat (in market with rapid technological progress) $\rightarrow$ \textit{Killer acquisitions} (Microsoft), \textit{Zombie acq.} (kept 'alive') or \textit{indirect} : weakening direct competitors by acquiring companies in complementary market
\\Even if not competitors (conglomerate) : SU's product may fit into ecosystem (e.g. Fitbit - Google) or complements may become substitutes (e.g. Instagram, from photo app to social medium)
\\~\\NE $\rightarrow$ Market tipping $\rightarrow$ competition between dominant firms \textit{for} the market (not \textit{in} anymore) - for dominance
\\\textit{Advertising-financed} platforms rely on consumer \textit{attention} $\rightarrow$ mergers may help change size + composition of audience and also allow multi-homing across platforms $\rightarrow$ siphoning (capturing) attention = foreclose entry

\paragraph{Conclusions} Need for reinforced merger rules : ex ante regime for firms with \textit{strategic market status}
\\Lesson from Facebook-Insta : importance of likehood of possibile counterfactual for ToH !
\end{document}
