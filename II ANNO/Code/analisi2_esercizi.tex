\documentclass[10pt, oneside]{article}
\usepackage[utf8]{inputenc}
\usepackage{graphicx} % Required for inserting images
\usepackage{amsmath}
\usepackage{bm}
\usepackage{esint}
\usepackage{amssymb}
\usepackage[a4paper,left=2.1cm, right=2.1cm, top=2cm, bottom=2cm]{geometry}
\usepackage{verbatim}
\usepackage[english]{babel}
\usepackage{hyperref}
\usepackage{tikz}
\usetikzlibrary{decorations.pathreplacing,calligraphy}
\usetikzlibrary { decorations.pathmorphing, decorations.shapes, }
\usepackage[cal=dutchcal]{mathalfa}
\usepackage{wrapfig}
\usepackage{physics}
\usepackage{listings}
\usepackage{comment}
\renewcommand{\rmdefault}{cmss}
\renewcommand{\arraystretch}{2}

\lstdefinestyle{mystyle}{	basicstyle=\ttfamily\footnotesize
}

\lstset{style=mystyle}

\newcommand{\limit}[2]{\lim\limits_{#1 \rightarrow #2}}

\newcommand{\infobox}[2]{\vspace{0.5cm}~\\ \textbf{#1} \hrulefill \vspace{0.2cm}\\#2 {}\,\\\hrule \vspace{0.5cm}}
\newcommand{\lawbox}[2]{\begin{center}
\framebox{
\parbox{\linewidth}{
\vspace{0.3cm}
\textbf{#1} \hfill $\displaystyle #2$
\vspace{0.3cm}
}
}
\end{center}}

\newcommand{\lawboxtext}[2]{\begin{center}
\framebox{
\parbox{\linewidth}{
\vspace{0.3cm}
\textbf{#1} \vspace{0.1cm} \\#2
\vspace{0.3cm}
}
}
\end{center}}


\title{Prontuario algoritmico per la prova di esercizi - Analisi II}
\author{Alberto Zaghini}
\date{2023}

\makeindex

\begin{document}

\maketitle

\section{Esercizio derivata direzionale}

Sono dati $f \, : \, \mathbb{R}^3 \rightarrow \mathbb{R}$, un insieme $\Gamma = \{(x,y,z) \in \mathbb{R}^3 \, : \, g(x,y,z) = 0\}$, un punto $P \equiv (x_P, y_P, z_P) \in \Gamma$ ed una condizione sul versore normale $\hat{\nu}$ in $P$.

\begin{enumerate}

\item Si calcola $\displaystyle \grad g = \big(
\pdv[•]{g}{x}, \pdv[•]{g}{y}, \pdv[•]{g}{z} \big)$

\item Si verifica che $\Gamma$ è una varietà regolare, ovvero che $\nexists (x,y,z) \in \Gamma$ t.c. $\grad g (x,y,z) = \mathbf{0}$

\item Si calcola il versore normale normalizzando il gradiente e imponendo la condizione data: $\displaystyle \hat{\nu} = \pm \frac{\grad g (P)}{\| \grad g (P) \|}$

\item Si calcola il gradiente di $f$ in $P$: $\displaystyle \grad f (P) = \big(
\pdv[•]{f}{x}, \pdv[•]{f}{y}, \pdv[•]{f}{z} \big) (P)$

\item Si calcola la derivata direzionale rispetto a $\hat{\nu}$ secondo $\displaystyle \pdv[•]{f}{\hat{\nu}} (P) = \langle \grad f(P), \hat{\nu} \rangle$

\end{enumerate}

\section{Esercizio punti critici}

\'E data una funzione $f \, : \, \mathbb{R}^3 \rightarrow \mathbb{R}$

\begin{enumerate}

\item Si determina $\displaystyle \grad f = \big(
\pdv[•]{f}{x}, \pdv[•]{f}{y}, \pdv[•]{f}{z} \big)$ e $\displaystyle H_f = \begin{pmatrix}
\displaystyle \pdv[2]{f}{x} & \displaystyle \pdv{f}{x}{y} & \displaystyle \pdv[•]{f}{x}{z} \\
\displaystyle \pdv[]{f}{y}{x} & \displaystyle \pdv[2]{f}{y} & \displaystyle \pdv[•]{f}{y}{z} \\
\displaystyle \pdv[]{f}{z}{x} & \displaystyle \pdv{f}{z}{y} & \displaystyle \pdv[2]{f}{z} \\
\end{pmatrix}$

(verifica: simmetrico se $f \, \in \, \mathcal{C}^2$)

\item Si impone $\grad f = \mathbf{0}$ e si determinano i punti (gli insiemi di pp.) che sono soluzione del sistema

\item Si studia la segnatura dell'Hessiano nei punti (siano $\lambda_i$ gli autovalori)

\begin{table}
\centering
\begin{tabular}{c|c|c}
Segnatura & Autovalori & punto critico\\
\hline
\textbf{Definita positiva} & $\lambda_i > 0 \quad \forall \lambda_i$ & di \textbf{minimo} locale\\
\textbf{Definita negativa} & $\lambda_i < 0 \quad \forall \lambda_i$  & di \textbf{massimo} locale\\
\textbf{Indefinita} & $\exists \lambda_i, \lambda_j \, : \, \lambda_i > 0 \, \land \, \lambda_j < 0$ & di \textbf{sella}\\
\textbf{Semidefinita positiva} & $\lambda_i \geq 0 \, \forall \lambda_i \, \land \, \exists \lambda_j = 0$  & di \textbf{minimo} locale o \textbf{sella}\\
\textbf{Semidefinita negativa} & $\lambda_i \leq 0 \, \forall \lambda_i \, \land \, \exists \lambda_j = 0$  & di \textbf{massimo} locale o \textbf{sella}

\end{tabular}
\end{table}

Si può applicare il criterio di Sylvester

\lawboxtext{Criterio di Sylvester (caso 2x2)}{
Sia 
\[\mathrm{Mat}_{2x2}(\mathbb{R}) \owns A = \begin{bmatrix}
a_{11} & a_{12}\\ a_{21} & a_{22}
\end{bmatrix} \neq \begin{bmatrix}
0 & 0\\ 0 & 0
\end{bmatrix}\]
Allora
\begin{itemize}
\item $\det A > 0$ $\Leftrightarrow$ $A$ definita (positiva o negativa)
\item $\det A = 0$ $\Leftrightarrow$ $A$ semidefinita (positiva o negativa)
\item $\det A < 0$ $\Leftrightarrow$ $A$ indefinita
\item $A$ definita positiva $\Leftrightarrow$ $\det A > 0$ e $a_{11} > 0$
\item $A$ definita negativa $\Leftrightarrow$ $\det A > 0$ e $a_{11} < 0$
\item $A$ semidefinita positiva $\Leftrightarrow$ $\det A = 0$ e $a_{11}, a_{22} \geq 0$
\item $A$ semidefinita negativa $\Leftrightarrow$ $\det A = 0$ e $a_{11}, a_{22} \leq 0$
\end{itemize}
}

\lawboxtext{Criterio di Sylvester (generale)}{
Sia $A \, \in \, \mathrm{Mat}_{nxn}(\mathbb{R})$ e sia $A_k$ il determinante del minore principale di ordine $k$ ($0 < k \leq n$)

\[A_k = \det \begin{bmatrix}
a_{11} & \cdots & a_{1k}\\
\vdots & \ddots & \vdots\\
a_{k1} & \cdots & a_{kk}
\end{bmatrix}\]

Allora
\begin{itemize}
\item $A$ definita positiva $\Leftrightarrow$ $A_k > 0$ $\forall k = 1, ..., n$
\item $A$ definita negativa $\Leftrightarrow$ $(-1)^k A_k > 0$ $\forall k = 1, ..., n$
\end{itemize}
}

\item Nel caso di Hessiano semidefinito, si verifica la natura del punto critico (o di un certo insieme di pp. cc.) dalla definizione

\lawboxtext{Punto di minimo \big[massimo\big] locale}{$\mathbf{x}_0$ se 
\[\exists \delta > 0 \, : \, f(\mathbf{x}) \geq f(\mathbf{x}_0) \,\, \big[ \, f(\mathbf{x}) \leq f(\mathbf{x}_0) \, \big] \quad \forall \mathbf{x} \, \in \, B_\delta(\mathbf{x}_0)\]
}

\lawboxtext{Punto critico di sella}{$\mathbf{x}_0$ se è punto critico (grad nullo) e
\[\forall \varepsilon > 0 \quad \exists \mathbf{x}_1, \mathbf{x}_2 \, \in \, B_\varepsilon(\mathbf{x}_0) \, : \, f(\mathbf{x}_1) > f(\mathbf{x}_0) > f(\mathbf{x}_2)\]
}

\end{enumerate}




\section{Esercizio estremanti vincolati}

\'E data una funzione $f \, : \, \mathbb{R}^3 \rightarrow \mathbb{R}$ e un insieme $A = \{ (x,y,z) \in \mathbb{R}^3 \, | \, g(x,y,z) = 0 \}$

\begin{enumerate}

\item Si calcola $\grad g$ e si verifica che $A$ è varietà regolare (formalmente anche che sia compatto e dunque per Weierstrass $f(A)$ intervallo chiuso e limitato)

\item Si costruisce la Lagrangiana $\displaystyle F(x,y,z,\lambda) = f(x,y,z) - \lambda g(x,y,z)$ e si impone $\displaystyle \grad F = \mathbf{0} \, \Longrightarrow \, \begin{cases}
\displaystyle \pdv[•]{f}{x} = \lambda \pdv[•]{g}{x} \\
\displaystyle \pdv[•]{f}{y} = \lambda \pdv[•]{g}{y} \\
\displaystyle \pdv[•]{f}{z} = \lambda \pdv[•]{g}{z} \\
\displaystyle g(x,y,z) = 0
\end{cases}$

\item Si calcola $f$ nei punti e si determina $f(A) = \big[\min\limits_{A} f, \max\limits_{A} f\big]$

\end{enumerate}

\section{Esercizio volume: teoremi di riduzione}

\'E dato un compatto (dunque misurabile con misura finita) $A \subseteq \mathbb{R}^3$.

\lawboxtext{Sui rettangoli}{Sia $K = [a,b] \times [c,d]$ e $f \, \in \, \mathcal{C}^0(K, \mathbb{R})$. Allora $G(y) = \int_a^b f(x,y) \dd[•]{x}$ e $F(x) = \int_c^d f(x,y) \dd[•]{y}$ sono continue e

\[\iint\limits_K f(x,y) \dd[•]{x} \dd[•]{y} = \int_{c}^{d} \bigg(\int_{a}^{b}f(x,y)\dd[•]{x}\bigg) \mathrm{d}y = \int_{a}^{b} \bigg(\int_{c}^{d}f(x,y)\dd[•]{y}\bigg) \mathrm{d}x\]

Se si può esprimere $f(x,y) = g(x) \cdot h(y)$ allora

\[\iint\limits_K f(x,y) \dd[•]{x} \dd[•]{y} = \iint\limits_K g(x) \, h(y) \dd[•]{x} \dd[•]{y} = \bigg(\int_a^b g(x) \dd[•]{x}\bigg) \cdot \bigg(\int_c^d h(y) \dd[•]{y}\bigg)\]

}

\lawboxtext{Doppio su dominio normale rispetto ad un asse}{Se $\phi, \psi \, \in \, \mathcal{C}^0 ([c,d], \mathbb{R})$ t.c. $\phi(y) \leq \psi(y)$ $\forall y \, \in \, [c,d]$,
\[A = \{(x,y) \in \mathbb{R} \times [c,d] \, | \, \phi(y) \leq x \leq \psi(y)\}\]
è normale rispetto all'asse $x$ e vale
\[\mu_2(A) = \int_c^d [\psi(y) - \phi(y)] \dd[•]{y}\]
Se $f \, \in \, \mathcal{C}^0(A, \mathbb{R})$
\[\iint\limits_A f(x,y) \dd[•]{x} \dd[•]{y} = \int_c^d \dd[•]{y} \bigg[\int_{\phi(y)}^{\psi(y)} f(x,y) \dd[•]{x}\bigg]\]
}

\lawboxtext{Triplo su dominio normale rispetto ad un asse}{Se $K \subseteq \mathbb{R}^2$ compatto misurabile e $\phi, \psi \, \in \, \mathcal{C}^0 (K, \mathbb{R})$ t.c. $\phi(x,y) \leq \psi(x,y)$ $\forall (x,y) \, \in \, K$,
\[A = \{(x,y,z) \in K \times \mathbb{R} \, | \, \phi(x,y) \leq z \leq \psi(x,y)\}\]
$A$ è normale rispetto all'asse $z$ e vale
\[\mu_3(A) = \int\limits_K \dd[•]{x} \dd[•]{y} [\psi(x,y) - \phi(x,y)]\]
Se $f \, \in \, \mathcal{C}^0(A, \mathbb{R})$
\[\iint\limits_A f(x,y,z) \dd[•]{x} \dd[•]{y} \dd[•]{z} = \int\limits_K \dd[•]{x} \dd[•]{y} \bigg[\int_{\phi(x,y)}^{\psi(x,y)} f(x,y,z) \dd[•]{z}\bigg]\]
}

Se $K$ a sua volta normale rispetto a $y$, ovvero esistono $g, h \, \in \, \mathcal{C}^0([a,b], \mathbb{R}$ t.c. $g(x) \leq h(x)$ $\forall x \, \in \, [a,b]$ e $K = \{(x,y) \, \in \, [a,b] \times \mathbb{R} \, | \, g(x) \leq y h(x)\}$ applicando nuovamente si ha

\[\iint\limits_A f(x,y,z) \dd[•]{x} \dd[•]{y} \dd[•]{z} =\int_a^b \dd[•]{x} \bigg[\int_{g(x)}^{h(x)} \dd[•]{y} \big(\int_{\phi(x,y)}^{\psi(x,y)} f(x,y,z) \dd[•]{z}\big)\bigg]\]

\lawboxtext{Teorema di Cavalieri}{Se $K \subset \mathbb{R}^3$ compatto solido di Cavalieri rispetto all'asse $z$, ovvero 
\[sez_z(K) \in \mathbb{J}(\mathbb{R}^3) \, \forall z \, \in \, [a,b] \quad \textrm{e} \quad sez_z(K) = 0 \, \forall z \, \in \, ]-\infty, a[ \, \cup \,  ]b, +\infty[ \]
allora
\[\mu_3(K) = \int_a^b \dd[•]{z} \mu_2 \big(sez_z(K)\big) = \int_a^b \dd[•]{z} \bigg[\iint\limits_{sez_z(K)} \dd[•]{x} \dd[•]{y}\bigg]\]
Inoltre se $f \, \in \, \mathcal{C}^0(K, \mathbb{R})$ vale
\[\iiint\limits_K f(x,y,z) \dd[•]{x} \dd[]{y} \dd[•]{z} = \int_a^b \dd[•]{z} \bigg[\iint\limits_{sez_z(K)} f(x,y,z) \dd[•]{x} \dd[•]{y}\bigg]\]
}

\lawboxtext{Cambiamento di variabile}{Se $A \subseteq \mathbb{R}^n$ aperto e $\Phi \, \in \mathcal{C}^1(A, \mathbb{R})$ iniettiva e t.c. $\det J_\Phi (\mathbf{u}) \neq 0$ $\forall \mathbf{u} \, \in \, A$ (o solamente su sottoinsiemi a misura nulla!). Allora se $K \subseteq A$ compatto misurabile lo è anche $\phi(K)$ e
\[\mu_n(\Phi(K)) = \iint \cdots \int\limits_{\hspace{-1cm} K} \big|\det J_\Phi(\mathbf{u})\big| \dd[•]{u_1} \cdots \dd[•]{u_n}\]
Se $f \, \in \, \mathcal{C}^0(\Phi(K), \mathbb{R})$ vale
\[\iint \cdots \int\limits_{\hspace{-1.1cm} \Phi(K)} f(\mathbf{x}) \dd[•]{x_1} \cdots \dd[•]{x_n} = \iint \cdots \int\limits_{\hspace{-1cm} K} (f \circ \Phi) (\mathbf{u}) \big|\det J_\Phi(\mathbf{u})\big| \dd[•]{u_1} \cdots \dd[•]{u_n}\]
}

\paragraph{Coordinate polari piane} con riscalamento

\[\begin{cases} \displaystyle x = \frac{\rho}{a} \cos \theta \\ \\
\displaystyle y = \frac{\rho}{b} \sin \theta \end{cases}
\qquad (\rho, \theta) \, \in \, [0, +\infty[ \, \times \, [0, 2\pi] \]

Si ha perdita di iniettività su $\{(0, \theta) \, : \, \theta \, \in \, [0,2\pi]\}$

\[J_\Phi = \begin{pmatrix}
\displaystyle \frac{\cos \theta}{a} & \displaystyle - \frac{\rho}{a} \sin \theta \\
\displaystyle \frac{\sin \theta}{b} & \displaystyle \frac{\rho}{b} \cos \theta \\
\end{pmatrix} \qquad \det J_\Phi = \frac{\rho}{ab}\]

\paragraph{Coordinate cilindriche} con riscalamento

\[\begin{cases} \displaystyle x = \frac{\rho}{a} \cos \theta \\ \\
\displaystyle y = \frac{\rho}{b} \sin \theta \\ \\
\displaystyle z = h
\end{cases}
\qquad (\rho, h, \theta) \, \in \, [0, +\infty[ \, \times \, \mathbb{R} \, \times \, [0, 2\pi] \]

Si ha perdita di iniettività su $\{(0, h, \theta) \, : \, (h, \theta) \, \in \, \mathbb{R} \, \times \, [0,2\pi]\}$

\[J_\Phi = \begin{pmatrix}
\displaystyle \frac{\cos \theta}{a} & \displaystyle - \frac{\rho}{a} \sin \theta & \displaystyle 0\\
\displaystyle \frac{\sin \theta}{b} & \displaystyle \frac{\rho}{b} \cos \theta & \displaystyle 0\\
\displaystyle 0 & \displaystyle 0 & \displaystyle 1
\end{pmatrix} \qquad \det J_\Phi = \frac{\rho}{ab}\]

\paragraph{Coordinate sferiche} con riscalamento

\[\begin{cases} \displaystyle x = \frac{\rho}{a} \sin \theta \cos \varphi\\ \\
\displaystyle y = \frac{\rho}{b} \sin \theta \sin \varphi\\ \\
\displaystyle z = \frac{\rho}{c} \cos \theta
\end{cases}
\qquad (\rho, \theta, \phi) \, \in \, [0, +\infty[ \, \times \, [0, \pi] \, \times \, [0, 2\pi] \]

Si ha perdita di iniettività su $\{(0, \theta, \varphi) \, : \, (\theta, \varphi) \, \in \, [0, \pi] \, \times \, [0,2\pi]\}$, su $\{(\rho, 0, \varphi) \, : \, (\rho, \varphi) \, \in \, [0, +\infty[ \, \times \, [0,2\pi]\}$ e $\{(\rho, \pi, \varphi) \, : \, (\rho, \varphi) \, \in \, [0, +\infty[ \, \times \, [0,2\pi]\}$

\[J_\Phi = \begin{pmatrix}
\displaystyle \frac{1}{a} \sin \theta \cos \varphi & \displaystyle \frac{\rho}{a} \cos \theta \cos \varphi & \displaystyle - \frac{\rho}{a} \sin \theta \sin \varphi\\
\displaystyle \frac{1}{b} \sin \theta \sin \varphi & \displaystyle \frac{\rho}{b} \cos \theta \sin \varphi & \displaystyle \frac{\rho}{b} \sin \theta \cos \varphi\\
\displaystyle \frac{\cos \theta}{c} & \displaystyle - \frac{\rho}{c} \sin \theta & \displaystyle 0 \\
\end{pmatrix} \qquad \det J_\Phi = \frac{\rho^2}{abc} \sin \theta\]

\section{Esercizio rotore}

\'E data una funzione $\mathbf{f} \, \in \, \mathcal{C}^1(\mathbb{R}^3, \mathbb{R}^3)$ ed una superficie regolare $\Sigma$ orientata con una condizione sull'orientamento $\hat{\nu}$

\subsection{Calcolo diretto}

\begin{enumerate}

\item Si calcola il rotore di $\mathbf{f}$ secondo 
\[\mathrm{rot} \mathbf{f} = \begin{vmatrix}
\hat{i} & \hat{j} & \hat{k} \\
\displaystyle \pdv[•]{•}{x} & \displaystyle \pdv[•]{•}{y} & \displaystyle \pdv[•]{•}{z} \\
\displaystyle F_x & \displaystyle F_y & \displaystyle F_z
\end{vmatrix} 
= \hat{i} \big(\pdv[•]{F_z}{y} - \pdv[•]{F_y}{z}\big) + \hat{j} \big(\pdv[•]{F_x}{z} - \pdv[•]{F_z}{x}\big) + \hat{k} \big(\pdv[•]{F_y}{x} - \pdv[•]{F_x}{y}\big)
\]

\item Si definisce una parametrizzazione $\mathbf{r} \, : \, \mathbb{R}^2 \supseteq \overline{\Omega} \rightarrow \Sigma$ e si esprime il rotore in funzione delle nuove variabili

\item Si determina l'orientamento indotto 
\[\mathbf{n}(u, v) = \begin{vmatrix}
\hat{i} & \hat{j} & \hat{k} \\
\displaystyle \pdv[•]{r_1}{u} & \displaystyle \pdv[•]{r_2}{u} & \displaystyle \pdv[•]{r_3}{u} \\
\displaystyle \pdv[•]{r_1}{v} & \displaystyle \pdv[•]{r_2}{v} & \displaystyle \pdv[•]{r_3}{v} \\
\end{vmatrix} = \hat{i} \big(\pdv[•]{r_2}{u} \pdv[•]{r_3}{v} - \pdv[•]{r_3}{u} \pdv[•]{r_2}{v}\big) + \hat{j} \big(\pdv[•]{r_3}{u} \pdv[•]{r_1}{v} - \pdv[•]{r_1}{u} \pdv[•]{r_3}{v}\big) + \hat{k} \big(\pdv[•]{r_1}{u} \pdv[•]{r_2}{v} - \pdv[•]{r_2}{u} \pdv[•]{r_1}{v}\big)
\]

e si verifica la compatibilità con $\hat{\nu}$

\item A seconda della compatibilità (ok o opposto) si calcola il flusso secondo:

\[\iint\limits_\Sigma \langle \mathrm{rot} \mathbf{f}, \hat{\nu} \rangle \dd[•]{\sigma} = \pm \iint\limits_{\overline{\Omega}} \langle \mathrm{rot} \mathbf{f} \big(\mathbf{r}(u,v)\big), \mathbf{n}(u,v) \rangle \dd[•]{u} \dd[•]{v}\]

\end{enumerate}

\subsection{Teorema di Stokes}

\lawboxtext{Teorema}{Sia $A$ aperto di $\mathbb{R}^3$ e $\mathbf{f} \, \in \, \mathcal{C}^1(A, \mathbb{R}^3)$ e sia $\Sigma \subseteq A$ superficie regolare con bordo con orientamento $\hat{\nu}$. Se $(\partial \Sigma, \hat{\tau})$ è il suo bordo con orientamento indotto canonicamente vale

\[\iint\limits_\Sigma \langle \mathrm{rot} \mathbf{f}, \hat{\nu} \rangle \dd[•]{\sigma} = \int\limits_{\partial\Sigma} \langle \mathbf{f}, \hat{\tau} \rangle \dd[•]{s}\]

}

\begin{enumerate}

\item Si determina il bordo di $\Sigma$ e l'orientamento indotto $\hat{\tau}$

\item Per ogni componente connessa $\partial \Sigma_i$ si definisce una parametrizzazione $\bm{\rho}^i \, \in \mathcal{C}^1([a_i, b_i], \mathbb{R}^3)$

\item Si determina l'orientamento indotto da ciascuna sul rispettivo sostegno

\[\dv[•]{\bm{\rho}^i}{t} = \big(\dv[•]{\rho^i_1}{t}, \dv[•]{\rho^i_2}{t}, \dv[•]{\rho^i_3}{t}\big)\]

\item In base alla compatibilità con $\hat{\tau}$, si calcola il lavoro su ogni curva

\[\int\limits_{\partial\Sigma_i} \langle \mathbf{f}, \hat{\tau} \rangle \dd[•]{s} = \pm \int_{a_i}^{b_i} \langle \mathbf{f}(\bm{\rho}^i(t), \dv[•]{\bm{\rho}^i}{t}(t) \rangle \dd[•]{t}\]

\item Il lavoro complessivo dà quindi il flusso secondo

\[\iint\limits_\Sigma \langle \mathrm{rot} \mathbf{f}, \hat{\nu} \rangle \dd[•]{\sigma} = \int\limits_{\partial\Sigma} \langle \mathbf{f}, \hat{\tau} \rangle \dd[•]{s} = \sum_i \int\limits_{\partial\Sigma_i} \langle \mathbf{f}, \hat{\tau} \rangle \dd[•]{s} = \sum_i \pm \int_{a_i}^{b_i} \langle \mathbf{f}(\bm{\rho}^i(t), \dv[•]{\bm{\rho}^i}{t} \rangle \dd[•]{t}\]

\end{enumerate}

\section{Esercizio divergenza}
\'E data una funzione $\mathbf{f} \, \in \, \mathcal{C}^1(\mathbb{R}^3, \mathbb{R}^3)$ ed alternativamente un insieme $A$ (con misura $\mu_3$ non nulla) o una superficie regolare $\Sigma$ con un dato orientamento.

\subsection{Calcolo diretto}
\begin{enumerate}

\item Se si richiede il flusso di $f$ attraverso $\Sigma$ il procedimento è analogo a quanto visto per il rotore.
\\Senza ripetere i passaggi si arriva a:

\[\iint\limits_\Sigma \langle \mathbf{f}, \hat{\nu} \rangle \dd[•]{\sigma} = \pm \iint\limits_{\overline{\Omega}} \langle \mathbf{f} \big(\mathbf{r}(u,v)\big), \mathbf{n}(u,v) \rangle \dd[•]{u} \dd[•]{v}\]

\item Se è richiesto l'integrale della divergenza si può applicare il Teorema e calcolare il flusso di $\mathbf{f}$ attraverso opportune superfici regolari orientabili che unite a $\Sigma$ diano una superficie regolare a tratti chiusa (per praticità denotata con $\partial A$).
\\Tenere conto della compatibilità degli orientamenti indotti dalle parametrizzazioni con quello \textbf{esterno}!

\end{enumerate}

\subsection{Teorema della Divergenza}

\lawboxtext{Teorema}{Sia $A \subseteq \mathbb{R}^3$ aperto regolare e $\mathbf{f} \, \in \, \mathcal{C}^1(\overline{A}, \mathbb{R}^3)$ e sia $(\partial A, \hat{\nu})$ la frontiera di $A$ orientata canonicamente. Allora

\[\iiint\limits_A \mathrm{div} \mathbf{f} \dd[•]{x} \dd[•]{y} \dd[•]{z} = \iint\limits_{\partial A} \langle \mathbf{f}, \hat{\nu} \rangle \dd[•]{\sigma}\]

}

\begin{enumerate}

\item Se necessario, si effettua un opportuno cambiamento di variabile:

\[\iiint\limits_A \mathrm{div} \mathbf{f} \dd[•]{x} \dd[•]{y} \dd[•]{z} = \iiint\limits_{\Phi^{-1}(A)} \mathrm{div} \mathbf{f}(\Phi(u,v,t)) \big|\det J_\Phi(u,v,t)\big| \dd[•]{u} \dd[•]{v} \dd[•]{t}\]

\item Si applicano i teoremi di riduzione

\item Se è richiesto il flusso attraverso una componente della frontiera $\partial A$ (con un certo dato orientamento) si calcola il flusso attraverso le altre componenti regolari orientate \textbf{verso l'esterno} e si ottiene quello cercato per differenza - a meno del segno \textbf{da determinarsi secondo la compatibilità}.

\end{enumerate}


\end{document}