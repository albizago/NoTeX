\documentclass[10pt, oneside]{article}
\usepackage[utf8]{inputenc}
\usepackage{graphicx} % Required for inserting images
\usepackage{amsmath}
\usepackage{esint}
\usepackage{amssymb}
\usepackage[a4paper,left=2.1cm, right=2.1cm, top=2cm, bottom=2cm]{geometry}
\usepackage{verbatim}
\usepackage[english]{babel}
\usepackage{hyperref}
\usepackage{tikz}
\usetikzlibrary{decorations.pathreplacing,calligraphy}
\usetikzlibrary { decorations.pathmorphing, decorations.shapes, }
\usepackage[cal=dutchcal]{mathalfa}
\usepackage{wrapfig}
\usepackage{physics}
\usepackage{listings}
\usepackage{comment}
\renewcommand{\rmdefault}{cmss}

\lstdefinestyle{mystyle}{	basicstyle=\ttfamily\footnotesize
}

\lstset{style=mystyle}

\newcommand{\limit}[2]{\lim\limits_{#1 \rightarrow #2}}

\newcommand{\infobox}[2]{\vspace{0.5cm}~\\ \textbf{#1} \hrulefill \vspace{0.2cm}\\#2 {}\,\\\hrule \vspace{0.5cm}}
\newcommand{\lawbox}[2]{\begin{center}
\framebox{
\parbox{\linewidth}{
\vspace{0.3cm}
\textbf{#1} \hfill $\displaystyle #2$
\vspace{0.3cm}
}
}
\end{center}}

\newcommand{\lawboxtext}[2]{\begin{center}
\framebox{
\parbox{\linewidth}{
\vspace{0.3cm}
\textbf{#1} \vspace{0.1cm} \\#2
\vspace{0.3cm}
}
}
\end{center}}


\title{Prontuario algoritmico per la prova di esercizi - Analisi II}
\author{Alberto Zaghini}
\date{2023}

\makeindex

\begin{document}

\maketitle

\section{Esercizio derivata direzionale}

Sono dati $f \, : \, \mathbb{R}^3 \rightarrow \mathbb{R}$, un insieme $\Gamma = \{(x,y,z) \in \mathbb{R}^3 \, : \, g(x,y,z) = 0\}$, un punto $P \equiv (x_P, y_P, z_P) \in \Gamma$ ed una condizione sul versore normale $\hat{\nu}$ in $P$.

\begin{enumerate}

\item Si calcola $\displaystyle \grad g = \big(
\pdv[•]{g}{x}, \pdv[•]{g}{y}, \pdv[•]{g}{z} \big)$

\item Si verifica che $\Gamma$ è una varietà regolare, ovvero che $\nexists (x,y,z) \in \Gamma$ t.c. $\grad g (x,y,z) = \mathbf{0}$

\item Si calcola il versore normale normalizzando il gradiente e imponendo la condizione data: $\displaystyle \hat{\nu} = \pm \frac{\grad g (P)}{\| \grad g (P) \|}$

\item Si calcola il gradiente di $f$ in $P$: $\displaystyle \grad f (P) = \big(
\pdv[•]{f}{x}, \pdv[•]{f}{y}, \pdv[•]{f}{z} \big) (P)$

\item Si calcola la derivata direzionale rispetto a $\hat{\nu}$ secondo $\displaystyle \pdv[•]{f}{\hat{\nu}} (P) = \langle \grad f(P), \hat{\nu} \rangle$

\end{enumerate}

\section{Esercizio punti critici}

\section{Esercizio estremanti vincolati}

\section{Esercizio volume: teoremi di riduzione}

\section{Esercizio flusso}

\subsection{Calcolo diretto}

\subsection{Teorema di Stokes}

\section{Esercizio divergenza}

\subsection{Calcolo diretto}

\subsection{Teorema della Divergenza}



\end{document}