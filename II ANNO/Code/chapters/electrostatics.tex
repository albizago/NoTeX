Lo studio dei fenomeni elettrici e magnetici ha le sue radici nell'antichità. In particolare è in Grecia che in concomitanza con la nascita della filosofia si assiste ad una prima indagine protoscientifica delle manifestazioni di interazioni di origine non gravitazionale.
\\Le prime esperienze di repulsione elettrostatica si hanno sfregando con della seta pezzi di ambra, in greco \selectlanguage{greek}ἤλεκτρον\selectlanguage{italian} (\textit{elektron}) - da cui il termine \textit{elettricità}.
\\Altre esperienze che complessivamente permettono una prima caratterizzazione della nuova forza elettrostatica sono le seguenti:
\begin{itemize}
\item Prima di eseguire alcuna operazione, avvicinando due bacchette di vetro o plastica non si ha alcuna interazione
\item Sfregando due bacchette di \textbf{vetro} con panni di \textbf{seta} queste si respingono
\item Sfregando due bacchette di \textbf{plastica} con panni di \textbf{pelle} queste si respingono
\item Caricando analogamente una bacchetta di vetro e una di plastica queste si respingono
\end{itemize}
Si ha dunque il fenomeno dell'\textbf{elettrificazione per strofinamento} o \textbf{triboelettricità}. L'utilizzo di un dinamometro permette di quantificare la forza esercitata.
\\Tale forza dunque
\begin{enumerate}
\item agisce a distanza (come la gravità)
\item può essere sia attrattiva che repulsiva (diversamente dalla g.)
\end{enumerate}

\subsection{Origine della triboelettricità}
Lo sfregamento porta ad uno spostamento di cariche microscopiche tra i materiali, con conseguente accumulo di una carica elettrica netta che dà luogo ad interazione. Per i costituenti atomici si ha
\begin{table}[h!]
\centering
\begin{tabular}{c c c c}
& M(kg) & d(m) & q(C)\\
p & $1,7 \times 10^{-27}$ & $10^{-15}$ & $1,6 \times 10^{-19}$ \\
n & $1,7 \times 10^{-27}$ & $10^{-15}$ & $0$ \\
e & $9,1 \times 10^{-31}$ & $10^{-18}$ & $-1,6 \times 10^{-19}$
\end{tabular}
\end{table}

Strofinando il vetro con la seta, \textbf{cariche negative si spostano dal vetro}, che rimane dunque caricato positivamente, \textbf{alla seta}. Viceversa strofinando la plastica con la pelle le cariche negative sono \textbf{asportate dalla pelle e depositate sulla plastica}. Il trasferimento delle cariche è dunque un processo di natura meccanica.

\section{La carica elettrica}
Si è dunque determinata una nuova proprietà fondamentale della materia, la \textbf{carica elettrica}. Questa può essere assente (come prima dello strofinamento) o presente in due tipologie. Convenzionalmente si assegna il \textbf{segno positivo o negativo}. Valgono le seguenti osservazioni:
\begin{itemize}
\item Cariche di segno uguale si respingono
\item Cariche di segno opposto si attraggono
\end{itemize}

\section{Strumenti di misura}
Lo strumento più rudimentale è \textbf{l'elettroscopio a foglie}. Questo permette di determinare semplicemente se due corpi posseggono la stessa carica, non di effettuare misure quantitative. Infatti quando viene avvicinata alla sfera metallica un corpo carico, questo induce una carica di segno opposto su di essa e conseguentemente l'accumulo di cariche del medesimo segno sulle foglie (vedremo perché trattando i conduttori), che si respingono. L'angolo di repulsione è proporzionale alla carica depositata sulle foglie, e dunque a quella del corpo (indipendentemente dal segno!).
\\\'E in realtà possibile sviluppare una prima scala di misura per la carica utilizzando cariche frazionarie e determinando la proporzionalità con l'angolo.

\section{La legge di Coulomb}
Nel 1785 Coulomb svolse una serie di esperienze cruciali di elettrostatica. Si trattò di studi sistematici dell'interazione tra cariche utilizzando una bilancia di torsione cui era appesa ad una estremità una sfera metallica carica ed all'altra un contrappeso.
\\Avvicinando alla sfera un'altra carica, si osservava una torsione fino ad una nuova posizione di equilibrio, con il pendolo ad un angolo $\theta$ rispetto a quella iniziale.
\infobox{Calcolare la forza esercitata}{
Imponendo le condizioni di equilibrio statico (I principio della statica):
\[ \begin{cases}
\sum \vec{F} = 0 \\ \sum \vec{\mathcal{M}} = 0
\end{cases} \]
Si noti che la prima condizione permette di scegliere un polo a piacere; sia questo il punto medio del pendolo - in cui è attaccato il filo. Il momento della forza elettrostatica vale
\[|\vec{\mathcal{M}}| = |\vec{R} \wedge \vec{F}| = R F \sin(\frac{\pi}{2} - \theta)\]
Per la forza elastica di richiamo del filo si ha $\displaystyle |\vec{\mathcal{M}}_r| \propto \theta$ da cui
\[F \frac{L}{2} \sin \varphi = c \theta \implies F = \frac{2 c \theta}{L \sin \varphi}\]
}
Le proprietà della forza scoperte da Coulomb sono
\begin{enumerate}
\item \'E diretta lungo la retta congiungente le cariche
\item \'E proporzionale al prodotto delle cariche e inversamente proporzionale al quadrato della distanza fra esse
\end{enumerate}
Ottenne così una legge empirica sul modello di quella di gravitazione universale newtoniana

\lawboxtext{Legge di Coulomb}{
Date due cariche a distanza $r$, tra di esse si esercita una forza direttamente proporzionale al prodotto delle cariche e inversamente proporzionale al quadrato di $r$. Tale forza è diretta lungo la congiungente ed è attrattiva se le cariche hanno segno opposto e repulsiva se il segno è il medesimo.
\[\vec{F} = k \frac{Q q}{R_{AB}^2} \hat{R_{AB}}\]
e dunque $\displaystyle |\vec{F}| = k \frac{|Q| |q|}{R_{AB}^2}$
}

La forza di Coulomb verifica il III principio della dinamica: due cariche esercitano reciprocamente forze uguali in modulo e contrarie in verso.

\subsection{Unità di misura}
Sono in uso due differenti sistemi: MKS (m, kg, s) e CGS (cm, g, s).
\\Nel CGS tutto è espresso in funzione delle dimensioni fondamentali della meccanica: massa, lunghezza, tempo. Si pone $k = 1$ e $[k] = [1]$ e si determina dunque l'u.m. della carica secondo
\[[Q] = [M]^{1/2} [L]^{3/2} [T] \]
Essa è detta \textbf{statcoulomb} (statc), che corrisponde alla carica di due cc. poste a 1cm tra cui si esercita una forza elettrostatica di 1 dyme ($10^{-5}$ N).
\\Nell'MKS si aggiunge una nuova dimensione fondamentale (MKSQ) con u.m. specifica, il Coulomb (C) e dunque si ha una costante non adimensionale
\[[k] = [M] [L]^3 [T]^{-2} [Q]^{-2}\]
Fino al 2019 l'unità era definita a partire dalla \textbf{corrente elettrica} (più facile da misurare), oggi si fa riferimento alla carica elementare
\[q = 1,6021766208 \times 10^{-19} \mathrm{C}\]
e per la costante di Coulomb vale
\[k = \frac{1}{4 \pi \varepsilon_0} = 8,99 \times 10^{-9} \newton \meters{•}{2} \coulombs{}{-2}\]
con la \textbf{costante dielettrica del vuoto}
\[\varepsilon_0 = 8,85 \times 10^{-12} \coulombs{}{2} \newton^{-1} \meters{•}{-2}\]

\subsection{Limiti dell'esperienza di Coulomb}
\begin{itemize}
\item Le sfere non sono cariche puntiformi, per quanto l'approssimazione possa funzionare se la loro forma è perfettamente sferica (vd teorema di Gauss o dei gusci di Newton)
\item Se si tenta fisicamente di concentrare una medesima carica in una sfera di raggio sempre minore si incorre nella dispersione della carica per repulsione elettrostatica (analogo all'effetto di dispersione delle punte)
\item Si ha ulteriore dispersione a prescindere per presenza dell'aria, e dunque la carica sulle sfere non è costante
\item \'E idealmente necessario schermare l'effetto di tutte le altre cariche dell'universo o assumere sia trascurabile
\item La legge è verificata indirettamente, come quelle che ne derivano
\end{itemize}

\subsection{Confronto con la gravità}
A livello atomico, considerando le interazioni di natura elettrostatica e gravitazionale tra protoni ed elettroni
\[F_C \sim 10^{-28} \newton \qquad F_g \sim 10^{-67} \newton \qquad \implies F_C / F_g \sim 10^{39}\]
dunque la gravità è assolutamente trascurabile su scala atomica (ed inferiori). A livello macroscopico si osservano invece maggiormente gli effetti gravitazionali i corpi sono ordinariamente \textbf{elettricamente neutri}.

\section{Proprietà della carica}
\begin{enumerate}
\item Esiste una carica minima in Natura, e di conseguenza la carica dei corpi è \textbf{quantizzata}. Questa è la carica del protone oppure, in segno opposto, dell'elettrone.
\\In realtà per la cromodinamica quantistica i quarks posseggono carica frazionaria. L'up ($u$) ha carica $+ \frac{2}{3}$ e il down ($d$) $- \frac{1}{3}$; il protone è formato da 2 up + 1 down e ha dunque carica 1, il neutrone da 2 down + 1 up è quindi neutro. Tuttavia non si sono mai osservati quark liberi e non se ne è mai misurata la carica, dunque resta valido quanto detto in precedenza.
\infobox{E se differissero?}{
\'E possibile stimare quali effetti si osserverebbero se la carica del protone e dell'elettrone fossero differenti, ovvero di conseguenza la materia ordinaria non fosse neutra. Assumendo una differenza relativa
\[\frac{|q_p| - |q_e|}{|q_p|} \approx 10^{-9} \qquad \textrm{ovvero} \quad \Delta q \approx 1,6 \times 10^{-28} \coulombs{}{}\]
Tra due sfere di ferro di 1 kg poste a distanza di 1 m si avrebbe:
\[M = 55 \quad Z = 26 \implies \Delta Q = \Delta q \cdot N_p = \Delta q \cdot \frac{m}{\mathcal{M}} \cdot Z \cdot N_A = 0,0455 \coulombs{}{} \implies F = \frac{1}{4 \pi \varepsilon_0} \frac{|\Delta Q|^2}{d^2} = 1,7 \times 10^{7} \newton\]
un valore estremamente elevato. Con tecniche più complesse attualmente il limite superiore alla differenza relativa è 
\[\frac{|q_p| - |q_e|}{|q_p|} < 10^{-21}\]
}

\item La carica \textbf{si conserva}: in un sistema isolato (ovvero che non scambia cariche con l'esterno) la c. totale è costante.
\\Questo vale anche nelle più elementari reazioni nucleari, e.g. il decadimento $\beta$
\[\underbrace{n}_{0} \rightarrow \underbrace{p + e + \overline{\nu}_e}_{1+(-1) + 0 = 0}\]
o nell'annichilazione tra elettrone e antielettrone
\[\underbrace{\overline{e} + e}_{1+(-1) = 0} \rightarrow \underbrace{\gamma + \gamma}_{0 + 0 = 0}\]

\item La carica è un invariante per cambio di sistema di riferimento (inerziale o non) e dunque indipendente dal SdR in cui viene misurata
\end{enumerate}

\section{Il principio di sovrapposizione}
Per sistemi multicarica si osserva \textbf{sperimentalmente} che la forza complessiva esercitata su ogni carica è data dalla \textbf{somma vettoriale} delle forze esercitate su di essa da ciascuna delle altre. Generalizzando:

\lawboxtext{Principio di sovrapposizione (forma discreta)}{
Dato un sistema di $N$ cariche (discrete), la forza elettrostatica risultante su una carica $q_P$ posta nel punto $P$ è data secondo
\[\vec{F}_T = \frac{1}{4 \pi \varepsilon_0} \sum\limits_{i=1}^{N} \frac{q_i}{R_{iP}^2}\hat{R_{iP}}\]
ove $\vec{R_{iP}}$ indica il vettore di posizione relativa della carica in $P$ rispetto all'$i$-esima del sistema, ovvero $\vec{R_{iP}} = \vec{R_P} - \vec{R_i}$
}

In natura si osservano più frequentemente distribuzioni di carica con un elevatissimo numero di cariche contigue e non isolate, approssimabili al cosiddetto \textbf{corpo continuo}. In tal caso si può introdurre la \textbf{densità volumetrica di carica}
\[\rho(\vec{r}) = \limit{\Delta \tau}{0} \frac{\Delta q}{\Delta \tau} = \dv[•]{q}{\tau} \implies \dd[•]{q} = \rho \dd[•]{\tau}\]
osservando che l'operazione di limite effettuata non corrisponde a livello fisico a quella matematica (che porterebbe ad una densità descritta da una Delta a causa della natura discreta della carica a livello fondamentale): il volume infinitesimo su cui si assume un valore uniforme di $\rho$ contiene comunque una carica sufficiente perché possa avere senso assegnarlo.
\\Analogamente si possono definire densità lineari e superficiali di carica:
\begin{itemize}
\item[Superficiale] \hfill $\displaystyle \sigma(\vec{r}) = \limit{\Delta S}{0} \frac{\Delta q}{\Delta S} = \dv[•]{q}{S} \quad \implies \dd[•]{q} = \sigma \dd[•]{S}$
\item[Lineare] \hfill $\displaystyle \lambda(\vec{r}) = \limit{\Delta l}{0} \frac{\Delta q}{\Delta l} = \dv[•]{q}{l} \quad \implies \dd[•]{q} = \lambda \dd[•]{l}$
\end{itemize} 

Si può così esprimere il principio in forma continua

\lawbox{Principio di sovrapposizione (forma continua)}{\vec{F}_T = \frac{1}{4 \pi \varepsilon_0} Q_P \int_\tau \rho(\vec{r}) \frac{\dd[•]{\tau}}{\Delta R^2} \hat{\Delta R}}

con $\vec{\Delta R} = \vec{r}_P - \vec{r}$

\section{Il campo elettrico}
\'E possibile esprimere la forza esercitata su una carica da un'altra anche introducendo una nuova grandezza vettoriale, definita \textbf{campo elettrico} (o più precisamente \textbf{elettrostatico}), secondo
\[\vec{F} = \frac{1}{4 \pi \varepsilon_0} \frac{Q q}{r^2} \hat{r} = q \bigg[\frac{1}{4 \pi \varepsilon_0} \frac{Q}{r^2} \hat{r}\bigg] = q \vec{E}\]
$\vec{E}$ è indipendente da $q$ ed il suo valore in un qualsiasi punto dello spazio moltiplicato per una carica in esso collocata dà la forza che agisce su questa, ovvero fisicamente $q$ \textbf{si accoppia} con il campo e l'interazione dà luogo alla forza (vd dopo per la questione della realtà fisica del campo).
\\Si tratta dunque di un campo vettoriale: $\vec{E} = \vec{E}(\vec{r}) = \vec{E}(x,y,z)$ se si pone l'origine del SdR nella carica che genera il campo. Più generalmente nel caso non stazionario il campo può avere anche dipendenza temporale $\vec{E} = \vec{E}(\vec{r}, t)$.
\\~\\Lo studio dell'elettrostatica si riduce dunque alla determinazione del valore del campo, indipendentemente dalla conoscenza della configurazione spaziale del sistema di cariche che lo genera. Trattandosi di una funzione vettoriale, essa corrisponde a tre ff. scalari.
\\A parità di carica, campi uguali producono forze uguali: dunque per misurare $\vec{E}$ si utilizzano \textbf{cariche esploratrici} (sufficientemente ridotte da non perturbare significativamente il campo, ovvero alterare le posizioni ed il moto delle sorgenti) e si misura la forza di Coulomb su di esse.

\subsection{Realtà fisica del campo}
Al termine della trattazione sarà chiaro come i campi elettromagnetici abbiano realtà propria, ovvero esistano non come meri artifici matematici ma come \textit{elementi di realtà} permeanti lo spazio in grado di interagire e dar luogo ad effetti peculiari non riconducibili alle loro sorgenti: a testimoniarlo inequivocabilmente sarà la natura della luce e la sua capacità di trasportare energia e momento.

\subsection{Sovrapposizione per il campo elettrico}
Il principio di sovrapposizione per il campo elettrico è diretta conseguenza di quello per la forza di Coulomb, in quanto la carica è una grandezza scalare e non modifica dunque la relazione vettoriale se semplificata. Per una distribuzione discreta dunque vale:
\[\vec{E}_T = \frac{1}{4 \pi \varepsilon_0} \sum\limits_{i=1}^{N} \frac{Q_i}{r_i^2}\hat{r}_i\]
e per una continua:
\[\vec{E}_T = \frac{1}{4 \pi \varepsilon_0} \int_\tau \rho(\vec{r}) \frac{\dd[•]{\tau}}{\Delta R^2} \hat{\Delta R}\]

\subsection{Le linee di forza / di campo}
\'E possibile rappresentare graficamente nello spazio la struttura del campo ricorrendo ad un utile costruzione matematica: le linee di campo. 
\\Si prenda un punto $P_1$ e si tracci il vettore $\vec{E_1}$ con la direzione ed il verso del campo nel punto ed una lunghezza proporzionale al modulo. Si percorra uno spostamento infinitesimo su tale vettore di lunghezza $\dd[•]{P}$ fino al punto $P_2$. Si tracci qui $\vec{E_2}$ in modo analogo e si ripeta la procedura; si iteri quindi per un numero di punti indefinito.
\\Congiungendo i punti si ottiene una linea che permette di rappresentare diverse informazioni sul campo sui punti che vi giacciono. Infatti
\begin{enumerate}
\item Per costruzione $\vec{E}$ è sempre tangente ad essa
\item Ha un orientamento che è dato dal verso di $\vec{E}$
\item Tracciando tutte le altre linee, la loro densità (numero per unità di volume o di area, se si considera una superficie trasversale) dà l'intensità del campo e dunque della forza esercitata su una carica posta in quella posizione
\end{enumerate}

\section{Campo del dipolo}
Un \textbf{dipolo elettrico} è definito come un sistema di due cariche di stesso modulo $q$ ma segno opposto poste ad una distanza fissa $d$. 
\\Per il campo sul piano passante per il punto medio ed ortogonale alla retta su cui giacciono le cariche, applicando il principio di sovrapposizione:
\[\vec{E_T} = \vec{E_+} + \vec{E_{-}} = \frac{1}{4 \pi \varepsilon_0} \big(\frac{q}{r_+^2} \hat{r_+} - \frac{q}{r_-^2} \hat{r_-}\big)\]
Esplicitando i versori e semplificando i contributi lungo l'asse $y$ si ha
\[\vec{E_T} = \frac{q}{4 \pi \varepsilon_0 r^2} \big(- \frac{d}{r} \hat{k}\big)\]
Se si introduce ora il \textbf{momento di dipolo elettrico}
\[\vec{P} = q d \hat{k}\]
con $\hat{k}$ versore dell'asse su cui giacciono le cariche orientato in modo che $\vec{P}$ sia diretto \textbf{dalla carica negativa a quella positiva} si ha che il campo elettrostatico generato da un dipolo elettrico lungo l'asse perpendicolare alla distanza tra le cariche vale
\[\vec{E} = - \frac{1}{4 \pi \varepsilon_0} \frac{\vec{P}}{r^3}\]
A valore costante del momento esso cala dunque come il reciproco di $r^3$: l'effetto di cancellazione delle componenti rende il decadimento più rapido del campo di una singola carica. Sostituendo alla carica negativa un'altra positiva si avrebbe invece un campo diretto lungo l'asse in verso positivo valente
\[\vec{E} = \frac{1}{4 \pi \varepsilon_0} \frac{q}{r^2} \big(\frac{2Y}{r} \hat{j}\big) = \frac{1}{2 \pi \varepsilon_0} \frac{Y}{r^3} \hat{j} \implies E \sim \frac{1}{r^3}\]
Si osserva che il momento di dipolo dipende dal prodotto tra carica e distanza e non dai due fattori indipendentemente: dunque un dipolo con cariche doppie e distanza dimezzata darà lo stesso campo sull'asse trasverso.

\section{Campo di una distribuzione lineare indefinita di cariche positive}
Si considera una distribuzione lineare di carica rettilinea infinita di densità $\lambda > 0$ uniforme. Allora dato un punto esterno e detta $r'$ la sua distanza dalla distribuzione, il contributo del campo nel punto di un elemento infinitesimo di filo è

\[\dd[•]{\vec{E}} = \frac{1}{4 \pi \varepsilon_0} \frac{\lambda \dd[•]{z}}{r^2} \hat{r}\]

ove $\vec{r}$ indica il vettore posizione relativa, fissato nell'elemento e con vertice in $P$. Applicando il principio di sovrapposizione in forma continua

\[\vec{E} = \int_{filo} \dd[•]{\vec{E}}\]

Essendo la densità lineare uniforme, la distribuzione presenta una simmetria rispetto all'asse ortogonale passante per il punto. Dunque chiaramente la componente parallela al filo del campo risultante è nulla, mentre quella trasversale lo è perché lo è quella di ciascun contributo: si ha campo risultante radiale e l'integrale si riduce dunque ad una dimensione:

\[\vec{E} = \vec{E_y} = E_y \hat{j} = \int_{-\infty}^{+\infty} \frac{1}{4 \pi \varepsilon_0} \frac{\lambda \dd[•]{z}}{r^2} \cos \theta\]

ove $\theta$ è l'angolo compreso tra il campo prodotto da ogni elemento e l'asse ortogonale, o equivalentemente tra il vettore posizione relativa e quest'ultimo. 
\\Si opera un cambio di variabile per sfruttare l'interrelazione tra le variabili nell'integrale (non indipendenti):

\[r \cos \theta = r' \implies r = \frac{r'}{\cos \theta} \qquad z = r' \tan \theta \implies \dd[•]{z} = \dv[•]{z}{\theta} \dd[•]{\theta} = \frac{r'}{\cos^2 \theta} \dd[•]{\theta}\]

da cui

\[E = \frac{\lambda}{4 \pi \varepsilon_0} \int_{-\pi/2}^{\pi/2} \cos \theta \cdot \frac{r'}{\cos^2 \theta} \cdot \frac{\cos^2 \theta}{(r')^2} \dd[•]{\theta} = \frac{\lambda}{4 \pi \varepsilon_0 r'} \underbrace{\int_{-\pi/2}^{\pi/2} \cos \theta \dd[•]{\theta}}_{1 - (-1) = 2} = \frac{\lambda}{2 \pi \varepsilon_0 r'}\]

Il campo, per quanto detto, presenta una simmetria cilindrica e decade proporzionalmente a $\frac{1}{}{r'}$ anziché $1/(r')^2$ come nel caso di carica puntiforme: l'estensione della sorgente a un oggetto 1-dim ha aumentato l'esponente di $1$. Si osserverà per distribuzione piana infinita, dunque con l'estensione a 2-dim, che l'esponente diviene $0$: il campo è uniforme in tutto lo spazio - intuitivamente perché la densità di linee di campo non può variare in quanto queste sono rettilinee e ortogonali al piano.

\section{Conservatività della forza di Coulomb e del campo elettrostatico}
Un campo di forze $\vec{F}$ è conservativo se può essere espresso come opposto del gradiente di un campo scalare detto \textbf{potenziale} o equivalentemente (su dominio semplicemente connesso o più generalmente connesso) se il lavoro compiuto su una traiettoria chiusa è nullo.
\\Ciò può nuovamente essere espresso come il fatto che il lavoro su una qualsiasi curva dipenda solo dagli estremi, e corrisponda in particolare all'opposto della differenza dei valori assunti in essi dalla funzione potenziale.
\[\int_\gamma \vec{F} \cdot \dd[•]{\vec{r}} = U(\vec{x}_i) - U(\vec{x}_f) = - \Delta U\]
Applicando il teorema delle forze vive:
\[\int_{\hspace{-0.45cm} \gamma \hspace{0.22cm} i}^f \vec{F} \cdot \dd[•]{\vec{r}} = T_f - T_i = \Delta T\]
osservando che l'energia cinetica \textbf{non è} invece una funzione univoca dei punti, in quanto dipende dal percorso di integrazione. Il segno della variazione di energia potenziale è scelto in conseguenza della definizione di energia:
\begin{description}
\item[Energia] capacità di compiere lavoro \textbf{positivo}
\end{description}
Il lavoro compiuto dalla forza di Coulomb su un corpo che si muove in direzione opposta è negativo, mentre viene aumentata la capacità del sistema di cariche di compiere lavoro positivo. Da qui il segno e quindi la convenzione $\vec{F} = - \vec{\nabla} U$, in quanto per il teorema del gradiente
\[U(A) - U(B) = - \int_A^B \vec{\nabla} U \cdot \dd[•]{\vec{r}}\]
Se il dominio di definizione è semplicemente connesso, $\vec{F}$ è conservativo anche se irrotazionale, ovvero se il suo rotore è nullo: $\displaystyle \vec{\nabla} \wedge \vec{F} = \vec{0}$. Applicando il teorema di Stokes per una curva chiusa si ha infatti:
\[\oint_\Gamma \vec{F} \cdot	\dd[•]{\vec{r}} = \iint\limits_{\Sigma(\Gamma)} (\vec{\nabla} \wedge \vec{F}) \cdot \dd[•]{\vec{\sigma}}\]
\\~\\
Si verifica quindi la conservatività della forza di Coulomb applicando quest'ultimo criterio. Si nota innanzitutto:
\[\hat{r} = \big(\frac{x}{r},\frac{y}{r},\frac{z}{r} \big) \quad r = \sqrt{x^2 + y^2 + z^2}\]
Ora
\[\vec{\nabla} \wedge \vec{F_C} = \begin{vmatrix}
\hat{i} & \hat{j} & \hat{k} \\ \pdv[•]{•}{x} & \pdv[•]{•}{y} & \pdv[•]{•}{z} \\ F_x & F_y & F_z
\end{vmatrix} = \hat{i} \big(\pdv[•]{F_z}{y} - \pdv[•]{F_y}{z}\big) + ...\]
Considerando ad esempio la terza componente del rotore:
\[ \pdv[•]{F_x}{y} = \frac{Qq}{4 \pi \varepsilon_0} \pdv[•]{•}{y} \big(\frac{x}{(x^2 + y^2 + z^2)^{3/2}}\big) = - 3 \frac{Qq}{4 \pi \varepsilon_0} \frac{xy}{(x^2 + y^2 + z^2)^{5/2}}\] 

\[ \pdv[•]{F_y}{x} = \frac{Qq}{4 \pi \varepsilon_0} \pdv[•]{•}{x} \big(\frac{y}{(x^2 + y^2 + z^2)^{3/2}}\big) = - 3 \frac{Qq}{4 \pi \varepsilon_0} \frac{xy}{(x^2 + y^2 + z^2)^{5/2}}\]

da cui
\[(\vec{\nabla} \wedge \vec{F_C}) \cdot \hat{k} = 0\]
ripetendo analogamente per le altre componenti ottiene lo stesso risultato, dunque
\[\vec{\nabla} \wedge \vec{F} = \vec{0}\]
\\~\\
Si può anche verificare la conservatività rifacendosi ad una delle definizioni date e considerando la natura centrale del campo di forze generato da una carica puntiforme.
\\Infatti ogni traiettoria può essere scomposta in spostamenti radiali ed archi di circonferenze (centrate nella carica sorgente) infinitesimi. Lungo i primi la forza è parallela allo spostamento e dunque $\vec{F_C} \cdot \dd[•]{\vec{s}} = (F_C \hat{r}) \cdot (\dd[•]{r} \hat{r}) = F_C \dd[•]{r}$, mentre lungo i secondi è ortogonale e dunque non compie lavoro (in coordinate polari sferiche, $\hat{r} \cdot \hat{\phi} = \hat{r} \cdot \hat{\theta} = 0$). Poiché il modulo della forza dipende solamente dalla distanza dalla sorgente, segue che il lavoro complessivo dipende solamente dalla differenza di distanza da questa degli estremi della traiettoria.
\[L_{AB} = \int_A^B \vec{F_C} \cdot \dd[•]{\vec{s}} = \int_{r_A}^{r_B} F(r) \dd[•]{r}\]
Da cui si può ricavare l'espressione per l'\textbf{energia potenziale di Coulomb} (si indica con $\vartheta$ l'angolo tra l'elemento di traiettoria e la direzione radiale nel punto di applicazione):
\[L = \frac{Qq}{4 \pi \varepsilon_0} \int_A^B \frac{\hat{r_P}}{r_P^2} \cdot \dd[•]{\vec{s}} = \frac{Qq}{4 \pi \varepsilon_0} \int_A^B \frac{1}{r_P^2} \cos \vartheta \dd[•]{s} =  \frac{Qq}{4 \pi \varepsilon_0} \int_{r_A}^{r_B} \frac{\dd[•]{r_P}}{r_P^2} = \frac{Qq}{4 \pi \varepsilon_0} \big(\frac{1}{r_A} - \frac{1}{r_B}\big) = U(A) - U(B)\]
da cui
\[U(\vec{r}) = \frac{1}{4 \pi \varepsilon_0} \frac{Qq}{r} + cost\]
Si osserva che per il principio di sovrapposizione la conservatività del campo di forze generato da una carica puntiforme implica quella del campo generato da qualsiasi altra configurazione stazionaria.

\section{Potenziale elettrico}
Poiché la carica $q$ è uno scalare, la conservatività di $\vec{F}$ implica quella di $\vec{E}$. Dunque la \textbf{forza elettromotrice} $\mathcal{E}$ lungo un circuito chiuso è sempre nulla nel caso elettrostatico:
\[\oint_\Gamma \vec{E} \cdot \dd[•]{\vec{s}} = \mathcal{E} = 0\]
Si può definire quindi un \textbf{potenziale elettrico (o elettrostatico)} secondo:
\[\int_A^B \vec{F} \cdot \dd[•]{\vec{s}} = U(A) - U(B) \qquad \int_A^B q \vec{E} \cdot \dd[•]{\vec{s}} = q \int_A^B \vec{E} \cdot \dd[•]{\vec{S}} \implies \int_A^B \vec{E} \cdot \dd[•]{\vec{S}} = q \big[\frac{U(A)}{q} - \frac{U(B)}{q}\big] \equiv q \big[V(A) - V(B)\big]\]
Da cui
\[V(\vec{r}) \equiv \frac{U(\vec{r})}{q} = \frac{1}{4 \pi \varepsilon_0} \frac{Q}{r} + cost\]
con $\ds \vec{E} = - \vec{\nabla} V$. L'unità di misura del potenziale è il Volt (V): 1 V = 1 J/C = 1 Nm/C.
\\~\\
Per il potenziale vale analogamente il principio di sovrapposizione. Dunque per una distribuzione discreta di cariche
\[V(P) = \frac{1}{4 \pi \varepsilon_0} \sum\limits_{i=1}^{N} \frac{Q_i}{r_i}\]
e al continuo
\[V(\vec{r}_P) = \frac{1}{4 \pi \varepsilon_0} \int \frac{\rho \dd[•]{\tau}}{\Delta r}\]
\\~\\
\'E possibile individuare nello spazio superfici a potenziale costante o equipotenziali. Per proprietà del gradiente esse risultano essere ortogonali al campo elettrico.
\\Il valore del potenziale in un punto è univoco a meno di una costante (per il teorema sulle funzioni a gradiente nullo). Essa può essere determinata ponendo arbitrariamente a $0$ il valore in un punto (eventualmente anche all'infinito) e calcolando il potenziale in un altro qualsiasi dalla differenza.

\section{Legge di Gauss per il campo elettrico (I equazione di Maxwell)}
Un angolo piano individua una regione di piano delimitata da due semirette con origine in comune nel vertice. Dato un arco di circonferenza a raggio costante $R$ sotteso dall'angolo di lunghezza $S$ vale $\ds \theta = \frac{S}{R}$ (in radianti), per ampiezza infinitesima $\ds \dd[•]{\theta} = \frac{\dd[•]{S}}{R}$.
\\Dato un altro arco sempre sotteso dall'angolo infinitesimo ma con direzione normale orientata con un angolo $\alpha$ rispetto alla bisettrice dell'angolo si ha
\[\dd[•]{S'} \cos \alpha = \dd[•]{S} \implies \dd[•]{\theta} = \frac{\dd[•]{S'} \cos \alpha }{R}\]
Un angolo solido individua invece una porzione di spazio delimitata da un fascio di semirette con origine nel vertice. Per un angolo infinitesimo e la porzione di calotta sferica di raggio $R$ da essa sottesa si ha $\ds \dd[•]{\Omega} = \frac{\dd[]{\Sigma_0}}{R}$. Per una superficie generica con normale orientata con angolo $\alpha$ rispetto a quella di $\dd[•]{\Sigma_0}$ vale
\[\dd[•]{\Sigma} \cos \alpha = \dd[•]{\Sigma_0} \implies \dd[•]{\Omega} = \frac{\dd[•]{\Sigma} \cos \alpha}{R}\]
Esprimendo l'elemento di calotta in coordinate sferiche è possibile determinare l'angolo solido complessivo:
\[\Omega_{tot} = \int_0^{2\pi} \big(\int_0^\pi \sin \theta \dd[•]{\theta}\big) \dd[•]{\varphi} = 4 \pi\]
\\~\\
Si consideri ora una superficie chiusa $\Sigma$ collocata nel campo generato da una carica puntiforme, con la cui posizione si fa coincidere l'origine del SdR. Un fascio di semirette con origine nella carica determina sulla superficie due superfici infinitesime $\dd[•]{S_1}$ e $\dd[•]{S_2}$ oppure nessuna; chiaramente quella più lontana dalla carica avrà area maggiore. L'intera superficie può ora essere scomposta in tali coppie di elementi infinitesimi, ovvero suddividendo il volume da essa racchiuso in tronchi di cono. Orientando ora ogni elemento con la normale esterna e considerando il flusso complessivo del campo elettrico attraverso $\Sigma$ questo corrisponde alla somma dei contributi di ogni coppia.
\\Ora si ha per la superficie più lontana
\[\dd[•]{\Phi_2} = \vec{E_2} \cdot \dd[•]{S_2} = E_2 \dd[•]{S_2} \cos \theta_2 = E_2 \frac{\dd[•]{S_2} \cos \theta_2}{r_2^2} r_2^2 = E_2 \dd[•]{\Omega} r_2^2\]
ma per quella più vicina 
\[\dd[•]{\Phi_1} = \vec{E_1} \cdot \dd[•]{S_1} = E_1 \dd[•]{S_1} \cos \theta_1 = - E_1 \frac{\dd[•]{S_1} \cos (\pi - \theta_1)}{r_1^2} r_1^2 = - E_1 \dd[•]{\Omega} r_1^2\]
in quanto l'angolo solido sotteso è il medesimo considerando l'angolo da utilizzarsi correttamente per proiettare la superficie $\dd[•]{S_1}$ sulla porzione di calotta corrispondente. Si ha quindi
\[\dd[•]{\Phi_1} + \dd[•]{\Phi_2} = \frac{1}{4 \pi \varepsilon_0} Q \dd[•]{\Omega} \big(\frac{r_2^2}{r_2^2} - \frac{r_1^2}{r_1^2}\big) = 0\]
da cui complessivamente
\[\Phi_\Sigma(\vec{E}) = \oint_\Sigma \vec{E} \cdot \dd[•]{\vec{S}} = 0\]
Dunque il flusso del campo elettrico attraverso una superficie che non racchiude carica è nullo. Si osservi che il risultato è conseguenza della dipendenza dall'inverso del quadrato della distanza, reciproca di quella dell'area sottesa dall'angolo solido. In realtà assumendo la legge di Gauss è possibile a ritroso ottenere Coulomb, dimostrando di fatto l'equivalenza delle due.
\\~\\
Se la carica è invece interna alla superficie, i contributi non si semplificano: ogni fascio di semirette con origine nella sorgente intercetta una sola superficie infinitesima e $\ds \dd[•]{\Phi} = E \dd[•]{S} \cos \theta \geq 0$ (assunto campo uscente) in quanto $\theta < 90°$. In realtà ciò vale per superfici convesse, ma ss. concave possono essere chiaramente suddivise in superfici convesse con facce in comune, applicando per quelle prive di carica all'interno il risultato precedentemente ottenuto.
\[\dd[•]{Phi} = E \frac{\dd[•]{S} \cos \theta}{r^2} r^2 = E \dd[•]{\Omega} r^2 = \frac{Q}{4 \pi \varepsilon_0} \dd[•]{\Omega}\]
Integrando sull'angolo solido
\[\Phi_\Sigma(\vec{E}) = \oiint\limits_\Sigma \dd[•]{\Phi} = \frac{Q}{4 \pi \varepsilon_0} \oiint\limits_\Sigma \dd[•]{\Omega} = \frac{Q}{4 \pi \varepsilon_0} \cdot 4 \pi = \frac{Q}{\varepsilon_0}\]
indipendentemente dalla posizione della carica $Q$.
\\La legge di Gauss è chiaramente generalizzabile ad un numero arbitrario di cariche puntiformi applicando la sovrapposizione e la linearità del prodotto scalare e dell'integrale:
\[\oiint\limits_\Sigma \vec{E} \cdot \vec{\dd[•]{S}} = \frac{1}{\varepsilon_0} \sum\limits_{i=1}^{N} Q_i = \frac{Q_{tot}}{\varepsilon_0}\]
e analogamente per le distribuzioni continue vale
\[\oiint\limits_\Sigma \vec{E} \cdot \vec{\dd[•]{S}} = \frac{1}{\varepsilon_0} \iiint\limits_{V(\Sigma)} \rho \dd[•]{\tau}\]
ove $V$ è il volume racchiuso da $\Sigma$.

\lawboxtext{Legge di Gauss}{
Il flusso del campo elettrico attraverso una qualsiasi superficie chiusa (\textit{orientata con normale esterna}) eguaglia la somma algebrica delle cariche contenute all'interno della superficie, comunque esse siano distribuite, divisa per la costante dielettrica del vuoto.
}

Ad esempio si può notare come il flusso del campo di un dipolo attraverso una superficie che racchiuda ambo le cariche sia nullo; in altri termini, il numero di linee di campo che entrano è uguale a quello di quante escono.
\\\'E possibile utilizzare la legge di Gauss per determinare il valore del campo elettrico su superfici matematiche in problemi che mostrino particolari simmetrie. Se esiste infatti una $\Sigma$ tale che il campo vi sia costante in modulo e orientazione relativa, calcolando la carica totale al suo interno si può calcolare il campo secondo
\[|\vec{E}| = \frac{Q_T}{\cos \theta \varepsilon_0 \oiint\limits_\Sigma \vec{E} \cdot \vec{\dd[•]{S}}}\]
\\~\\
Applicando il teorema della divergenza (se il volume e la superficie ne soddisfano le ipotesi, ma queste sono di rado vincolanti nello studio di problemi fisici) si ricava la forma puntuale della Legge:
\[\iint\limits_\Sigma \vec{E} \cdot \dd[•]{\vec{S}} = \iiint\limits_{V(\Sigma)} (\vec{\nabla} \cdot \vec{E}) \dd[]{\tau} = \frac{1}{\varepsilon_0} \iiint\limits_{V(\Sigma)} \rho \dd[•]{\tau}\]
valendo la relazione indipendentemente dal volume e dalla superficie considerati, si impone l'uguaglianza degli integrandi:
\[\vec{\nabla} \cdot \vec{E} = \frac{\rho}{\varepsilon_0}\]
Considerando la divergenza in coordinate cartesiane, si verifica facilmente nel caso di carica puntiforme 

\[\vec{\nabla} \cdot \vec{E} = \pdv[•]{E_x}{x} + \pdv[•]{E_y}{y} + \pdv[•]{E_z}{z} = 0\]
per $r \neq 0$ (la densità corrispondente è infatti una delta).
\\Infatti

\[\pdv[•]{E_x}{x} = \frac{Q}{4 \pi \varepsilon_0} \pdv[•]{•}{x} \big(\frac{x}{(x^2 + y^2 + z^2)^{3/2}}\big) = \frac{Q}{4 \pi \varepsilon_0} \bigg(\frac{1}{(x^2 + y^2 + z^2)^{3/2}} - \frac{x (3 x)}{(x^2 + y^2 + z^2)^{5/2}}\bigg) = \]
\[= \frac{Q}{4 \pi \varepsilon_0 r^6} \big[r^3 - 3 x^2 r\big]\]

e analogamente

\[\pdv[•]{E_y}{y} = \frac{Q}{4 \pi \varepsilon_0 r^6} \big[r^3 - 3 y^2 r\big] \qquad \pdv[•]{E_z}{z} = \frac{Q}{4 \pi \varepsilon_0 r^6} \big[r^3 - 3 z^2 r\big]\]

da cui

\[\vec{\nabla} \cdot \vec{E} = \pdv[•]{E_x}{x} + \pdv[•]{E_y}{y} + \pdv[•]{E_z}{z} = \frac{Q}{4 \pi \varepsilon_0 r^6} \big[r^3 - 3x^2 r + r^3 - 3y^2 r + r^3 - 3z^2 r\big] = \frac{3 Q}{4 \pi \varepsilon_0 r^3} (1-1) = 0\]


\subsection{Applicazione: distribuzione lineare indefinita}

Si considera una superficie cilindrica coassiale alla distribuzione di altezza $H$ e raggio di base $r$. Per le considerazioni di simmetria già effettuate nella trattazione precedente, il campo è costante in modulo sulla superficie laterale e in orientamento relativo in quanto radiale. Sulle superfici circolari è invece parallelo e dunque non contribuisce al flusso.

\[\oiint\limits_{\Sigma} \vec{E} \cdot \dd[•]{\vec{S}} = \iint\limits_{\Sigma_L} \vec{E} \cdot \dd[•]{\vec{S_L}} = \iint\limits_{\Sigma_L} E \dd[•]{S_L} = E  \iint\limits_{\Sigma_L}\dd[•]{S_L} = E (2 \pi H r)\]

Applicando Gauss

\[Q_T = \lambda \cdot H \implies (2 \pi H r) E = \frac{\lambda H}{\varepsilon_0} \implies E = \frac{\lambda}{2 \pi \varepsilon_0 r}\]


\subsection{Applicazione: distribuzione piana indefinita}
Sia $\sigma > 0$. Considerando un punto sulla superficie ed una qualsiasi retta giacentevi cui il punto appartiene, questa costituisce una distribuzione lineare uniforme; dunque per quanto visto la linea di campo uscente dal punto è ortogonale alla retta. Poiché ciò vale per ogni retta del fascio proprio, si ha che le linee di campo sono \textbf{ortogonali al piano}. Considerando una superficie cilindrica a cavallo del piano, con facce circolari ad esso parallele, il contributo al flusso del campo dalla superficie laterale risulta chiaramente nullo. Sia $A$ l'area di base e $r$ la distanza delle facce dal piano. Allora 

\[\oiint\limits_{S} \vec{E} \cdot \dd[•]{\vec{S}} = \iint\limits_{S_{sx}} \vec{E} \cdot \dd[•]{\vec{S_{sx}}} + \iint\limits_{S_{dx}} \vec{E} \cdot \dd[•]{\vec{S_{dx}}} = E_{sx} \iint\limits_{S_{sx}}\dd[•]{S_{sx}} + E_{dx} \iint\limits_{S_{dx}}\dd[•]{S_{dx}}\]

essendo le due facce equidistanti dal piano per costruzione, per simmetria il campo deve avere lo stesso valore su di esse: $E_{dx} = E_{sx} = E(r)$. Dunque

\[E_{sx} \iint\limits_{S_{sx}}\dd[•]{S_{sx}} + E_{dx} \iint\limits_{S_{dx}}\dd[•]{S_{dx}} = 2 E A\]

Applicando Gauss

\[Q_T = \sigma \cdot A \implies 2 E A = \frac{\sigma A}{\varepsilon_0} \implies E = \frac{\sigma}{2 \varepsilon_0}\]

Si osserva dunque che il campo è uniforme nello spazio, \textit{indipendentemente dalla distanza dalla superficie}.

\section{Discontinuità del campo, condizioni al contorno}
Scegliendo un versore normale arbitrario, si osserva che attraversando una distribuzione piana indefinita il campo presenta una discontinuità nella componente normale:
\[\Delta \vec{E} = \Delta E_n \hat{n} = 2 E \hat{n} = \frac{\sigma}{\varepsilon_0} \hat{n}\]
Per generalizzare ad una qualsiasi superficie carica si considera un cilindro gaussiano di altezza $\dd[•]{h}$ posto a cavallo di questa, assumendo che la densità di carica superficiale sia costante nell'intorno infinitesimo intercettato. 
\\Applicando il teorema di Gauss:
\[\dd[•]{\Phi} = \vec{E_u} \cdot \dd[•]{\vec{S_u}} + \vec{E_d} \cdot \dd[•]{\vec{S_d}} + \vec{E_l} \cdot \dd[•]{\vec{S_l}} = \frac{\dd[•]{q}}{\varepsilon_0}\]
Compattando la superficie, ovvero passando al limite $\dd[•]{h} \rightarrow 0$, il contributo della superficie laterale tende ad annullarsi. Inoltre in modulo i valori del campo sopra e sotto tendono al medesimo, dunque
\[\dd[•]{Phi} \rightarrow E (\dd[•]{S_u}) \cos \theta_u + E (\dd[•]{S_d}) \cos \theta_d\]
Scelto ora un versore normale arbitrario, sia quello di $\dd[•]{\vec{S_u}}$, si ha dunque
\[\dd[•]{A} ( E_u^\perp - E_d^\perp) = \dd[•]{A} \frac{\sigma}{\varepsilon_0} \implies \Delta E^\perp = \frac{\sigma}{\varepsilon_0}\]
Si dimostra ora che non si ha discontinuità della componente tangente come conseguenza della conservatività del campo. Considerando un circuito rettangolare sempre a cavallo della superficie con lati $\dd[•]{l}$ e $\dd[•]{h}$:
\[\oint_\gamma \vec{E} \cdot \dd[•]{\vec{\gamma}} = 0 = \sum\limits_{i=1}^{4} \vec{E_i} \cdot \dd[•]{\vec{r_i}}\]
Passando al limite $\dd[•]{h} \rightarrow 0$ i contributi dei lati trasversi tendono ad annullarsi e si ha
\[\oint_\gamma \vec{E} \cdot \dd[•]{\vec{\gamma}} = \vec{E_1} \cdot \dd[•]{\vec{r_1}} + \vec{E_3} \cdot \dd[•]{\vec{r_3}} = E_1 \dd[•]{l} \cos \theta_1 + E_3 \dd[•]{l} \cos \theta_3 = \dd[•]{l} (E_1^{\parallel} + E_3^{\parallel}) = 0\]
Da cui $E_1^{\parallel} + E_3^{\parallel} = 0$. Fissando ora convenzionalmente un versore tangente si ha
\[E_1^{\parallel} \hat{n} - E_3^{\parallel} \hat{n} = \vec{0} \Longleftrightarrow \Delta E^{\parallel} = 0\]
dunque attraversando una superficie carica non si ha discontinuità nella componente tangente del campo elettrico.

\chapter{Elettrostatica nei materiali}

\section{Conduttori}
\begin{description}
\item[Conduttore] solido (materiale) che contiene elettroni (o più generalmente cariche) \textbf{liberi}, ovvero in grado di muoversi liberamente nel reticolo ma non di uscirne
\end{description}
In realtà si possono avere anche conduttori liquidi, in cui le cariche libere sono ioni. Gli elettroni liberi nei conduttori metallici sono quelli di valenza, che hanno sufficiente energia per accedere alla \textbf{banda di conduzione}. 
\infobox{Esempio: il rame}{
Nel Cu si ha Z = 29, A = 64, $\rho$ = 8,96 $\mathrm{g/cm^3}$. Assumendo un elettrone libero per atomo si ha
\[n_{e}^{cond} = N_A \frac{\rho}{\mu} \approx 8 \times 10^{22} \meters{c}{-3}\]
}

I metalli approssimano bene la condizione di \textbf{conduttore ideale}, in cui la scorta di cariche libere è infinita.
\\~\\
Applicando un campo elettrostatico esterno ad un conduttore, non si osservano correnti che perdurano nel tempo. Gli atomi restano fissi nel reticolo, mentre gli elettroni sono accelerati da una forza $\vec{F} = e \vec{E}$. 
\\Si assiste al fenomeno dell'\textbf{induzione elettrica}: in un tempo caratteristico dell'ordine dei $\mathrm{\mu s}$, le cariche si redistribuiscono di modo da rendere nullo il campo all'interno: il controcampo scherma completamente il campo esterno e si raggiunge così l'equilibrio, ovvero la condizione di assenza di ulteriori movimenti di cariche.

\lawboxtext{I proprietà dei conduttori immersi in campo elettrico esterno}{
All'equilibrio, il campo totale all'interno del conduttore è nullo e le cariche si dispongono in modo tale da schermare (annullare) il campo esterno.
}

Applicando Gauss in forma differenziale per un punto interno
\[\div \vec{E} = \div \vec{0} = 0 = \rho/\varepsilon_0 \implies \rho = 0\]
o integrale per una superficie gaussiana tutta contenuta all'interno
\[\oiint\limits_\Sigma \vec{E_T} \cdot \dd[•]{\vec{S}} = \oiint\limits_\Sigma \vec{0} \cdot \dd[•]{\vec{S}} = 0 = Q_T / \varepsilon_0 \implies Q_T = 0\]
si ottiene

\lawboxtext{II proprietà}{
All'equilibrio, in ogni punto interno del conduttore la carica è nulla.
}

Da cui segue

\lawboxtext{III proprietà}{
Lo spostamento di cariche elettriche determinato dal campo elettrico esterno all'equilibrio si risolve in un riarrangiamento di carica che interessa solo la superficie del conduttore.
}

Considerando la costituzione microscopica dei materiali, si osserva più precisamente che le cariche si dispongono entro due layer atomici dalla superficie del corpo.
\\La configurazione assunta permette di minimizzare l'energia potenziale del conduttore.

\paragraph{Campo in superficie}
Per quanto visto, essendo nullo il campo e dunque la sua componente tangente all'intero, questa lo sarà per continuità anche immediatamente all'esterno. Per la componente normale:
\begin{itemize}
\item In zone a carica superficiale positiva il campo immediatamente fuori è uscente in quanto altrimenti si avrebbe una forza verso l'interno che contraddirebbe la condizione di equilibrio, provocando un movimento di cariche (le cariche non possono infatti lasciare il conduttore).
\item Per motivazione analoga in zone a carica superficiale negativa il campo immediatamente fuori è diretto verso l'interno del conduttore
\end{itemize}
Si possono altresì verificare le condizioni applicando Gauss ad un cilindro a cavallo della superficie, la cui faccia interiore non contribuisce a causa del campo nullo.
\\Dunque l'introduzione del conduttore modifica la configurazione del campo esterno, che si modifica di modo che le linee di forza entrino ed escano perpendicolarmente alla superficie del conduttore. \textit{Se si osservano angoli diversi da 90° si tratta dunque di un dielettrico}

\lawboxtext{IV proprietà}{
L'immersione di un conduttore in un campo elettrico esterno all'equilibrio altera le linee dei forza del campo anche all'esterno del conduttore. Il cambiamento delle linee di campo dipende dalla geometria del conduttore e le linee sulla superficie sono normali ad essa e hanno modulo $\frac{\sigma}{\varepsilon_0}$
}

Per il potenziale chiaramente l'integrale lungo una qualsiasi curva tra due punti interni è nullo:
\[\Delta V_{AB} = V(A) - V(B) = \int_A^B \vec{E} \cdot \dd[•]{\vec{\gamma}} = \int_A^B \vec{0} \cdot \dd[•]{\vec{\gamma}} = 0 \implies V(A) = V(B)\]
Analogamente integrando sulla superficie essendo il campo normale:
\[\Delta V_{CD} = V(C) - V(D) = \int_C^D \vec{E} \cdot \dd[•]{\vec{\gamma}} = \int_C^D 0 = 0 \implies V(C) = V(D)\]

\lawboxtext{V proprietà}{
La superficie di un conduttore in un c.e. esterno è all'equilibrio equipotenziale, così come il suo interno.
}

\paragraph{Cavità} in presenza di una cavità all'interno, si dimostra che non può essere all'equilibrio presente alcuna carica sulla sua superficie ed il campo al suo interno resta nullo. 
\\Applicando Gauss ad una superficie che racchiude la cavità si ottiene che la carica totale interna è nulla, ma si tratta di un'informazione inutile in quanto non esclude possano esserci cariche di diverso segno disposte intorno alla cavità e quindi un campo al suo interno.
\\Applicando invece la conservatività del campo ad una linea di circuitazione chiusa la cui porzione interna alla cavità (di estremi $A$ e $B$) segue quella di un'ipotetica linea di campo:
\[\Gamma = \Gamma_{AB} \cup \Gamma_{BA} \implies \oint_\Gamma \vec{E} \cdot \dd[•]{\vec{\gamma}} = 0 = \int_{\hspace{-0.45cm} \Gamma_{AB} \hspace{0.22cm} A}^B \vec{E} \cdot \dd[•]{\vec{\gamma}} + 0 \]
in quanto il campo all'interno è nullo. Ma per costruzione il campo è tangente a $\Gamma_{AB}$ e dunque si ha
\[\int_{\hspace{-0.45cm} \Gamma_{AB} \hspace{0.22cm} A}^B \vec{E} \cdot \dd[•]{\vec{\gamma}} = \int_{\hspace{-0.45cm} \Gamma_{AB} \hspace{0.22cm} A}^B E \dd[•]{\gamma}\]
Perché si annulli l'integrale si deve chiaramente annullare ogni contributo infinitesimo (in quanto sono tutti $\geq 0$, da cui segue che il campo debba essere nullo anche all'interno della cavità.

\lawboxtext{VI proprietà}{
All'equilibrio il campo elettrico all'interno di una cavità del conduttore è nullo e non vi sono cariche indotte sulla superficie della cavità.
}

Si tratta del principio applicativo della \textbf{Gabbia di Faraday (o schermo elettrostatico)}. \'E in realtà possibile aggirare lo schermo se il campo elettrico varia significativamente in un tempo inferiore al tempo caratteristico di redistribuzione delle cariche nel conduttore.
\\La presente trattazione è valida in generale nel caso elettrostatico (dunque con campo non variabile nel tempo), anche in presenza di campo non omogeneo.

\paragraph{Carica nella cavità} se la sorgente del campo esterno è invece collocata all'interno di una cavità del conduttore, la distribuzione delle cariche sul conduttore è differente. Per il principio dell'induzione le cariche si dispongono di modo da schermare all'interno il campo esterno, dunque \textbf{la superficie interna si carica in segno opposto alla carica esterna}.
\\Applicando Gauss ad una superficie all'interno del conduttore racchiudente la cavità si ha
\[\oiint\limits_\Sigma \vec{E_T} \cdot \dd[•]{\vec{S}} = \oiint\limits_\Sigma \vec{0} \cdot \dd[•]{\vec{S}} = 0 = \frac{1}{\varepsilon_0} (Q + Q_I) \implies Q_I = - Q\]
Poiché il conduttore è complessivamente neutro, valendo la conservazione della carica si ha $Q_E = - Q_I = Q$. Applicando quindi Gauss ad una superficie esterna al conduttore si osserva che la carica racchiusa all'interno è pari a $Q + Q_I + Q_E = Q$, come se il conduttore fosse assente. In realtà valgono le medesime considerazioni sulla perturbazione della configurazione del campo, ma a grandi distanze (molto maggiori del diametro della superficie esterna) la perturbazione locale si attenua e le linee ritornano radiali: il campo diviene indistinguibile da quello in assenza di conduttore.
\\Si può inoltre osservare che
\begin{itemize}
\item Lo schermo elettrostatico non scherma il campo generato all'interno
\item L'approssimazione a grandi distanze è in realtà valida per qualsiasi distribuzione di cariche indipendentemente dalla forma: il campo è ben approssimato da quello di una carica puntiforme posta nel centro di carica della distribuzione (se questa non è complessivamente neutra!)
\item Poiché in questo caso entrambe le superfici del conduttore sono cariche si parla di \textbf{induzione completa}, che si verifica quando tutte le linee di campo toccano la superficie del conduttore
\end{itemize} 

\paragraph{Conduttore carico} nel caso di un conduttore carico le cariche depositate si respingono fino a raggiungere la configurazione di minima energia. \textbf{All'equilibrio valgono esattamente le medesime proprietà viste per conduttori neutri.}

\subsection{Potere di dispersione delle punte}
Si considera un sistema di due sfere conduttrici collegate da un filo di spessore infinitesimo. Si fornisca una carica $+Q$ dall'esterno: questa si distribuisce fino al raggiungimento dell'equilibrio, in cui non si ha più movimento di cariche. 
\\Essendo collegate dal filo, le due sfere compongono un unico conduttore, la cui superficie deve essere equipotenziale per quanto visto. Ponendo il potenziale a distanza $\infty$ dal centro delle sfere a $0$ e considerando che il campo all'interno è nullo all'equilibrio, si calcola il potenziale sulla superficie delle due sfere:
\[V(R_1) = V(\infty) + \int_{R_1}^\infty \vec{E} \cdot \dd[•]{\vec{r}} = \frac{1}{4 \pi \varepsilon_0} \frac{Q_1}{R_1} \qquad V(R_2) = \frac{1}{4 \pi \varepsilon_0} \frac{Q_2}{R_2}\]
imponendone l'uguaglianza si ottiene
\[\frac{Q_1}{R_1} = \frac{Q_2}{R_2} \implies \sigma_1 R_1 = \sigma_2 R_2\]
ovvero
\[Q_i \propto R_i \qquad \sigma_i \propto \frac{1}{R_i}\]
dunque le porzioni di superficie di materiali conduttori all'equilibrio con raggio di curvatura minore (ovvero curvatura maggiore) presentano \textbf{maggiori addensamenti di carica}. In particolare ciò si verifica nelle punte, in cui la forza di repulsione elettrostatica tra le cariche addensatevi può addirittura essere sufficientemente elevata da superare la barriera di potenziale della superficie e portare alla fuoriuscita di cariche libere: si parla allora di \textbf{potere di dispersione}.
\\Il principio esposto è anche alla base della \textbf{messa a terra}: il pianeta può essere infatti considerato un enorme conduttore con un raggio estremamente maggiore di quello equivalente di qualsiasi sistema elettrico convenzionale, a cui possono essere quindi facilmente trasferite cariche in eccesso. Si utilizza in realtà talvolta del materiale conduttore sbriciolato nel terreno per rendere ottimale la conduzione.

\section{I tubi di flusso}
Data una regione di spazio attraversata da linee di campo elettrico, è possibile costruire un oggetto geometrico di particolare utilità. Il \textbf{tubo di flusso} è una superficie chiusa le cui pareti laterali sono definite dalle linee di campo che poggiano su di una stessa linea chiusa; i due 'tappi' sono invece superfici ortogonali alle linee di campo determinati dalla linea stessa e da quella corrispondente all'altra estremità.
\\Per costruzione, applicando Gauss al tubo il contributo della superficie laterale è nulla. Per superfici estreme infinitesime vi si può assumere il campo uniforme, da cui
\[\vec{E_1} \cdot \dd[•]{\vec{S_1}} + \vec{E_2} \cdot \dd[•]{\vec{S_2}} = \pm E_1 \dd[•]{S_1} \pm E_2 \dd[•]{S_2} = \frac{\dd[]{Q_T}}{\varepsilon_0}\]
Considerando una superficie conduttrice carica con $\sigma > 0$ e costruendo un tubo di flusso infinitesimo a cavallo di essa prolungando analiticamente le linee di campo all'interno si ha
\[\Phi = E_1 \dd[•]{S_1} + E_2 \dd[•]{S_2} = E_2 \dd[•]{S_2} = \frac{\dd[•]{Q}}{\varepsilon_0} \implies E_2 = \frac{\dd[•]{Q}}{\dd[•]{S_2} \varepsilon_0} = \frac{\dd[•]{S}}{\dd[•]{S_2}} \frac{\sigma}{\varepsilon_0}\]
con $\dd[•]{S}$ intercettata dal tubo sulla superficie. Si osserva chiaramente la proporzionalità all'inverso del quadrato della distanza, espressa dall'area del 'tappo' esterno del tubo di flusso. Una diminuzione della densità di linee di campo corrisponde infatti ad un ampliamento di tale superficie.
\\Si osservi che per la superficie piana indefinita l'area resta invece costante in quanto le linee di campo sono sempre parallele fra loro: il campo è infatti costante in tutto lo spazio.

\section{Condensatori}
\begin{description}
\item[Condensatore] sistema composto da due armature metalliche qualsiasi (o più generalmente due conduttori) con disposizione qualsiasi aventi cariche opposte in segno e uguali in modulo
\end{description}
La condizione sulla carica è necessaria perché si abbia induzione completa, ovvero tutte le linee di campo uscenti dal conduttore $A$ entrino in $B$ o viceversa.

\paragraph{Differenza di potenziale} Essendo il campo elettrico conservativo e le due superfici equipotenziali il valore ottenuto è univoco. Si considera la linea di campo $L_i$ ed un tubo di flusso che la racchiuda con superficie trasversale infinitesima. Considerando la porzione di tale tubo fino ad una data distanza $r_i$, con superficie trasversale $\dd[•]{S_i}$, e detta $\dd[•]{Q_i}$ la porzione di carica di $A$ sull'elemento di superficie intercettato si ha
\[E_i = \frac{\dd[•]{Q_i}}{\varepsilon_0 \dd[•]{S_i}}\]
e dunque integrando sulla linea di campo
\[\Delta V = V(A) - V(B) = \int_{\hspace{-0.45cm} L_i \hspace{0.22cm} A}^B \frac{\dd[•]{Q_i}}{\varepsilon_0 \dd[•]{S_i}} \dd[]{r_i} = \dd[•]{Q_i} \int_{\hspace{-0.45cm} L_i \hspace{0.22cm} A}^B \frac{ \dd[]{r_i}}{\varepsilon_0 \dd[•]{S_i}}\]
portando fuori la carica in quanto non sono presenti cariche nello spazio intermedio o circostante $A$ e $B$. \'E ora possibile isolare il termine della carica e, considerando che l'induzione completa fa sì l'espressione sia valida per ogni elemento di carica superficiale di $A$, sommare su tutte le linee di campo ottenendo la carica complessiva:
\[\dd[•]{Q_i} = \frac{\Delta V}{\ds \int_{\hspace{-0.45cm} L_i \hspace{0.22cm} A}^B \frac{ \dd[]{r_i}}{\varepsilon_0 \dd[•]{S_i}}} \implies Q_A = +Q = \sum\limits_{i=1}^{N} \dd[•]{Q_i} = \sum\limits_{i=1}^{N} \frac{\Delta V}{\ds \int_{\hspace{-0.45cm} L_i \hspace{0.22cm} A}^B \frac{ \dd[]{r_i}}{\varepsilon_0 \dd[•]{S_i}}} = \Delta V \sum\limits_{i=1}^{N} \frac{1}{\int_{\hspace{-0.45cm} L_i \hspace{0.22cm} A}^B \frac{ \dd[]{r_i}}{\varepsilon_0 \dd[•]{S_i}}}\]
il fattore di proporzionalità tra la carica accumulata sulle armature e la differenza di potenziale tra di esse dipende esclusivamente dalla geometria del sistema, ovvero dalla forma delle armature e la loro disposizione reciproca (oltre che dal materiale: si è assunto il vuoto e dunque $\varepsilon = \varepsilon_0$).
\\Si definisce tale grandezza \textbf{capacità} del condensatore
\[C \equiv \sum\limits_{i=1}^{N} \frac{1}{\int_{\hspace{-0.45cm} L_i \hspace{0.22cm} A}^B \frac{ \dd[]{r_i}}{\varepsilon_0 \dd[•]{S_i}}}\]
E si può dunque esprimere la proprietà caratteristica dei condensatori come
\[Q = C \Delta V \]
Dunque la carica depositata sulle armature di un condensatore, in assenza di variazione della sua configurazione e del materiale in cui queste sono immerse, è direttamente proporzionale alla differenza di potenziale che si instaura - e viceversa.

\section{Condensatore piano a facce parallele}
Si considerano due armature di superficie $S$ e spessore trascurabile rispetto alla distanza reciproca $d$. Assumendo che si abbia $d^2 \ll S$ (più generalmente che la distanza reciproca sia trascurabile rispetto all'ampiezza delle superfici) è possibile adottare l'approssimazione con due superfici piane cariche indefinite.
\\All'equilibrio per attrazione reciproca la configurazione delle cariche minimizzante l'energia del sistema è quella per cui la $\sigma$ è uniforme sulle facce delle armature. Il campo è dunque non nullo e uniforme nella porzione di spazio a forma di parallelepipedo delimitata dalle facce interno e nullo all'esterno, in quanto
\[\vec{E}_+ = \frac{|\sigma|}{2 \varepsilon_0} \hat{n_+} \qquad \vec{E}_- = - \frac{|\sigma|}{2 \varepsilon_0} \hat{n_-}\]
Il calcolo della ddp è triviale: applicando la sovrapposizione all'interno si ha $\vec{E}_+ \, \parallel \, \vec{E}_-$ e dunque $|\vec{E_T}| = 2 |\vec{E}_+| = \frac{\sigma}{\varepsilon_0}$
\\Integrando tra le due facce su una linea di campo rettilinea ad esse ortogonale
\[\Delta V = \int_A^B \vec{E} \cdot \dd[•]{\vec{r}} = \int_{x_A}^{x_B} \frac{\sigma}{\varepsilon_0} \dd[•]{r} = \frac{\sigma d}{\varepsilon_0}\]
da cui
\[C = \frac{Q}{\Delta V} = \frac{\sigma S \varepsilon_0}{\sigma d} = \frac{S \varepsilon_0}{d}\]
Si osserva chiaramente la validità di quanto affermato in generale sulla capacità: essa dipende da fattori geometrici (intrinseci: $S$ e della disposizione: $d$) e dal materiale interposto ($\varepsilon_0$).
\\~\\
Per determinare come siano disposte le cariche sulle facce esterne ed interne dei condensatori, si costruiscono cilindri gaussiani a cavallo di entrambe (si consideri l'armatura positiva $A$). Per il cilindro sulla faccia esterna:
\[\Phi = \oiint\limits_{\Sigma_e} \vec{E} \cdot \dd[•]{\vec{S}} = 0 = \frac{\sigma_e \dd[•]{S_e}}{\varepsilon_0} \implies \sigma_e = 0\]

\lawboxtext{Proprietà dei condensatori a facce piane parallele}{
In condizioni stazionarie, le cariche di ogni armatura sono disposte solamente sulla faccia interna della stessa.
}

Per il cilindro sulla f. interna, considerato il contributo nullo della superficie laterale in quanto il campo è ortogonale alle armature:
\[\Phi = \oiint\limits_{\Sigma_i} \vec{E} \cdot \dd[•]{\vec{S}} = \frac{\sigma}{\varepsilon_0}  \dd[•]{S_i} = \frac{\sigma_i \dd[•]{S_i}}{\varepsilon_0} \implies \sigma_i = \sigma\]

\subsection{Condensatori in parallelo}
Un sistema di due o più condensatori è detto in parallelo quando \textbf{è presente la medesima differenza di potenziale ai capi di ognuno}. Per la capacità equivalente si ha quindi:
\[\Delta V = \frac{Q_i}{C_i} = \frac{Q_{tot}}{C_{eq}} = \frac{1}{C_{eq}} \sum_i Q_i = \frac{1}{C_{eq}} \sum_i C_i \Delta V \implies C_{eq} = \sum_i C_i\]

\lawboxtext{Capacità equivalente per cc. in parallelo}{
La capacità equivalente per un sistema di condensatori in parallelo equivale alla somma delle singole capacità
}

Dunque la capacità di un sistema di condensatori in parallelo è sempre maggiore di quella dei singoli condensatori.

\subsection{Condensatori in serie}
Un sistema di due o più condensatori è detto in serie quando le facce di carica opposta sono collegate a due a due e (per induzione) la carica è la stessa su ogni condensatore. Per la capacità equivalente si ha quindi:
\[\Delta V_{tot} = \sum_i \Delta V_i = \sum_i \frac{Q_i}{C_i} = Q \sum_i \frac{1}{C_i} = \frac{Q}{C_{eq}} \implies \frac{1}{C_{eq}} = \sum_i \frac{1}{C_i}\]

\lawboxtext{Capacità equivalente per cc. in serie}{
Il reciproco della capacità equivalente per un sistema di condensatori in serie equivale alla somma dei reciproci delle singole capacità
}

Dunque la capacità di un sistema di condensatori in serie è sempre minore di quella dei singoli condensatori.

\section{Energia e lavoro elettrostatici}
\begin{description}
\item[Energia (potenziale elettrostatica) di un sistema di cariche] lavoro necessario per assemblare il sistema - ovvero portare le cariche in posizione da distanza iniziale infinita
\end{description}

Per una coppia di cariche $+Q$ e $+q$ in posizioni finali $O$ e $P$ il lavoro è compiuto opponendosi alla forza elettrica:
\[L^e = \int_\infty^P \vec{F_{ext}} \cdot \dd[•]{\vec{r}} = - \int_\infty^P q\vec{E} \cdot \dd[•]{\vec{r}} = - (U(\infty) - U(P)) = U(P) - U(\infty)\]
posto $U(\infty) = 0$, si ottiene la definizione data.
\\Più generalmente per un sistema di più cariche, e.g. 3, si considera il lavoro compiuto contro la forza tra ogni coppia di cc:
\[U = L_1 + L_2 + L_3 = 0 + L_2 + L_3 = U_{12} + (U_{13} + U_{23})\]
in quanto per fissare la prima carica in assenza delle altre non si compie lavoro. Ora
\[L_2 = U_{12} = \frac{1}{4 \pi \varepsilon_0} \frac{q_1 q_2}{r_{12}} = q_2 \bigg[\frac{1}{4 \pi \varepsilon_0} \frac{q_1}{r_{12}}\bigg] = q_2 V_{12} = q_1 V_{21} = U_{21}\]
ove il primo indice fa riferimento alla sorgente del campo, il secondo alla posizione in cui si calcola il potenziale.
\[L_3 = U_{13} + U_{23} = q_3 \big[V_{13} + V_{23}\big] = q_1 V_{31} + q_2 V_{32} = U_{31} + U_{32}\]
dunque generalizzando ed applicando il principio di sovrapposizione
\[U_T = \frac{1}{2} \sum\limits_{i,j=1}^{N} U_{ij} = \frac{1}{2} \sum\limits_{i=1}^{N} \frac{1}{2} q_i \bigg[\sum\limits_{j=1, j \neq i}^{N} \frac{q_j}{4 \pi \varepsilon_0 x_{ij}}\bigg] = \sum\limits_{i=1}^{N} q_i V_{i}^{T}\]
che nel continuo diviene (con $\tau$ volume che racchiude le cariche)
\[U_T = \frac{1}{2} \iiint_\tau \rho V_T \dd[•]{\tau}\]
\\~\\
Nel caso di un condensatore generico, lo spostamento di una prima carica positiva da $B$ ad $A$ non richiede alcun lavoro, in quanto i conduttori sono inizialmente neutri. Per la seconda carica $+$ si ha però $L_{est} \neq 0$ e in particolare $L_{est} > 0$ in quanto compiuto in opposizione alla forza di repulsione elettrostatica esercitata dalla carica già presente, e così per le successive fino al completamento del processo di \textbf{carica del condensatore}.
\[\dd[•]{L_C} = \dd[•]{q} \vec{E} \cdot \dd[•]{\vec{r}} = \dd[•]{q} (- \dd[•]{V}) = - \dd[•]{q} \dd[•]{V} = - \dd[•]{L_{est}}\]
integrando per ottenere il lavoro complessivo per spostare la carica infinitesima
\[\Delta L_{est} = \int_B^A - \dd[•]{q} \vec{E} \cdot \dd[•]{\vec{r}} = \dd[•]{q} \Delta V\]
esprimendo ora la ddp in funzione della carica già depositata
\[\Delta V = \frac{q}{C} \implies \Delta L_{est} = \frac{q \dd[•]{q}}{C}\]
da cui integrando su tutto il processo di carica
\[L_{est} = \int_0^Q \Delta L_{est} = \int_0^Q \frac{q \dd[•]{q}}{C} = \frac{Q^2}{2C} = \frac{1}{2} C \Delta V^2 = \frac{1}{2} Q \Delta V\]
Per l'energia potenziale elettrostatica si verifica infatti, considerato che le superfici sono equipotenziali ed è dunque possibile integrare sui contributi degli elementi di carica sulle armature
\[U = \frac{1}{2}(Q_A V_A + Q_B V_B) = \frac{1}{2} Q_A (V_A - V_B) = \frac{1}{2} Q \Delta V = \frac{1}{2} C \Delta V^2\]
che conferma il risultato.
\\Per un condensatore a facce piane parallele, in particolare
\[U_E = \frac{1}{2} C \Delta V^2 = \frac{1}{2} \frac{S \varepsilon_0}{d} (E d)^2 = \frac{1}{2} \varepsilon_0 E^2 (Sd) = \frac{1}{2} \varepsilon_0 E^2 \tau\]
ove $\tau$ è il volume del parallelepipedo in cui il campo è non nullo. Se si interpreta l'espressione come quella dell'energia contenuta nel campo (uniforme), si ottiene la \textbf{densità di energia elettrostatica} come
\[\mathcal{u}_E = \frac{U_E}{\tau} = \frac{1}{2} \varepsilon_0 E^2\]
si vedrà che tale espressione è ben più generale con il Teorema di Poynting. Per ora si consideri che per un volume qualsiasi in caso di campo non uniforme vale
\[U_E = \frac{1}{2} \varepsilon_0 \iiint\limits_\tau \vec{E} \cdot \vec{E} \dd[•]{\tau} = \frac{1}{2} \varepsilon_0 \iiint\limits_\tau |\vec{E}|^2 \dd[•]{\tau}\]

\section{Dipolo elettrico: campo e potenziale a grandi distanze}
Assumendo $r \gg a$, ove $a$ indica la distanza tra le cariche del dipolo e $r$ quella dall'origine del sistema di riferimento collocata nel suo punto medio, ed applicando il principio di sovrapposizione ai potenziali dovuti alle due cariche:
\[V(P) = V_+(P) + V_-(P) = \frac{1}{4 \pi \varepsilon_0} \frac{q}{r_+} + \frac{1}{4 \pi \varepsilon_0} \frac{q}{r_-} = \frac{q}{4 \pi \varepsilon_0} \frac{r_- - r_+}{r_- r_+}\]
Ora, nell'approssimazione è possibile assumere paralleli i due vettori di posizione relativa $\vec{r_+}$ e $\vec{r_-}$ (vd. giustificazione formale dopo) e dunque considerare la distanza tra la carica $+$ e la direzione di $\vec{r_-}$ ortogonale ad entrambi. Segue che l'angolo formato tra $a$ e $\vec{r_-}$ sia il medesimo $\theta$ tra l'asse delle cariche e $\vec{r}$. 
\\Dunque $|r_- - r_+| \approx a \cos \theta$ e $r_+ r_- \approx r^2$ da cui
\[V(P) \approx \frac{q}{4 \pi \varepsilon_0} \frac{a \cos \theta}{r^2} = \frac{1}{4 \pi \varepsilon_0} \frac{\vec{P} \cdot \vec{r}}{r^3}\]
Si osserva che 
\begin{itemize}
\item Il potenziale dipende dal momento di dipolo e non da $q$ e $a$ indipendentemente
\item Cala più rapidamente di quello della singola carica ($\propto 1/r^2$ anziché $1/r$)
\item Considerando l'angolo zenitale $\theta$ come definito
\begin{itemize}
\item per $\theta = \frac{\pi}{2}$, ovvero sull'asse trasverso, il potenziale è nullo (cambia di segno)
\item per $\theta < \frac{\pi}{2}$, ovvero nel semispazio $+$, il potenziale è positivo
\item per $\theta > \frac{\pi}{2}$, ovvero nel semispazio $-$, il potenziale è negativo
\end{itemize}
\end{itemize}

Si calcola quindi il campo utilizzando il gradiente in coordinate cartesiane, sostituendo secondo
\[z = r \cos \theta \qquad r = \sqrt{x^2 + y^2 + z^2} \implies \vec{P} \cdot \vec{r} = P z \implies V(x,y,z) = \frac{1}{4 \pi \varepsilon_0} \frac{Pz}{\sqrt{x^2 + y^2 + z^2}}\]
da cui
\[E_z = E_\parallel = - \pdv[•]{V}{z} = - \frac{P}{4 \pi \varepsilon_0} \bigg[\frac{1}{(x^2 + y^2 + z^2)^{3/2}} - \frac{3 z^2}{(x^2 + y^2 + z^2)^{5/2}}\bigg] = \frac{P}{4 \pi \varepsilon_0 r^3} \big(3 (\frac{z}{r})^2 - 1\big) = \]
\[= \frac{P}{4 \pi \varepsilon_0 r^3} (3 cos^2 \theta - 1)\]
Per la componente perpendicolare all'asse si ha quindi (non può essere presente una trasversale per simmetria):
\[E_\perp = \sqrt{E_x^2 + E_y^2} = \sqrt{\big(\pdv[•]{V}{x}\big)^2 + \big(\pdv[•]{V}{y}\big)^2} = \frac{P}{4 \pi \varepsilon_0} \bigg(\big(\frac{3zx}{(x^2 + y^2 + z^2)^{5/2}}\big)^2 + \big(\frac{3zy}{(x^2 + y^2 + z^2)^{5/2}}\big)^2\bigg)^{1/2} = \]
\[= \frac{P}{4 \pi \varepsilon_0} \frac{3z}{r^5} \sqrt{x^2 + y^2} = \frac{3P}{4 \pi \varepsilon_0} \frac{\cos \theta \sin \theta}{r^3}\]
Si ha quindi (sommando i contributi delle componenti ortogonali determinate)
\[|\vec{E}| = \frac{P}{4 \pi \varepsilon_0 r^3} \sqrt{3 \cos^2 \theta + 1}\]
da cui si ottiene che il campo si annulla per $\theta = \arccos \frac{1}{\sqrt{3}} \approx 54,7°$
\\Il modello chiaramente è valido solo nell'approssimazione a grandi distanze: in posizioni intermedie alle cariche o prossime al dipolo le linee divergono da quelle descritte dalle equazioni.

\section{Dipolo in campo esterno}
Si considera un dipolo immerso in campo elettrostatico qualsiasi. Fissata un'origine del SdR, si assume $r \gg a$ e si approssima l'energia potenziale del sistema
\[U_P = q V(\vec{r} + \vec{a}) - q V(\vec{r}) = q \big[V(x + a_x, y + a_y, z + a_z) - V(x,y,z)\big]\]
considerando il vettore $\vec{a}$ come spostamento infinitesimo $(\dd[•]{x}, \dd[•]{y}, \dd[•]{z})$
\[U_P \approx q \dd[•]{V} = q (\vec{\nabla} V \cdot \dd[•]{\vec{r}}) = q (- \vec{E} \cdot \vec{a}) = - q \vec{a} \cdot \vec{E} = - \vec{P} \cdot \vec{E} = - P E \cos \theta\]
in quanto $\dd[•]{V}$ differenziale esatto e il prodotto scalare è commutativo e lineare.
\\Segue che
\begin{itemize}
\item Il sistema si trova in condizioni di equilibrio \textbf{stabile} per energia minima, ovvero $\theta = 0$: il dipolo è allineato al campo. Se perturbato, tendera ad oscillare intorno alla posizione.
\item Il sistema si trova in condizioni di equilibrio \textbf{instabile} per energia massima, ovvero $\theta = \pi$: il dipolo è orientato in verso opposto (antiparallelo) al campo. Appena perturbato, tenderà a muoversi verso la posizione di equilibrio stabile.
\end{itemize}

Si osserva che il valore del campo è da considerarsi calcolato in un punto qualsiasi del segmento, in quanto nell'approssimazione effettuata si è operata la procedura di passaggio al limite con l'introduzione del differenziale, in cui una carica tende a sovrapporsi all'altra.
\\~\\
Nell'approssimazione, sul dipolo non si esercita alcuna forza risultante in quanto indipendentemente dalla sua uniformità globale il campo è assunto uniforme nell'intorno. 
\[\vec{F_T} = q \vec{E_+} - q \vec{E_-} \approx q (\vec{E} - \vec{E}) = \vec{0}\]
Si ha invece un momento, che causa il movimento rotatorio verso la posizione di equilibrio stabile. La forza risultante nulla permette di scegliere un polo a piacere: sia l'origine del SdR fissato.
\[\vec{\mathcal{M}_T} = \vec{r_+} \wedge \vec{F_+} + \vec{r_-} \wedge \vec{F_-} = (\vec{r_+} - \vec{r_-}) \wedge q \vec{E} = \vec{a} \wedge q \vec{E} = \vec{P} \wedge \vec{E}\]
per linearità del prodotto vettoriale. 
\[|\vec{\mathcal{M}}| = P E \sin \theta\]
ovvero
\[\mathcal{M} = \dv[•]{U_P}{\theta}\]
Dunque
\begin{itemize}
\item Il momento è nullo nelle due posizioni di equilibrio
\item Esso è massimo in $\pi/2$
\item Per simmetria del seno, esso tende a riportare il dipolo nella posizione $\theta = 0$ ed invece ad allontanarlo da $\theta = \pi$
\end{itemize}

\section{Distribuzione generica. Espansione in multipolo}
Si considera una distribuzione generica racchiusa in un volume di diametro $d$. Fissando l'origine del SdR al suo interno e considerando un punto $P$ di vettore posizione $\vec{r}$, si applica l'approssimazione $r \gg d$.
\\Per sovrapposizione in caso di distribuzione discreta
\[V(P) = \frac{1}{4 \pi \varepsilon_0} \sum\limits_{i=1}^{N} \frac{q_i}{r_i}\]
ora è possibile assumere $\vec{r_i} \parallel \vec{r}$ e dunque, detto $\vec{d_i}$ il vettore posizione di ciascuna carica
\[r - r_i = d_i \cos \theta_i = \vec{d_i} \cdot \hat{r}\]
da cui
\[\frac{1}{r_i} = \frac{1}{r - \vec{d_i} \cdot \hat{r}} \cdot \frac{r + \vec{d_i} \cdot \hat{r}}{r + \vec{d_i} \cdot \hat{r}} = \frac{r + \vec{d_i} \cdot \hat{r}}{r^2 - (\vec{d_i} \cdot \hat{r})^2}\]
ora, per Cauchy-Schwarz
\[(\vec{d_i} \cdot \hat{r})^2 \leq d_i^2\]
ma $d_i^2$ infinitesimo del II ordine (più precisamente $(d_i/r)^2$ lo è) e dunque è possibile trascurarlo. Il risultato ottenuto è analogo ad uno sviluppo in serie di Taylor al primo ordine
\[V(P) \approx \frac{1}{4 \pi \varepsilon_0} \sum\limits_{i=1}^{N} q_i \frac{r + \vec{d_i} \cdot \hat{r}}{r^2} = \frac{1}{4 \pi \varepsilon_0} \frac{Q}{r} + \frac{1}{4 \pi \varepsilon_0} \frac{1}{r^3} \big(\sum\limits_{i=1}^{N} q_i \vec{d_i}\big) \cdot \vec{r}\]
Il primo termine è un contributo equivalente a quello di una carica puntiforme nell'origine pari alla carica totale del sistema, indipendente dalla configurazione spaziale delle cariche a differenza del secondo. Esso è definito \textbf{termine di monopolo} ($V_0$). Chiaramente se il sistema è complessivamente neutro il termine di monopolo si annulla.
\\Definendo ora il \textbf{momento di dipolo elettrico del sistema} come
\[\vec{\mathbb{P}} = \sum\limits_{i=1}^{N} q_i \vec{d_i}\]
il secondo termine si riduce ad uno del tutto analogo al potenziale di un dipolo. Si definisce infatti \textbf{termine di dipolo} dello sviluppo ($V_{dip}$).
\\Si tratta solamente dei primi due termini di un'espansione in armoniche sferiche del potenziale generato dalla distribuzione. Ogni termine ha una dipendenza crescente dall'inverso della distanza ed è dunque trascurabile rispetto al precedente *. I coefficienti dell'espansione sono detti \textbf{momenti di multipolo} e ogni termine corrisponde all'approssimazione con un sistema di numero crescente di cariche:
\begin{itemize}
\item monopolo \dotfill $V \propto 1/r$
\item dipolo \dotfill $V \propto 1/r^2$
\item quadripolo \dotfill $V \propto 1/r^3$
\item ottopolo \dotfill $V \propto 1/r^4$
\item esadecapolo (16) \dotfill $V \propto 1/r^5$
\end{itemize}
Si osserva per i primi due termini
\[|\vec{\mathbb{P}}| \approx \sum_i q_i d_i \approx \sum_i q_i d = Q d \qquad |\vec{\mathbb{P}} \cdot \hat{r}| \approx |\vec{\mathbb{P}}|\]
e dunque
\[\frac{V_{dip}}{V_0} \approx \frac{d}{r} \ll 1\]
il che verifica quanto detto *.
\\~\\
Si assuma ora il corpo sia neutro, dunque il corrispondente termine di monopolo si annulli. Allora esplicitando
\[Q_T = 0 = Q^+ + Q^- = \sum_i q_i^+ - \sum_j |q_j^-| \implies \vec{\mathbb{P}} = \sum_i q_i^+ \vec{d_i^+} - \sum_j |q_j^-| \vec{d_i^-}\]
definendo il \textbf{centro delle cariche positive} e analogamente il centro delle cariche negative come i punti geometrici dati dalla media pesata delle posizioni delle cariche dei due segni:
\[\vec{d_+} = \frac{1}{Q} \sum_i q_i^+ \vec{d_i^+} \qquad \vec{d_-} = \frac{1}{Q} \sum_j |q_j^-| \vec{d_i^-}\]
si ha
\[\vec{\mathbb{P}} = Q(\vec{d_+} - \vec{d_-}) \equiv Q \vec{\delta}\]
Dunque se il sistema è complessivamente neutro \textbf{e i centri delle cariche positive e negative non coincidono} il potenziale (e dunque il campo) generato è approssimabile ad un dipolo per $r \gg d$ (in quanto i termini di ordine superiore divengono trascurabili). Tale approssimazione è indipendente dalla complessità del sistema.

\section{Dielettrici}
Si consideri un condensatore a facce piane parallele, lo si carichi e si assuma quindi di porre tra le armature corpi di diverso materiale.
\\~\\
Inserendo in primo luogo una lastra di conduttore (metallico) di superficie uguale a quella delle armature e spessore $D$, all'equilibrio questa escluderà il campo elettrico al suo interno e lo lascerà inalterato all'esterno (in quanto le facce sono normali al campo nel condensatore). Detta $d'$ la distanza dall'armatura positiva, integrando per ottenere la nuova ddp si ha
\[\Delta V = \int_0^d E \dd[•]{r} = \int_0^{d'} E \dd[•]{r} + \int_{d'}^{d'+D} E \dd[•]{r} + \int_{d'+D}^d E \dd[•]{r} = E(d+d' - (d'+D)) = E(d-D)\]
dunque $\Delta V < \Delta V_0$ a causa della presenza del conduttore ed \textit{indipendentemente dalla sua distanza dalle lastre}.
\\~\\
Se si inserisce invece uno strato di dielettrico, si assiste ad una riduzione della differenza di potenziale ma, a parità di spessore, questa è inferiore a quella osservata con il conduttore.
\\Aumentando lo spessore la ddp continua a diminuire, fino ad un valore minimo per $D' = d$ - ovvero quando il dielettrico occupa completamente lo spazio intermedio. Tale valore $\Delta V_k$ dipende dal materiale e, per quanto detto, si osserva sempre $0 < \Delta V_k / \Delta V_0 < 1$ (non si annulla a differenza di quanto avrebbe fatto con un conduttore, che all'equilibrio è equipotenziale al proprio interno).
\\~\\
Si definisce quindi la \textbf{costante dielettrica relativa del mezzo} come il rapporto
\[k \equiv \frac{\Delta V_0}{\Delta V_k} \implies k > 1 \textrm{ (1 per il vuoto)}\]
Permanendo la proporzionalità tra carica depositata e ddp, il sistema resta un condensatore e se ne può dunque definire la nuova capacità:
\[C_k = \frac{Q}{\Delta V_k} = k \frac{Q}{\Delta V_0} = k C_0 = (k \varepsilon_0) \frac{S}{d}\]
si definisce quindi la \textbf{costante dielettrica del mezzo} 
\[\varepsilon \equiv k \varepsilon_0 \implies C_k = \frac{\varepsilon S}{d}\]

\subsection{Perché diminuisce la ddp?}
Si osserva
\[\Delta V_k = \int_0^d \vec{E_k} \cdot \dd[•]{\vec{r}} = \frac{\Delta V_0}{k} = \frac{1}{k} \int_0^d \vec{E_0} \cdot \dd[•]{\vec{r}}\]
poiché i due integrali si equivalgono su ogni linea di campo ed i due campi sono entrambi costanti (paralleli e differenti solo in modulo) si può ricavare:
\[E_k = \frac{E_0}{k} = \frac{\sigma}{\varepsilon}\]
L'unica possibile spiegazione è che si abbia una differente densità di carica superficiale sulle armature, ovvero $\sigma_k$ t.c.
\[\frac{\sigma_k}{\varepsilon_0} = \frac{\sigma}{\varepsilon} \implies \sigma_k = \frac{\sigma}{k}\]
Si considera un cilindro gaussiano di area di base $A$ a cavallo della superficie di contatto tra armatura positiva e dielettrico. Applicando Gauss il contributo della faccia all'interno dell'armatura è nullo in quanto non vi è campo (conduttore), come anche quello della superficie laterale in quanto il campo è normale alle armature. Dunque
\[\Phi = \iint\limits_{S_d} \vec{E_k} \cdot \dd[•]{\vec{S_d}} = E_k \iint\limits_{S_d} \dd[•]{S_d} = E_k \cdot A = \frac{Q_T}{\varepsilon_0}\]
Tale carica è comprensiva di quella della superficie del conduttore intercettata \textbf{ma anche di quella sulla superficie del dielettrico}, che scherma il campo all'interno \textbf{ma non completamente come in un conduttore}. Si può affermare che la carica è concentrata sulla sola superficie di contatto in quanto variando l'altezza del cilindro gaussiano il flusso resta invariato pur racchiudendo un volume differente del dielettrico. Si ha quindi
\[\sigma_k = \sigma - \sigma_P\]
ove $\sigma_P$ è la densità di \textbf{carica} negativa \textbf{di polarizzazione} indotta sulla superficie del dielettrico. La schermatura non completa si traduce in $\sigma_k > 0 \Longleftrightarrow \sigma > \sigma_P$. Inoltre
\[E_k = \frac{\sigma}{k \varepsilon_0} = \frac{\sigma}{\varepsilon_0} \big(\frac{1}{k} + 1 - 1\big) = \frac{\sigma}{\varepsilon_0} \big(1 - \frac{k-1}{k}\big) = \frac{\sigma}{\varepsilon_0} - \frac{\sigma}{\varepsilon_0}\big(\frac{k-1}{k}\big) = \frac{\sigma_k}{\varepsilon_0}\]
da cui
\[\sigma_P = \sigma \big(\frac{k-1}{k}\big) \implies \sigma_P < \sigma\]

\section{Origine microscopica della dielettricità}
Gli atomi sono ordinariamente neutri. Inoltre in condizioni normali per simmetria della nuvola elettronica il centro di carica positiva e negativa coincidono; in altre parole il momento di dipolo dell'atomo è nullo. 
\\In presenza di un campo elettrico esterno, però, la forza esercitata sugli elettroni porta ad una asimmetria degli orbitali (ovvero delle funzioni d'onda) con conseguente allontanamento dei due centri di carica. L'atomo acquisisce un momento di dipolo
\[\vec{P} = q \vec{\delta} = Z e \vec{\delta}\]
A distanze sufficientemente grandi in rapporto alla scala atomica (ovvero quelle cui fanno riferimento le considerazioni macroscopiche precedenti), gli atomi sono di fatto perfettamente assimilabili a dipoli - in quanto il termine di dipolo nello sviluppo è dominante e quello di monopolo non è presente essendo neutri.
\\Si definisce quindi il \textbf{vettore polarizzazione} come il momento di dipolo risultante per unità di volume
\[\vec{\mathbb{P}} = n \vec{P} = n q \vec{\delta}\]
Le cariche dei dielettrici non sono libere ma vincolate ai singoli atomi, e ciò comporta che a differenza dei conduttori questi non siano in grado di schermare completamente il campo all'interno.

\subsection{Polarizzazione uniforme}
Nel caso di p. uniforme, ovvero di un dielettrico \textbf{omogeneo} sottoposto a campo uniforme, considerando strati di spessore $\delta$ le cariche positive di ogni strato di dipoli sono compensate da quelle negative del successivo: per la densità volumetrica vale $\rho_P = 0$. Gli unici strati non neutri sono quelli superficiali, per cui si ha una densità superficiale
\[\sigma_P = nq \delta = |\vec{\mathbb{P}}|\]
in quanto data dalla densità volumetrica moltiplicata per lo spessore dello strato. Si osserva in particolare che per lo strato superficiale caricato negativamente gli elettroni sono attratti \textbf{all'esterno} del volume del materiale, mentre per quello caricato positivamente sono attratti all'interno e dunque le cariche positive, ovvero le porzioni di atomi in difetto di elettroni, sono \textbf{all'interno} del volume.
\\~\\
In caso di campo elettrico non perpendicolare alle facce del dielettrico, rispetto alla cui normale forma e.g. un angolo $\theta < \pi/2$, ma comunque uniforme, lo spessore degli strati diviene $h = \delta \cos \theta$ e dunque
\[\sigma_P = n q \delta \cos \theta = |\vec{\mathbb{P}}| \cos \theta = \vec{\mathbb{P}} \cdot \hat{n}\]
che è la relazione più generale per polarizzazione uniforme.
\\~\\
\'E possibile quindi ricavare la relazione tra la polarizzazione ed il campo elettrico (:
\[\vec{E} = \frac{\sigma - \sigma_P}{\varepsilon_0} \hat{n} = \vec{E_0} - \frac{\vec{\mathbb{P}}}{\varepsilon_0} \implies \vec{\mathbb{P}} = \varepsilon_0(\vec{E_0} - \vec{E})\]
Sostituendo
\[E = \frac{E_0}{k} \implies \vec{\mathbb{P}} = \varepsilon_0 (k-1) \vec{E} = \varepsilon \chi \vec{E}\]
ove si è introdotta la \textbf{suscettività elettrica} $\chi$. Tale relazione vale per tutti e soli i \textbf{dielettrici lineari} (in particolare non solo puntualmente, per precisione, se anche omogenei). Nel caso più generale, che comprende anche i dielettrici \textbf{anisotropi}, si ha la relazione matriciale (utilizzando convenzione di Einstein):
\[P_i = \varepsilon_0 \chi_{ij} E_j\]
con il \textbf{tensore di suscettività}, ovvero una matrice 3x3.

\subsection{Caso non omogeneo}
In caso di polarizzazione non omogenea, si può chiaramente avere la presenza di addensamenti di carica anche all'interno del materiale. Considerando una superficie gaussiana all'interno, una densità di carica netta non nulla in punti interni indica che durante il processo di polarizzazione si ha un flusso netto non nullo di carica verso l'interno o l'esterno della superficie (per conservazione). Ora, poiché nei dielettrici non si hanno cariche libere, la carica che può attraversare $\Sigma$ risiede entro una distanza $\delta$ dalla superficie. Per una porzione infinitesima 
\[\dd[•]{q} = \sigma_P \dd[•]{S} = (\vec{\mathbb{P}} \cdot \hat{n}) \dd[•]{S} = \vec{\mathbb{P}} \cdot \dd[]{\vec{S}} = \dd[•]{\Phi}_\Sigma(\vec{\mathbb{P}})\]
dunque poiché la carica totale che attraversa la superficie è pari all'opposto (per orientamento esterno) di quella accumulata al termine della polarizzazione si ha
\[\Delta Q = - \oiint\limits_\Sigma \vec{\mathbb{P}} \cdot \dd[]{\vec{S}}\]
applicando il teorema della divergenza
\[\oiint\limits_\Sigma \vec{\mathbb{P}} \cdot \dd[]{\vec{S}} = \iiint\limits_{\tau(\Sigma)} (\vec{\nabla} \cdot \vec{\mathbb{P}}) \dd[]{\tau}\]
ma per definizione
\[\Delta Q = \iiint\limits_{\tau(\Sigma)} \rho_P \dd[]{\tau}\]
essendo l'uguaglianza indipendente dal volume considerato, si impone quella degli integrandi ottenendo l'espressione puntuale:
\[\vec{\nabla} \cdot \vec{\mathbb{P}} = - \rho_P\]
Dalla legge si verifica nuovamente che per polarizzazione uniforme la densità volumetrica è nulla.

\section{Il campo $\vec{D}$}
Le leggi di Gauss e della circuitazione per il campo elettrostatico valgono in generale per campi generati da cariche libere o di polarizzazione. Dalle espressioni ricavate è possibile ottenere in forma differenziale:
\[\vec{\nabla} \cdot \vec{E} = \frac{\rho_{tot}}{\varepsilon_0} = \frac{\rho_f + \rho_P}{\varepsilon_0} = \frac{\rho_{f}}{\varepsilon_0} - \frac{1}{\varepsilon_0} \vec{\nabla} \cdot \vec{\mathbb{P}}\]
da cui per linearità della divergenza
\[\rho_f = \varepsilon_0 \vec{\nabla} \cdot \vec{E} + \vec{\nabla} \cdot \vec{\mathbb{P}} = \vec{\nabla} \cdot (\varepsilon_0 \vec{E} + \vec{\mathbb{P}})\]
Si introduce ora il vettore \textbf{spostamento elettrico} definito secondo
\[\vec{D} = \varepsilon_0 \vec{E} + \vec{\mathbb{P}}\]
($D$ per displacement) per il quale si osserva subito che la legge di Gauss diviene
\[\vec{\nabla} \cdot \vec{D} = \rho_f \, \Leftrightarrow \, \oiint\limits_{\Sigma} \vec{D} \cdot \vec{\dd[•]{S}} = Q_f\]
Si tratta di uno strumento matematico (più che un campo con realtà fisica propria) che permette di considerare isolatamente gli effetti delle sole cariche libere. Dunque dentro un dielettrico
\[\vec{\nabla} \cdot \vec{D} = 0\]
e il flusso attraverso qualsiasi superficie chiusa è nullo.
\\~\\
Per dielettrici isotropi e lineari, si ottengono quindi ulteriori relazioni
\[\vec{D} = \varepsilon \vec{E} \qquad \vec{\mathbb{P}} = \frac{k-1}{k} \vec{D}\]
mentre per gli omogenei, per i quali dunque $k$ è uniforme e può essere portata fuori dal prodotto scalare con l'operatore di divergenza
\[\vec{\nabla} \cdot \vec{\mathbb{P}} = \frac{k-1}{k} \vec{\nabla} \cdot \vec{D}\]
In realtà si possono avere relazioni analoghe nel caso anisotropo, a patto che il tensore di suscettività sia invertibile.

\subsection{Altre considerazioni sul caso anisotropo}
Generalmente il tensore di suscettività gode di alcune importanti proprietà:
\begin{itemize}
\item \'E reale, ovvero $\chi_{ij} \in \mathbb{R} \, \forall i,j \in [3]$
\item \'E simmetrico, ovvero $\chi_{ij} = \chi_{ji} \, \forall i,j \in [3] \, i \neq j$
\end{itemize}
Soddisfa dunque le ipotesi del teorema spettrale, da cui segue che ammette una base spettrale ortogonale rispetto alla quale è espresso in forma diagonale. Gli autovettori della base individuano una terna di \textbf{assi ottici (o cristallografici)} del dielettrico che sono, appunto, mutualmente ortogonali. Chiaramente se il campo esterno è diretto parallelamente ad uno degli assi si avrà una polarizzazione parallela ad esso.
\\Nel caso particolare di un dielettrico isotropo le proprietà del tensore restano valide, ma esso è banalmente in forma diagonale come multiplo dell'identità. Si ha 
\[\chi_{ij} = \chi \delta_{ij}\]
Le proprietà viste sono alla base dei materiali cosiddetti birifrangenti, nei quali la velocità di propagazione della luce dipende dall'angolo di incidenza (come la calcite).

\section{Superficie di contatto tra due dielettrici}
Si studia l'andamento dello spostamento elettrico costruendo un cilindro gaussiano a cavallo della superficie tra due dielettrici indicati con $1$ e $2$.
\\La carica sulla superficie è chiaramente solo di polarizzazione, dunque $\sigma_f = 0$ e di conseguenza per la legge di Gauss
\[\oiint\limits_\Sigma \vec{D} \cdot \dd[]{\vec{S}} = 0\]
Passando ora al limite $\dd[•]{h} \rightarrow 0$ per lo spessore il contributo della superficie laterale tende ad annullarsi e dunque si ha
\[\Phi = \vec{D_1} \cdot \dd[]{\vec{S_1}} + \vec{D_2} \cdot \dd[]{\vec{S_2}} = D_1 \dd[•]{A} \cos \theta_1 + D_2 \dd[•]{A} \cos \theta_2 = 0 \implies D_1 \cos \theta_1 + D_2 \cos \theta_2 = 0\]
fissando ora il versore normale $\hat{n}$ si può riformulare
\[- \vec{D_1} \cdot \hat{n} + \vec{D_2} \cdot \hat{n} = 0 \implies \Delta D^n = 0\]
ovvero la componente normale dello spostamento elettrico è continua attraversando la superficie.
\\Per quanto visto la componente tangente del campo $\vec{E}$ è anch'essa continua mentre è discontinua quella normale. Unendo le condizioni:
\[D_1^n = \varepsilon_0 k_1 E_1^n = D_2^n = \varepsilon_0 k_2 E_2^n \qquad E_2^n - E_1^n = \frac{\sigma_P}{\varepsilon_0} \qquad E_1^\tau = E_2^\tau\]
si ottiene il sistema
\[
\begin{cases}
E_1 \sin \theta_1 = E_2 \sin \theta_2 \\
\\
E_1 \cos \theta_1 k_1 = E_2 \cos \theta_2 k_2
\end{cases}
\]
da cui, dividendo membro a membro
\[\frac{\tan \theta_1}{k_1} = \frac{\tan \theta_2}{k_2} \, \Longleftrightarrow \, \frac{\tan \theta_1}{\tan \theta_2} = \frac{k_1}{k_2}\]
da cui è possibile predire esattamente la variazione della direzione del campo elettrico. Nota anche la variazione del modulo, si ha una conoscenza completa delle condizioni su superfici cariche. Poiché $k > 1$ per qualunque materiale, la legge implica che l'angolo $\theta_2$ di rifrazione sia maggiore di quello di incidenza $\theta_1$ se all'esterno si ha il vuoto: le linee di forza rifratte dunque divergono.
\\Si osserva che per angolo di incidenza nullo si ha solo variazione del modulo.
\\~\\
Quanto ottenuto è alla base della \textbf{Legge di Snell} per la rifrazione della luce, originariamente solo fenomenologica prima delle scoperte di Maxwell, che è interamente spiegata dalle proprietà del campo. 
\\~\\
La configurazione delle linee di campo incidenti, come detto in precedenza, permette quindi di distinguere dielettrici e conduttori.

\section{Energia del campo}
Considerando il condensatore a facce piane parallele con dielettrico, si ha
\[C_k = k C_0 = \frac{\varepsilon S}{d} \qquad E_k = \frac{E}{k} = \frac{\sigma}{\varepsilon}\]
dunque
\[U_k = \frac{1}{2} \frac{(\sigma S)^2}{C_k} = \frac{1}{2} \frac{\sigma^2 S d}{\varepsilon} = \frac{1}{2} \varepsilon \big(\frac{\sigma}{\varepsilon}\big)^2 S d\]
da cui per la densità energetica
\[\mathcal{u}_E = \frac{1}{2} \varepsilon \big(\frac{\sigma}{\varepsilon}\big)^2 = \frac{1}{2} \varepsilon E_k^2 = \frac{1}{k} \mathcal{u}_0\]
Essendo $k > 1$, a parità di carica sulle superfici si ha una densità di energia minore nel campo: la differenza è infatti spesa per polarizzare il dielettrico (lavoro positivo della forza elettrostatica per creare i dipoli).
\\A parità di campo, analogamente, si ha un'energia totale maggiore per il medesimo motivo. Infatti per separare i singoli centri di carica in un'unità di volume
\[\dd[•]{W}_{singolo} = E q \dd[•]{x} = E \dd[•]{P} \implies \dd[•]{W}_\tau = E \dd[•]{\mathbb{P}} = \varepsilon_0 (k-1) E \dd[•]{E} \implies W_\tau = \int_t \dd[•]{W} = \frac{1}{2} \varepsilon_0 (k-1) E^2 \]
con l'integrale fatto sul processo di caricamento del condensatore. Poiché l'energia spesa per caricare le armature e dunque immagazzinata nell'unità di volume in cui è presente il campo generato dalle cariche cariche su queste è invece 
\[W_{C} = \frac{1}{2} \varepsilon_0 E^2\]
si ritrova l'espressione ottenuta.
\\Generalizzando a dielettrici anisotropi vale
\[\mathcal{u}_E = \frac{1}{2} \vec{D} \cdot \vec{E}\]

\section{Correnti}
Nei conduttori metallici gli atomi risultano facilmente ionizzabili; infatti le bande di conduzione sono facilmente accessibili agli elettroni di valenza che divengono liberi. Si sono viste cifre indicative per il rame; in generale il numero di \textbf{portatori di carica} per unità di volume è nell'ordine di $10^{28} e^- \meters{•}{-3}$
\\~\\
In assenza di un campo esterno, gli elettroni sono soggetti alla sola agitazione termica e dunque si ha una distribuzione isotropa delle velocità. Per un volume sufficientemente grande da contenere $N$ statisticamente significativo:
\[\vec{v_m} = \frac{1}{N} \sum\limits_{i=1}^{N} \vec{v_i} = \vec{0}\]
Applicando il modello cinetico (di un gas di elettroni) si ha
\[\frac{1}{2} m_e \overline{v^2} = \mean{\frac{1}{2} m_e v^2} = \frac{3}{2} k_B T\]
da cui per $T = 300 \kelvin{•}$, $m_e = 9,1 \times 10^{-31} \mathrm{kg}$
\[v_{rms} = \sqrt[•]{\overline{v^2}} \approx 1,16 \times 10^{5} \meters{•}{•}\sec^{-1} \approx 120 \meters{k}{•}/\mathrm{h}\]
Applicando invece il più corretto modello quantistico di Sommerfeld, basato sulla statistica di Fermi-Dirac (per gas di particelle a spin semintero, per cui vale dunque Pauli) si ottiene un differente spettro energetico. Facendo riferimento all'\textbf{energia di Fermi}, ovvero la banda massima accessibile a 0K, che corrisponde al limite inferiore della banda di conduzione, si ha
\[E_F \approx 7,1 \evolt \implies v_{rms} \approx 1580 \meters{k}{•}/\mathrm{h}\]
che è un valore ben maggiore della stima classica.
\\Gli elettroni nella nuvole durante il loro moto urtano contro i nuclei fissi nei nodi del reticolo cristallino; essendo il rapporto di massa tra elettroni e protoni $\sim 10^{-3}$ tali urti possono essere considerati con ottima approssimazione anelastici.
\\~\\
Applicando ora un campo esterno, si ha una redistribuzione della nuvola elettronica (in cui persiste comunque il moto di agitazione); tuttavia la \textbf{corrente}, ovvero il moto orientato di cariche, dura per il solo tempo caratteristico di induzione, al termine del quale si ha equilibrio.
\\Utilizzando invece un \textbf{generatore di forza elettromotrice (f.e.m.)}, ovvero un dispositivo in grado di mantenere costante il campo all'interno del conduttore e dunque la ddp ai suoi capi, si ottiene un moto continuo. Le cariche attraversano il conduttore e si addensano ai capi del generatore, che le trasferisce da uno all'altro mantenendo costante la ddp.
\\Si instaura dunque un regime di \textbf{equilibrio dinamico}, in cui il moto orientato dei portatori è costante nel tempo: si ha il fenomeno della \textbf{conduzione elettrica} (che avviene anche in condizioni di non stazionarietà) e la corrente elettrica è \textbf{stazionaria}.
\\$\vec{v_m} \neq \vec{0}$: si ha una direzione privilegiata per il moto dei portatori, ed è quella del campo applicato, con verso concorde o discorde a seconda del segno della carica. I portatori possono essere infatti
\begin{itemize}
\item Nei conduttori non metallici ioni (elettroliti nell'acqua o altri in gas con agenti ionizzanti che favoriscano la conducibilità) ed elettroni
\item Nei semiconduttori, ovvero silici (naturalmente dielettrici) drogati sia cariche libere positive che negative. Il prefisso "semi" deriva dalle caratteristiche delle bande energetiche e dal differente ordine di grandezza per la densità di portatori: $n \sim 10^{15}$!
\end{itemize}
A campo costante, per il II principio della dinamica i portatori sono accelerati uniformemente con
\[\vec{a} = \frac{q}{m} \vec{E}\]
Essi però urtano anelasticamente contro i nuclei e dopo ogni collisione riaccelerano in direzione del campo; gli urti sono isotropi e l'anelasticità implica la cancellazione dell'informazione sull'accelerazione precedente. Si tratta dunque di una situazione concettualmente analoga all'attrito viscoso: a livello macroscopico si osserva il rallentamento del moto del 'fluido' di elettroni, che raggiunto l'equilibrio dinamico procede alla velocità limite costante, denominata \textbf{velocità di deriva} $\vec{v_m} = \vec{v_d}$.
\\Nel caso di conduttori metallici, il baricentro della nuvola elettronica si muove quindi di moto uniforme in direzione del campo e verso opposto.
\\L'effetto complessivo degli urti che porta al moto di deriva è detto \textbf{effetto di resistenza}; esso è assente nei cosiddetti materiali superconduttori.

\section{Intensità di corrente}
Si consideri un conduttore cilindrico attraversato da corrente costante, supponendo positivo il segno dei portatori di carica. Detta $\Sigma$ una superficie aperta corrispondente alla sezione trasversale del conduttore, si definisce l'\textbf{intensità di corrente} come la carica per unità di tempo che vi transita attraverso:
\[i \equiv \limit{\Delta t}{0} \frac{\Delta Q}{\Delta t} = \dv[•]{q}{t}\]
L'unità di misura è l'Ampére (A), equivalente a C/m.
\\La convenzione comunemente adottata per la corrente nei conduttori metallici è quella di assumere positivo il segno dei portatori, anche se praticamente sono gli elettroni a muoversi: le descrizioni con i due segni sono assolutamente equivalenti, ma questa è più pratica concettualmente.
\\~\\
Considerando più generalmente una superficie trasversale, non necessariamente ortogonale al moto delle cariche, si consideri un suo elemento infinitesimo e si indichi con $\theta$ l'angolo tra la sua normale ed il campo elettrico. Il numero di cariche che attraversano $\dd[•]{S}$ in $\Delta t$ è pari a quelle presenti in un cilindro con questa come base e altezza $v_d \Delta t$. Ora, il volume del cilindro è
\[\tau = (v_d \Delta t) \dd[•]{S} \cos \theta\]
e dunque
\[\Delta q = N_\tau e = n e \tau = n e (v_d \Delta t) \dd[•]{S} \cos \theta \implies \dd[•]{i} = n e v_d \dd[•]{S} \cos \theta = n e \vec{v_d} \cdot \dd[•]{\vec{S}}\]
introducendo ora il vettore \textbf{densità di corrente}
\[\vec{j} \equiv n e \vec{v_d} \implies \dd[•]{i} = \dd[•]{\Phi(\vec{j})}\]
Analoga alla densità di flusso diffusivo, di corrente o termico, la densità di corrente descrive il moto della quantità di carica nel volumetto differenziale.
\\Dunque
\[i = \iint\limits_\Sigma \dd[•]{i} = \iint\limits_\Sigma \vec{j} \cdot \dd[•]{\vec{S}}\]
Si osserva ora che la densità di corrente permette di verificare l'equivalenza dei modelli di conduzione metallica a portatori positivi e negativi:
\begin{itemize}
\item se positivi, la carica è $+e$ e la velocità di deriva $+\vec{v_d}$, da cui $\vec{j} = n e \vec{v_d}$
\item se negativi, la carica è $-e$ e la velocità di deriva $-\vec{v_d}$, da cui $\vec{j} = n e \vec{v_d}$ analogamente
\end{itemize}

Le misure di corrente non permettono infatti di determinare il segno dei portatori: non vi sono effetti macroscopici differenti (salvo e.g. l'effetto Hall, che si vedrà).

\infobox{Una stima della $v_d$}{
Si consideri un filo di rame di sezione circolare, percorso da una corrente di 15 A. Poiché la densità è uniforme sulla sezione
\[i = j \pi R^2 \implies v_d = \frac{i}{n e \pi R^2}\]
per $n \sim 8,5 \times 10^28$ $e^-$/m${}^3$, R = 0,8mm si ha una densità volumetrica di carica di conduzione (*) di 13,6 Coulomb per mm cubo (estremamente elevata) ma una velocità di deriva
\[v_d \approx 5 \times 10^{-4} \meters{•}{•}/\sec = 2 \meters{•}{•}/\mathrm{h}\]
un valore estremamente inferiore alla velocità quadratica media per agitazione termica.
\\~\\
(*) considerando infatti anche gli atomi ionizzati il conduttore è complessivamente neutro anche localmente.
}

\subsection{Principio di continuità}
Si tratta di un principio equivalente alla conservazione della carica in presenza di correnti elettriche. Considerando una superficie chiusa $\Sigma$ attraversata da densità di corrente, per il flusso attraverso un elemento infinitesimo si ha
\[\dd[•]{\Phi} = \dd[•]{i} = \vec{j} \cdot \dd[•]{\vec{S}} = j \dd[•]{S} \cos \theta\]
con $\theta \, \in \, [0, \pi]$. Se la corrente è tangente, ovvero $\theta = \pi/2$, si ha $\dd[•]{i} = 0$; se invece è uscente ($\theta < \pi/2$) $\dd[•]{i} > 0$; se ancora è entrante ($\theta > \pi/2$) si ha $\dd[•]{i} < 0$. Si osserva che con portatori di carica negativi una corrente uscente corrisponde a portatori entranti e viceversa una entrante a portatori uscenti.
\\Dunque integrando sulla superficie
\[\Phi = \oiint\limits_\Sigma \dd[•]{i} = i\]
Supposto e.g. $i > 0$ per conservazione della carica essendovi una corrente netta uscente la carica nel volume definito da $\Sigma$ deve \textbf{diminuire} (o equivalentemente deve aumentare in modulo la carica negativa). Si ottiene così il seguente principio

\lawbox{Equazione di continuità della corrente}{i = \oiint\limits_\Sigma \vec{j} \cdot \vec{\dd[•]{S}} = - \dv[•]{q_{int}}{t}}

Nel caso $i = 0$ non si ha alcuna dipendenza temporale della carica racchiusa: si è in \textbf{regime stazionario} (tipico dei circuiti CC), in cui non si hanno nè accumuli ne diminuzioni di carica.
\\Esprimendo ora la carica come integrale della densità:
\[\oiint\limits_\Sigma \vec{j} \cdot \vec{\dd[•]{S}} = - \dv[•]{}{t} \big(\iiint\limits_{\tau(\Sigma)} \rho \dd[•]{\tau}\big)\]
Assumendo che la superficie (e quindi il volume) non varino nel tempo è possibile portare la derivata dentro l'integrale:
\[- \dv[•]{}{t} \big(\iiint\limits_{\tau(\Sigma)} \rho \dd[•]{\tau}\big) = \iiint\limits_{\tau(\Sigma)} \pdv[•]{\rho}{t} \dd[•]{\tau}\]
applicando ora il teorema della divergenza all'integrale di superficie:
\[\oiint\limits_\Sigma \vec{j} \cdot \vec{\dd[•]{S}} = \iiint\limits_{\tau(\Sigma)} (\vec{\nabla} \cdot \vec{j}) \dd[•]{\tau}\]
l'uguaglianza vale indipendentemente dalla superficie (ovvero dal volume) considerato: si impone dunque l'uguaglianza locale degli integrandi

\lawbox{Equazione di continuità della corrente elettrica}{\vec{\nabla} \cdot \vec{j} = - \pdv[•]{\rho}{t}}

\section{Modello di Drude e legge di Ohm}
Il modello classico della condizione fu proposto da Drude nel 1900 e perfezionato da Lorentz nel 1906. In esso, come visto anche in precedenza, gli ioni sono assunti fissi sui vertici del reticolo cristallino e gli elettroni di conduzioni sono liberi di muovervisi intorno, interagendovi tramite urti. Il moto libero degli elettroni è rettilineo ed in assenza di campo esterno è casuale: non si ha complessivamente alcun flusso.
\\Considerando la velocità prima e dopo ogni urto di un singolo elettrone si ha
\[|\vec{v}_i| = |\vec{v}_{i+1}| = \mathcal{v} = \frac{l}{\tau}\]
con gli urti considerabili anelastici e $l$ corrispondente al mfp e $\tau$ al tempo medio di interazione.
\\~\\
Applicando un campo esterno, gli elettroni sono sottoposti ad una forza costante $\vec{F} = m_e \dv[•]{\vec{v}}{t} = - e \vec{E}$. Mediando ora sugli urti del singolo elettrone, si osserva ora che la velocità risultante non è più nulla, in quanto è presente una direzione di moto privilegiata (quella del campo):
\[\mean{\vec{v_{i+1}}} = \underbrace{\frac{1}{N} \sum_j \vec{v_j}}_{\vec{0}} - \frac{e \vec{E}}{m} \tau = \frac{e \vec{E}}{m} \tau \equiv \vec{v_d}\]
ove il primo contributo si annulla in quanto la distribuzione delle velocità tra gli urti per agitazione termica rimane casuale e dunque isotropa (per cancellazione dell'informazione). Si ottiene così la

\lawbox{Legge di Ohm}{\vec{j} = \frac{n e^2 \tau}{m} \vec{E}}

Si osserva che $\tau$ è inversamente proporzionale alla densità del materiale, mentre $n e$ indica la densità di carica libera disponibile per la conduzione. Inoltre essendo $e$ presente al quadrato la legge è indipendente dal segno dei portatori.
\\Definendo ora la \textbf{conduttività}, che quantifica localmente la facilità con cui un materiale conduce corrente in presenza di campo esterno:
\[\sigma \equiv \frac{n e^2 \tau}{m}\]
si ha
\[\vec{j} = \sigma \vec{E}\]
per i semiconduttori, con portatori di entrambi i segni, la legge diviene:
\[\vec{j} = ne^2 \big(\frac{\tau_{+}}{m_{+}} + \frac{\tau_{-}}{m_{-}}\big) \vec{E}\]
definendo invece la \textbf{resistività}, che quantifica localmente la tendenza del materiale ad opporsi alla conduzione di corrente:
\[\rho \equiv \frac{1}{\sigma} \implies \vec{E} = \rho \vec{j}\]

\subsection{Potenza erogata}
L'applicazione del campo elettrico modifica l'energia del conduttore, in quanto la potenza sviluppata dalla forza elettrica è fornita alle cariche che la dissipano negli urti - che producono infatti calore poi ceduto all'ambiente.
\\Per singola carica:
\[P = \dv[•]{W}{t} = \frac{\vec{F} \cdot \dd[•]{\vec{s}}}{t} = \underbrace{\dv[•]{\vec{F}}{t}}_{0} \cdot \dd[•]{\vec{s}} + \vec{F} \cdot \dv[•]{\vec{s}}{t} = \vec{F} \cdot \vec{v}\]
in quanto il campo è costante nel tempo. Ora per unità di volume vale $\frac{1}{\tau}(\sum \vec{v}) = n \vec{v_d}$ da cui
\[P_\tau = n (\vec{F} \cdot \vec{v_d}) = n (e \vec{E}) \cdot \vec{v_d} = \vec{j} \cdot \vec{E}\]
Applicando quindi Ohm:
\[P_\tau = \sigma E^2 = \rho j^2\]

\subsection{Filo metallico}
Considerando nuovamente un conduttore cilindrico metallico di lunghezza $l$ e sezione $S$, con campo costante all'esterno, per la ddp agli estremi si ha:
\[V(A) - V(B) = \Delta V = \int_A^B \vec{E} \cdot \dd[•]{\vec{l}} = E l\]
per Ohm se il materiale è omogeneo il campo uniforme implica una densità di corrente uniforme: integrando su una sezione trasversale:
\[i = \iint\limits_\Sigma \vec{j} \cdot \vec{\dd[•]{S}} = j S \implies \frac{\Delta V}{l} = \rho \frac{i}{S}\]
ove nell'ultima eguaglianza si è applicata la legge di Ohm. Si ha dunque $\ds \Delta V = \frac{\rho l}{S} i$, ove il fattore di proporzionalità dipende esclusivamente dalla geometria ($l$ e $S$) e dalla composizione ($\rho$) del conduttore.
\\Si definisce \textbf{resistenza} (elettrica) e quantifica globalmente l'inerzia del conduttore al passaggio di corrente:
\[R \equiv \frac{\rho l}{S}\]
si ha quindi la

\lawbox{Legge di Ohm per conduttori metallici}{\Delta V = R i}

L'unità di misura è l'ohm ($\Omega$). Un ohm è la resistenza di un conduttore in cui scorre una corrente di 1A applicando una ddp ai capi di 1V. Si osserva che:
\begin{itemize}
\item La resistenza è più alta, a parità di altri fattori, per materiali più resistivi $R \propto \rho$
\item Aumenta con la lunghezza del conduttore (un maggiore percorso da percorrere implica più interazioni) $R \propto l$
\item Diminuisce aumentando la sezione (si ha un maggiore flusso a parità di campo applicato) $R \propto 1/S$
\end{itemize}

\subsection{Regime stazionario}
Valgono le condizioni di stazionarietà, ottenute imponendo l'annullamento delle derivate temporali:
\[
\begin{cases}
\vec{\nabla} \cdot \vec{j} = 0 & \textrm{in ogni punto}\\
\\
\oiint\limits_\Sigma \vec{j} \cdot \vec{\dd[•]{S}} = 0 & \textrm{attraverso ogni sup. chiusa}
\end{cases}
\]
Considerando un conduttore a sezione variabile e considerandone una porzione cilindrica di spessore infinitesimo $\dd[•]{l}$ e area di base $\Sigma$ si ha 
\[\dd[•]{R} = \frac{\rho \dd[]{l}}{\Sigma} \implies R = \int_A^B \frac{\rho \dd[]{l}}{\Sigma}\]
Integrando quindi sul cilindro, con il contributo della superficie laterale nullo perché la densità di corrente vi è tangente:
\[\oiint\limits_\Sigma \vec{j} \cdot \vec{\dd[•]{S}} = 0 = \iint\limits_{\Sigma_1} \vec{j_1} \cdot \vec{\dd[•]{S_1}} + \iint\limits_{\Sigma_2} \vec{j_2} \cdot \vec{\dd[•]{S_2}} = - i_1 + i_2\]
con i segni determinati dall'orientamento esterno delle superfici. Dunque in condizioni di stazionarietà
\[i_1 = i_2\]
ovvero la corrente è uniforme e dunque vale generalmente la legge di Ohm
\[\Delta V = \int_A^B - \vec{E} \cdot \dd[•]{\vec{l}} = \int_A^B \frac{\rho \dd[]{l}}{\Sigma} i = Ri\]

\subsection{Agitazione termica}
Per conduttori metallici a temperature nel range $20\celsius \pm 10\celsius$ vale
\[\rho = \rho_{20} (1 + \alpha \Delta T)\]
con $\alpha$ coefficiente termico specifico. Si osserva che $\ds \dv[•]{\rho}{T} = \alpha \rho_{20}$ può essere positivo o negativo a seconda dei materiali (e.g. $<0$ per il carbonio); dunque scaldando (limitatamente) nel secondo caso la resistenza diminuisce anziché crescere.
\\~\\
Per la potenza dissipata per \textbf{effetto Joule} considerando sempre la porzione cilindrica si ha
\[\dd[•]{P} = P_\tau \Sigma \dd[•]{h} = \rho j^2 \Sigma \dd[•]{h} = \rho \frac{i^2}{\Sigma^2} \Sigma \dd[•]{h} = \frac{\rho \dd[•]{h}}{\Sigma} i^2\]
Integrando
\[P = \int_A^B \dd[•]{P} = i^2 \int_A^B \frac{\rho \dd[•]{h}}{\Sigma} = R i^2\]
analogamente per la potenza erogata
\[\dd[•]{W} = V \dd[•]{q} = V i \dd[•]{t} \implies P = i V\]
infatti in condizioni stazionarie tutta la potenza erogata è dissipata in quanto il moto di deriva degli elettroni è uniforme. Per il lavoro a corrente costante si ha quindi
\[W = R i^2 t = i V t\]
Tale lavoro va ad aumentare l'energia interna del conduttore tramite gli urti con gli ioni del reticolo. Nel caso di un conduttore isolato, impossibilitato dunque a dissipare tale energia, si può arrivare alla fusione! Se si ha invece contatto termico con l'ambiente la temperatura aumenta fino all'equilibrio (se il valore finale è inferiore a quella di fusione), ovvero l'energia è convertita in quella degli ioni che si comportano come oscillatori, ed il lavoro in eccesso è quindi ceduto come calore.

\section{Resistori in serie e in parallelo}
\begin{description}
\item[Resistore] conduttore ohmico con resistenza caratteristica ben definita (a temperatura ambiente)
\end{description}
\infobox{E i fili?}{
I fili sono di consueto trascurati nel computo delle resistenze. Basti pensare che 1cm di rame di sezione 1mm${}^2$ ha R = 1,6 $\times 10^{-4} \, \Omega$!
}

Un sistema di resistori si dice in serie se i loro capi sono collegati in sequenza a due a due. In stato stazionario, sono attraversati dalla stessa corrente.
\[\Delta V_j = i R_j \implies \Delta V_{tot} = \sum_j \Delta V_j = \sum_j i R_j = i \sum_j R_j = i R_{eq} \]
da cui
\[R_{eq} = \sum_j R_j\]

\lawboxtext{Resistenza equivalente per resistori in serie}{
La r. equivalente equivale alla somma delle singole rr.
}

Da cui per la potenza

\[P = (\sum_j R_j) i^2\]

Un sistema di resistore si dice in parallelo se i capi positivi e negativi di tutti sono rispettivamente collegati, ovvero è presente la medesima differenza di potenziale agli estremi di ognuno.

\[\Delta V = i_j R_j \implies i_{tot} = \sum_j \frac{\Delta V}{R_j} = \Delta V \sum_j \frac{1}{R_j} = \frac{\Delta V}{R_{eq}}\]
da cui
\[\frac{1}{R_{eq}} = \sum_j \frac{1}{R_j}\]

\lawboxtext{Resistenza equivalente per resistori in parallelo}{
Il reciproco della r. equivalente equivale alla somma dei reciproci delle singole rr.
}

Da cui per la potenza

\[P = \Delta V^2 \big(\sum_j \frac{1}{R_j}\big) = \frac{i^2}{\sum_j \frac{1}{R_j}}\]

\infobox{Applicazioni dei sistemi in serie ed in parallelo}{
Si osserva chiaramente che per resistori in serie $R_{eq}$ è maggiore di ogni $R_j$: dunque è possibile aumentare la resistenza complessiva. Viceversa per resistori in parallelo $R_{eq}$ è minore di ogni $R_j$, che permette di diminuire la r. complessiva.
}

\section{Generatori}
\begin{description}
\item[Generatore] dispositivo in grado di mantenere una ddp costante ai capi del circuito
\end{description}
ricordando quindi la definizione di \textbf{forza elettromotrice (f.e.m.)}:
\[\mathcal{E} = \oint_l \vec{E} \cdot \dd[•]{\vec{l}}\]
un generatore di corrente è quindi un g. di f.e.m.
\\~\\
Si osserva che un condensatore non può essere un generatore, in quanto collegato ad un circuito si scaricherebbe in un tempo finito durante il quale solo si avrebbe corrente non nulla.
\\~\\
In stato stazionario la potenza erogata dal generatore è
\[(R+r)i^2\]
ove $r$ indica la \textbf{resistenza interna}. Essa equivale alla potenza spesa per muovere la carica complessiva dei portatori lungo il circuito, ovvero $\mathcal{E} i$; uguagliando le due espressioni si ottiene la legge di Ohm.
\\Ma per corrente non nulla questa implica che la f.e.m. nel circuito non sia nulla, ovvero \textbf{che il campo $\vec{E}$ nel circuito non sia conservativo}. La non conservatività si deve al campo all'interno del generatore, che mantiene le cariche sui poli costanti nel tempo. Tale campo $\vec{E*}$ è di verso opposto a $\vec{E}_{el}$ e di modulo maggiore in quanto sposta i portatori tra i due poli di modo da mantenervi costante la carica accumulata. Si definisce \textbf{campo elettromotore}.
\\Applicando ora le proprietà dell'integrale di linea:
\[\oint_l \vec{E} \cdot \dd[•]{\vec{l}} = \int_A^B \vec{E}_{el} \cdot \dd[•]{\vec{l}} + \int_B^A (\vec{E*} + \vec{E}_{el}) \cdot \dd[•]{\vec{l}} = \int_A^B \vec{E}_{el} \cdot \dd[•]{\vec{l}} + \int_B^A \vec{E*} \cdot \dd[•]{\vec{l}} + \int_B^A \vec{E}_{el} \cdot \dd[•]{\vec{l}} =\]
\[= \oint_l \vec{E}_{el} \cdot \dd[•]{\vec{l}} + \int_B^A \vec{E*} \cdot \dd[•]{\vec{l}} = \int_B^A \vec{E*} \cdot \dd[•]{\vec{l}}\]
in quanto $\vec{E}_{el}$ conservativo. Il campo elettromotore è invece non conservativo in quanto \textbf{nullo} fuori dal generatore, da cui segue che la sua circuitazione lungo qualsiasi linea tra $A$ e $B$ esterna al g. è nulla, a differenza di quella all'interno che vale $\mathcal{E} \neq 0$.
\\Ora si ha per Ohm
\[V_A - V_B = \int_A^B \vec{E} \cdot \dd[•]{\vec{l}} = R i\]s
da cui
\[\mathcal{E} = (R + r)i = Ri + ri = V_A - V_B + ri \implies \mathcal{E} - ri = V_A - V_B\]

\infobox{Misura della fem}{
In presenza di corrente la relazione precedente implica
\[\mathcal{E} > V_A - V_B\]
ovvero che la ddp misurata ai capi del generatore sia minore della f.e.m. a causa della resistenza interna. Per misurare la f.e.m. effettiva si stacca quindi il conduttore esterno (con un interruttore) e si ottiene la condizione di equilibrio nel generatore. Si dice che il circuito è \textbf{aperto}.
\\Le cariche accumulate ai poli non si muovono più a causa della repulsione elettrostatica. In assenza di corrente vale quindi
\[\mathcal{E} = V_A - V_B\]
che costituisce la vera e propria \textbf{definizione operativa} della f.e.m. di un generatore.
}

\subsection{Rami e leggi di Kirchhoff}
Si derivano le seguenti leggi valide per circuiti \textbf{stazionari}.
\begin{description}
\item[Ramo] porzione di circuito attraversata da una corrente uniforme
\end{description}
Per quanto visto, fissato un verso arbitrario della corrente (il risultato è il medesimo) e calcolando il contributo dei generatori secondo
\begin{itemize}
\item $+\mathcal{E}_k$ se attraversato \textbf{dalla faccia - a +}
\item $-\mathcal{E}_k$ in caso contrario
\end{itemize}
si ha la 

\lawbox{Legge di Ohm generalizzata}{V_A - V_B + \sum_k \varepsilon_k = i \cdot \sum_j R_j}

\begin{description}
\item[Nodo] punto di un circuito in cui convergono \textbf{almeno tre} conduttori (fili)
\end{description}

Per la continuità della corrente segue la

\lawboxtext{I Legge di Kirchhoff - legge dei nodi}{
La somma algebrica delle correnti confluenti in un nodo (fissata una convenzione di segno) è nulla, ovvero
\[\sum\limits_A \pm i_j = \sum\limits_{entranti} i_j - \sum\limits_{uscenti} i_k = 0\]
}

\begin{description}
\item[Maglia] Insieme di rami chiusi su se stessi, o equivalentemente cammino chiuso nella rete
\end{description}

Applicando le leggi ottenute

\lawboxtext{II Legge di Kirchhoff - legge delle maglie}{
Fissato un verso di percorrenza e determinati di conseguenza i segni dei vari contributi, la somma algebrica delle variazioni di tensione lungo una maglia è nulla, ovvero
\[\sum_k \varepsilon_k = \sum i_j R_j\]
}

\section{Circuito RC: carica di un condensatore}
Si intende studiare l'andamento delle grandezze elettriche in un circuito di singola maglia con un resistore, un condensatore ed un generatore, che viene chiuso al tempo $t = 0$ dando avvio al processo di progressiva carica del condensatore.
\\Si adotta l'approssimazione del \textbf{regime quasi-stazionario}, ovvero in cui la corrente varia nel tempo ma istantaneamente è uniforme nel circuito.
\\~\\
Il processo di carica dura finché il condensatore è pienamente carico; allora non circola più alcuna corrente. Per ricavare le relazioni differenziali e quindi integrali, si considera che ad ogni istante valgono, sotto l'assunto di quasistazionarietà:
\[C = \frac{q(t)}{\Delta V_C} \qquad \Delta V_{R} = iR \qquad \mathcal{E} = \Delta V_C + \Delta V_R\]
oltre che chiaramente per definizione
\[i(t) = \dv[•]{q}{t}(t)\]
sostituendo e risolvendo l'equazione differenziale ordinaria di I grado risultante per separazione delle variabili:
\[\frac{q(t)}{C} + R \dv[•]{q}{t}(t) = \mathcal{E} \implies - \frac{\dd[•]{t}}{RC} = \frac{\dd[•]{q}}{q - \mathcal{E}C} \implies - \frac{1}{RC} \int_0^t \dd[•]{t'} = \int_{q(0) = 0}^{q(t)} \frac{\dd[•]{q}}{q - \mathcal{E}C} \implies\]
\[\implies - \frac{t}{RC} = \ln \bigg( \frac{q(t) - \mathcal{E}C}{-\mathcal{E}C} \bigg)\]
da cui
\[q(t) = \mathcal{E}C \big(1 - e^{-t/RC}\big)\]
Seguono quindi le espressioni per le altre grandezze
\[i(t) = \frac{\mathcal{E}}{R} e^{-t/RC} \qquad \Delta V_C = \mathcal{E} \big(1 - e^{-t/RC}\big) \qquad \Delta V_R = \mathcal{E}e^{-t/RC}\]
Si osserva che
\begin{itemize}
\item $q(t)$ tende \textbf{asintoticamente} alla piena carica, ovvero al valore nominale $\mathcal{E}C$:
\[\lim\limits_{t \rightarrow +\infty} q(t) = \mathcal{E}C\]
\item $i(t)$ tende invece asintoticamente ad annullarsi
\end{itemize}
Anche se i portatori sono discreti, ciò si traduce nel fatto che fisicamente non si avrà mai il depositarsi dell'ultimo e dunque il raggiungimento del pieno equilibrio in tempi finiti!
\\Tuttavia è possibile determinare il tempo necessario per raggiungere una data percentuale della carica completa:
\[\frac{q(t)}{\mathcal{E}C} = 1 - e^{-t/RC} = f \implies t_f = RC \ln \frac{1}{1-f}\]
In particolare si definisce il \textbf{tempo caratteristico} $\tau \equiv RC$, che corrisponde al tempo necessario perché si depositi una frazione $\ds \frac{e-1}{e}$ della carica completa.

\subsection{Potenza ed energia}
Per la potenza erogata dal generatore si ha
\[P_{gen} = \mathcal{E} i(t) = \frac{\mathcal{E}}{R} e^{-t/RC} \]
mentre per quella dissipata sulla resistenza
\[P_R = \frac{\mathcal{E}^2}{R} e^{-2t/RC}\]
Si osservi il fattore 2 nell'esponenziale, che comporta $P_{gen} \big/ P_{diss} = e^{t/RC} > 0$ per $t > 0$. Dove finisce la potenza in eccesso? Risposta: nella carica del condensatore, ovvero nell'energia elettrostatica accumulata nel campo al suo interno.
\\~\\
Assumendo che il processo vada avanti fino ad un tempo infinito, integrando per trovare il lavoro complessivo:
\[L_{gen} = \frac{\mathcal{E}^2}{R} \int_0^\infty e^{-t/RC} \dd[•]{t} = \mathcal{E}^2 C \int_0^\infty e^{-u} \dd[•]{u} = \mathcal{E}^2 C\]
\[L_{diss} = \frac{\mathcal{E}^2}{R} \int_0^\infty e^{-2t/RC} \dd[•]{t} = \frac{1}{2} \mathcal{E}^2 C \int_0^\infty e^{-u} \dd[•]{u} = \frac{1}{2} \mathcal{E}^2 C\]
da cui
\[L_{gen} - L_{diss} = \frac{1}{2} \mathcal{E}^2 C\]
ma l'energia immagazzinata nel condensatore carico è 
\[U_C = \frac{1}{2} \mathcal{E}^2 C\]
Si osserva che le considerazioni fatte sono indipendenti dai valori specifici delle caratteristiche del circuito: \textbf{metà del lavoro erogato dal generatore è sempre spesa per caricare il condensatore e l'altra metà è dissipata sulla resistenza}.



