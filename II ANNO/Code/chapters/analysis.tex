\section{Gradiente}

Sia dato un \textbf{campo scalare}, ovvero una funzione

\[\varphi \, : \, \mathbb{R}^3 \rightarrow \mathbb{R} \qquad \varphi \, : \, (x,y,z) \mapsto \varphi(x,y,z)\]

sotto opportuni assunti di regolarità, è possibile definirne il \textbf{gradiente} in un punto secondo

\[\vec{\mathrm{grad}} \varphi = \big(\pdv[•]{\varphi}{x}, \pdv[•]{\varphi}{y}, \pdv[•]{\varphi}{z}\big) = \pdv[•]{\varphi}{x} \hat{i} + \pdv[•]{\varphi}{y} \hat{j} + \pdv[•]{\varphi}{z} \hat{k}\]

il gradiente della funzione definisce un \textbf{campo vettoriale}

\[\vec{\mathrm{grad}} \varphi \, : \, \mathbb{R}^3 \rightarrow \mathbb{R}^3\]

introducendo ora l'\textbf{operatore Nabla} $\ds \vec{\nabla} \equiv \big(\pdv[•]{}{x}, \pdv[•]{}{y}, \pdv[•]{}{z}\big) = \pdv[•]{}{x} \hat{i} + \pdv[•]{}{y} \hat{j} + \pdv[•]{}{z} \hat{k}$ si può riformulare

\[\vec{\mathrm{grad}} \varphi = \vec{\nabla} \varphi\]

si osserva che l'operatore è \textbf{covariante} per cambiamento di sistema di riferimento (segue dalla proprietà delle singole componenti): è un tensore covariante di rango 1. Ad esempio per una rotazione di angolo $\theta$ in senso antiorario sul piano xy

\[\begin{cases}
\ds \pdv[•]{•}{x} = \pdv[•]{•}{x\prime} \cos \theta - \pdv[•]{•}{y\prime} \sin \theta \\
\\
\ds \pdv[•]{•}{y} = \pdv[•]{•}{x\prime} \cos \theta - \pdv[•]{•}{y\prime} \cos \theta
\end{cases}\]

Si può quindi esprimere il differenziale di $\varphi$ come

\[\dd[•]{\varphi} = \vec{\nabla} \varphi \cdot \dd[•]{\vec{l}} = \big(\pdv[•]{\varphi}{x}, \pdv[•]{\varphi}{y}, \pdv[•]{\varphi}{z}\big) \cdot (\dd[•]{x}, \dd[•]{y}, \dd[•]{z}) = \pdv[•]{\varphi}{x} \dd[•]{x} + \pdv[•]{\varphi}{y} \dd[•]{y} + \pdv[•]{\varphi}{z} \dd[•]{z}\]

Può essere anche riformulato utilizzando la notazione intrinseca per lo spostamento lungo una curva

\[\dd[•]{\vec{l}} = \dd[•]{l} \hat{u_\tau} \implies \dd[•]{\varphi} =  \pdv[•]{\varphi}{x} \hat{i} \cdot \hat{u_\tau} \dd[•]{l} + \pdv[•]{\varphi}{y} \hat{j} \cdot \hat{u_\tau} \dd[•]{l} + \pdv[•]{\varphi}{z} \hat{k} \cdot \hat{u_\tau} \dd[•]{l}\]

Si ha quindi il seguente teorema, che estende il t. fond. del calcolo integrale a dimensioni superiori:

\lawboxtext{Teorema del gradiente}{
Sia $\varphi$ un campo scalare e $\vec{\nabla} \varphi$ il suo gradiente. Allora l'integrale del gradiente tra due punti $A$ e $B$ lungo una qualsiasi curva orientabile è pari alla differenza del valore di $\varphi$ nei due punti (e viceversa)

\[\varphi (B) - \varphi (A) = \int_A^B \vec{\nabla} \varphi \cdot \dd[•]{\vec{l}}\]

}

L'indipendenza del valore dell'integrale curvilineo tra due punti dalla traiettoria \textbf{non vale} in generale per tutti i campi vettoriali. Se vale, un campo si dice \textbf{conservativo} ed allora esiste una funzione potenziale di cui è il gradiente.

\section{Flusso}

Si consideri un campo vettoriale

\[\vec{f} \, : \, \mathbb{R}^3 \rightarrow \mathbb{R}^3\]

ed una superficie infinitesima di modulo $\dd[•]{S}$ e orientata con versore $\hat{n}$ ortogonale. Si nota che ocalmente qualsiasi superficie sufficientemente regolare è approssimabile con un piano e dunque la direzione è univoca, mentre il verso può essere fissato arbitrariamente.
\\Si definisce \textbf{flusso infinitesimo} del campo $\vec{f}$ attraverso $\dd[•]{S}$ con orientamento $\hat{n}$ lo scalare

\[\dd[•]{\Phi} = \vec{f} \cdot \dd[•]{S} \hat{n} = f \dd[•]{S} \cos \alpha\]

ove $\alpha$ è l'angolo compreso tra $\vec{f}$ in corrispondenza della superficie (che si può assumere localmente uniforme) e il versore normale.
\\Considerando una superficie $S$ orientata, ovvero con fissato un campo continuo di versori normali, si definisce il flusso di $\vec{f}$ attraverso di essa secondo

\[\Phi_S(\vec{f}) = \iint\limits_{S} \vec{f} \cdot \hat{n}\dd[•]{S} = \iint\limits_{S} f \dd[•]{S} \cos \alpha\]

\infobox{Intuito fisico}{
Considerando ad esempio il campo di velocità dell'aria in presenza di vento, assumendo per semplicità sia omogeneo in direzione e verso, si ha che 
\begin{itemize}
\item per una finestra posta parallelamente ad esso (ovvero con versore normale \textit{ortogonale} al campo sulla sua superficie) il flusso è nullo: difatti non vi è aria che la attraversa

\item per una finestra posta ortogonalmente al vento (ovvero con versore normale \textit{parallelo} o antiparallelo al campo sulla superficie) il flusso è diverso da $0$ e in particolare massimo o minimo (considerato il segno, altrimenti sempre massimo in modulo) in quanto essendo l'orientamento relativo costante, derivando l'espressione di $\Phi$ rispetto ad $\alpha$ si ottiene l'annullamento per $\alpha = 0, \pi$. Difatti si ha la maggiore quantità possibile di aria che attraversa la finestra nell'unità di tempo.
\end{itemize}
}

\section{Divergenza}

Si consideri una superficie chiusa (dunque priva di bordo) $S$ orientata con \textbf{normale esterna} $\hat{n}$ ed il flusso di un c.v. $\vec{f}$ attraverso di essa. Sia $V$ il volume racchiuso dalla superficie.
\\La si suddivida ora in due sottosuperfici aperte $S_1$, $S_2$ orientate con $\hat{n_1}$, $\hat{n_2}$ uguali ad $\hat{n}$; si unisca ad entrambi un diaframma $D$ che permetta di ottenere due superfici chiuse $S_1' = S_1 \cup D$, $S_2' = S_2 \cup D$. Si estenda su ciascuna $S_i'$ l'orientamento $\hat{n_i}$ \textbf{verso l'esterno} - dunque ottenendo $\hat{n_1}$, $\hat{n_2}$ opposti sul diaframma. 
\\Segue che

\[\Phi_{D_1}(\vec{f}) = \iint\limits_{D} \vec{f} \cdot \hat{n_1}\dd[•]{S} = - \iint\limits_{D} \vec{f} \cdot \hat{n_2}\dd[•]{S} = - \Phi_{D_2}(\vec{f}) \implies \Phi_{D_1}(\vec{f}) + \Phi_{D_2}(\vec{f}) = 0\]

Suddividendo il flusso attraverso la superficie $S$ in quello attraverso le due sottosuperfici aperte:

\[\Phi_S(\vec{f}) = \oiint\limits_{S} \vec{f} \cdot \hat{n}\dd[•]{S} = \iint\limits_{S_1} \vec{f} \cdot \hat{n_1}\dd[•]{S_1} + \iint\limits_{S_2} \vec{f} \cdot \hat{n_2}\dd[•]{S_2} = \oiint\limits_{S} \vec{f} \cdot \hat{n}\dd[•]{S} = \]
\[=\iint\limits_{S_1} \vec{f} \cdot \hat{n_1}\dd[•]{S_1} + \iint\limits_{S_2} \vec{f} \cdot \hat{n_2}\dd[•]{S_2} + \underbrace{\big(\iint\limits_{D} \vec{f} \cdot \hat{n_1}\dd[•]{S} + \iint\limits_{D} \vec{f} \cdot \hat{n_2}\dd[•]{S}\big)}_{= 0} = \]

\[ = \oiint\limits_{S_1'} \vec{f} \cdot \hat{n_1}\dd[•]{S_1'} + \iint\limits_{S_2'} \vec{f} \cdot \hat{n_2}\dd[•]{S_2'}\]

Iterando il procedimento, si suddividono a loro volta le sottosuperfici chiuse ottenute in ulteriori sempre più piccole, partizionando il volume racchiuso da $S$. Per quanto visto, il flusso netto attraverso ogni parete comune all'interno è nullo: dunque $\Phi_S(\vec{f})$ dipenderà solamente da quello attraverso la superficie esterna originaria $S$. Dunque, normalizzando sui volumetti:

\[\Phi_S(\vec{f}) = \sum\limits_{i=1}^{N} \oiint\limits_{S_i} \vec{f} \cdot \hat{n_i}\dd[•]{S_i} = \sum\limits_{i=1}^{N} \frac{ \ds\oiint_{S_i} \vec{f} \cdot \hat{n_i}\dd[•]{S_i}}{V_i} V_i\]

definendo ora la \textbf{divergenza di $\vec{f}$} come

\[\mathrm{div} \vec{f} \equiv \lim\limits_{V_i \rightarrow 0} \frac{\ds \oiint_{S_i} \vec{f} \cdot \hat{n_i}\dd[•]{S_i}}{V_i}\]

si osserva che questa descrive \textbf{la quantità di linee di forza che entrano o escono (nettamente) da ogni superficie chiusa}, normalizzata sul volume da essa delimitato. Si hanno quindi localmente:

\begin{itemize}

\item Sorgenti del campo (punti da cui diverge) se $\mathrm{div} \vec{f} > 0$

\item Pozzi del campo (punti in cui converge) se $\mathrm{div} \vec{f} < 0$

\item Punti a divergenza nulla se non vi sono né pozzi né sorgenti, dunque nessuna convergenza o divergenza delle linee di campo

\end{itemize}

Se $\vec{f}$ ha divergenza nulla ovunque si dice \textbf{solenoidale}.
\\Applicando ora il limite all'espressione per il flusso il membro di destra diviene un integrale di volume e si ha il

\lawbox{Teorema della divergenza (o di Gauss)}{\oiint\limits_{S} \vec{f} \cdot \hat{n}\dd[•]{S} = \iiint\limits_{V(S)} (\mathrm{div} \vec{f}) \dd[]{V}}

In coordinate cartesiane, vale

\[\mathrm{div} \vec{f} = \vec{\nabla} \cdot \vec{f} = \pdv[•]{f_x}{x} + \pdv[•]{f_y}{y} + \pdv[•]{f_z}{z}\]

come si dimostra di seguito.

\subsection{Divergenza in coordinate cartesiane}
Si consideri un volume infinitesimo a forma di parallelepipedo di lati $\Delta x$, $\Delta y$, $\Delta z$. Per il flusso di un generico campo $\vec{f}$ attraverso di esso si ha

\[\Delta \Phi_{tot}(\vec{f}) = \Delta \Phi_{davanti}(\vec{f}) + \Delta \Phi_{dietro}(\vec{f}) + \Delta \Phi_{destra}(\vec{f}) + \Delta \Phi_{sinistra}(\vec{f}) + \Delta \Phi_{sopra}(\vec{f}) + \Delta \Phi_{sotto}(\vec{f})\]

ove l'orientamento esterno delle facce è

\[\hat{n}_{davanti} = \hat{i} \quad \hat{n}_{dietro} = -\hat{i} \quad \hat{n}_{destra} = \hat{j} \quad \hat{n}_{sinistra} = -\hat{j} \quad \hat{n}_{sopra} = \hat{k} \quad \hat{n}_{sotto} = -\hat{k}\]

considerando solo le facce $sopra$ e $sotto$ si ha

\[\Delta \Phi_{sopra}(\vec{f}) = \vec{f} \cdot \hat{k} (\Delta x \Delta y) = f_z(x,y,z + \Delta z) \Delta x \Delta y\]

\[\Delta \Phi_{sotto}(\vec{f}) = \vec{f} \cdot (- \hat{k}) (\Delta x \Delta y) = - f_z(x,y,z) \Delta x \Delta y\]

\[\Delta \Phi_{sopra}(\vec{f}) + \Delta \Phi_{sotto}(\vec{f}) = [f_z(x,y,z + \Delta z) - f_z(x,y,z)] (\Delta x \Delta y)\]

Ripetendo analogamente per le altre coppie di facce con versori antiparalleli:

\[\Delta \Phi_{davanti}(\vec{f}) + \Delta \Phi_{dietro}(\vec{f}) = [f_x(x + \Delta x,y,z) - f_z(x,y,z)] (\Delta y \Delta z)\]
\[\Delta \Phi_{destra}(\vec{f}) + \Delta \Phi_{sinistra}(\vec{f}) = [f_y(x,y + \Delta y,z) - f_z(x,y,z)] (\Delta x \Delta z)\]

da cui

\[\Delta \Phi_{tot}(\vec{f}) = [f_x(x + \Delta x,y,z) - f_z(x,y,z)] (\Delta y \Delta z) + [f_y(x,y + \Delta y,z) - f_z(x,y,z)] (\Delta x \Delta z) + \]
\[ + [f_z(x,y,z + \Delta z) - f_z(x,y,z)] (\Delta x \Delta y) \approx \]
\[\approx \big[\pdv[•]{f_x}{x} \Delta x\big](\Delta y \Delta z) + \big[\pdv[•]{f_y}{y} \Delta y\big](\Delta x \Delta z) + \big[\pdv[•]{f_z}{z} \Delta z\big](\Delta x \Delta y) = \]
\[= \big[\pdv[•]{f_x}{x} + \pdv[•]{f_y}{y} + \pdv[•]{f_z}{z}\big](\Delta x \Delta y \Delta z) = \big[\pdv[•]{f_x}{x} + \pdv[•]{f_y}{y} + \pdv[•]{f_z}{z}\big] \Delta V\]

ma per il teorema della divergenza si ha $\Delta \Phi = \mathrm{div} \vec{f} \Delta V$ e dunque

\[\mathrm{div} \vec{f} = \pdv[•]{f_x}{x} + \pdv[•]{f_y}{y} + \pdv[•]{f_z}{z}\]

\section{Circuitazione}

Si consideri una linea chiusa $L$ su cui è fissato un orientamento $\hat{\tau}$. Si definisce il lavoro infinitesimo di un c.v. $\vec{f}$ lungo un elemento di linea $\dd[•]{\vec{l}}$ come

\[\dd[•]{\Gamma} = \vec{f} \cdot \dd[•]{\vec{l}}\]

e quindi la \textbf{circuitazione} lungo la linea come l'integrale su tutta $L$

\[\Gamma = \oint_L \vec{f} \cdot \dd[•]{\vec{l}}\]+

Sia $L$ orientata in senso antiorario. 
\\Si determinino due punti $A$ e $B$ su $L$, che la suddividono in due linee aperte $L_1$ (orientata da $A$ ad $B$) e $L_2$ (orientata da $B$ a $A$). Si congiungano ora i due punti con una terza linea $G$ e si prolunghi l'orientamento originario su $l_1 \cup g \equiv l_1*$ e $l_1 \cup g \equiv l_2*$ di modo che sia continuo su ognuna: $g$ sarà di conseguenza percorsa in senso opposto nel calcolo della circuitazione sulle due curve ottenute. Segue che

\[\oint_L \vec{f} \cdot \dd[•]{\vec{l}} = \int_{\hspace{-0.45cm} l_1 \hspace{0.22cm} A}^B \vec{f} \cdot \dd[•]{\vec{l_1}} + \int_{\hspace{-0.45cm} l_2 \hspace{0.22cm} B}^A \vec{f} \cdot \dd[•]{l_2} + \underbrace{\big(\int_{\hspace{-0.45cm} g \hspace{0.22cm} B}^A \vec{f} \cdot \dd[•]{\vec{g_1}} + \int_{\hspace{-0.45cm} g \hspace{0.22cm} A}^B \vec{f} \cdot \dd[•]{\vec{g_2}}\big)}_{= 0} = \]
\[= \oint_{l_1*} \vec{f} \cdot \dd[•]{\vec{l_1*}} + \oint_{l_2*} \vec{f} \cdot \dd[•]{\vec{l_2*}}\]

dividendo ulteriormente le curve chiuse ottenute con un procedimento analogo, si ottengono 
percorsi chiusi più piccoli \textbf{orientati in senso antiorario} i cui tratti comuni danno complessivamente \textbf{un contributo nullo} alla circuitazione. Si definiscano quindi \textbf{superfici orientate} con ciascuno di essi come bordo, orientate secondo la regola della mano destra.
\\Esprimendo la circuitazione come somma dei contributi e normalizzando sull'area delle superfici

\[\Gamma_L(\vec{f}) = \sum\limits_{i=1}^{N} \frac{\ds \oint_{l_1*} \vec{f} \cdot \dd[•]{\vec{l_i}}}{S_i} S_i\]

definendo ora al limite $S_i \rightarrow 0$ la circuitazione normalizzata come proiezione su $S_i$ di una grandezza definita \textbf{rotore di $\vec{f}$} si ha

\[(\mathrm{rot} \vec{f}) \cdot \hat{n_i} \equiv \lim\limits_{S_i \rightarrow 0} \frac{\ds \oint_{l_1*} \vec{f} \cdot \dd[•]{\vec{l_i}}}{S_i}\]

e dunque passando al limite nell'espressione precedente, ove la sommatoria diviene un integrale di superficie si ha il

\lawbox{Teorema del rotore (o di Stokes)}{\oint\limits_L \vec{f} \cdot \dd[•]{\vec{l}} = \iint\limits_{S_L} (\mathrm{rot} \vec{f}) \cdot \dd[•]{\vec{S}}}

Si osserva che non si è in alcun modo assunto che le linee costruite giacciano sul piano della curva chiusa originaria (e neppure che questa sia piana per esattezza), dunque \textbf{il risultato è indipendente dalla superficie considerata}, a patto che il suo bordo sia $\Gamma$.
\\~\\In coordinate cartesiane vale

\[\mathrm{rot} \vec{f} = \vec{\nabla} \wedge \vec{f} = \big(\pdv[•]{f_z}{y} - \pdv[•]{f_y}{z}, \pdv[•]{f_x}{z} - \pdv[•]{f_z}{x}, \pdv[•]{f_y}{x} - \pdv[•]{f_x}{y}\big)\]

come si dimostra di seguito.

\subsection{Rotore in coordinate cartesiane}

Si considera un circuito rettangolare $L$ sul piano $xy$, orientato in senso antiorario, e la superficie piana $S$ da esso delimitata, orientata secondo la regola della mano destra con $\hat{n} = \hat{k}$. Siano $A$, $B$, $C$, $D$ i vertici del rettangolo, $(x_0, y_0) \equiv A$ il suo vertice più prossimo all'origine e $\Delta x$, $\Delta y$ la lunghezza dei lati. Calcolando la circuitazione del c.v. $\vec{f}$ su $L$

\[\oint_L \vec{f} \cdot \dd[•]{\vec{l}} = \int_A^B \vec{f} \cdot \hat{i} \dd[•]{x} + \int_B^C \vec{f} \cdot \hat{j} \dd[•]{y} + \int_C^D \vec{f} \cdot (-\hat{i}) \dd[•]{y} + \int_D^A \vec{f} \cdot (-\hat{j}) \dd[•]{y} = \]
\[ = \int_A^B f_x(x, y=y_0, 0) \dd[•]{x} + \int_B^C f_y(x = x_0 + \Delta x, y, 0) \dd[•]{y} - \int_C^D f_x(x, y = y_0 + \Delta y, 0) \dd[]{x} - \int_D^A f_y(x = x_0, y, 0) \dd[•]{y}\]

Applicando ora il teorema della media integrale ($\vec{f}$ è supposta possedere la regolarità necessaria), e.g. per il primo termine

\[\exists \, \overline{x} \in \quad \big] \, x_0, x_0 + \Delta x \, \big[ \quad : \quad \int_A^B f_x(x, y=y_0, 0) \dd[•]{x} = f_x(\overline{x}, y_0, 0) \mu([x_0,x_0 + \Delta x]) = f_x(\overline{x}, y_0, 0) \Delta x\]

per intervalli sufficientemente ridotti si può assumere

\[f_x(\overline{x}, y_0, 0) \approx f_x(x_0, y_0, 0)\]

Applicando lo stesso procedimento agli altri contributi si ottiene

\[\oint_L \vec{f} \cdot \dd[•]{\vec{l}} \approx f_x(x_0, y_0, 0) \Delta x + f_y(x_0 + \Delta x, y_0) \Delta y - f_x(x_0, y_0 + \Delta y) \Delta x - f_y(x_0, y_0) \Delta y = \]
\[ = - \Delta x \big[f_x(x_0,y_0 + \Delta y,0) - f_x(x_0, y_0, 0)\big] + \Delta y \big[f_y(x_0 + \Delta x, y_0, 0) - f_y(x_0, y_0, 0)\big] \approx - \big[\pdv[•]{f_x}{y} \Delta y\big] \Delta x + \big[\pdv[•]{f_y}{x} \Delta x\big] \Delta y = \]
\[= \big(\pdv[•]{f_y}{x} - \pdv[•]{f_x}{y}\big) \underbrace{\Delta x \Delta y}_{0}\]

applicando il teorema si ha

\[\oint_L \vec{f} \cdot \dd[•]{\vec{l}} = \iint\limits_{S} (rot \vec{f}) \cdot \dd[•]{\vec{S}} = \iint\limits_{S} (\mathrm{rot} \vec{f}) \cdot \hat{k}\dd[•]{S} \approx (\mathrm{rot} \vec{f}) \cdot \hat{k} S \implies\]

\[\implies (\mathrm{rot} \vec{f}) \cdot \hat{k} = \pdv[•]{f_y}{x} - \pdv[•]{f_x}{y}\]

Ripetendo analogamente per superfici rettangolari orientate ortogonalmente agli altri due assi si ottengono le altre componenti del rotore.

\subsection{Significato fisico}

Si considera ad esempio una vasca da bagno con lo scarico aperto, in cui il c.v. è quello della velocità dell'acqua. Collocando un bastone sulla superficie, il rotore del campo descriverà \textbf{il moto di rotazione indotto} nel bastone, ovvero \textit{quanto il c.v. ruota attorno al suo centro di massa}: $\mathrm{rot} \vec{v} \propto \vec{\omega}$
\\Fissato un SdR cartesiano con assi $x$ e $y$ sul piano della superficie, si ha ad esempio

\begin{itemize}

\item Se $\vec{f} = (y, 0, 0)$, il suo rotore è $(0,0,-1)$, dunque orientato nel verso negativo delle $z$. Si ha una rotazione del bastone in senso \textit{orario}

\item Se $\vec{f} = (y, -x, 0)$, il suo rotore è $(0,0,-2)$: si ha un effetto analogo ma maggiore in intensità

\item Se $\vec{f} = (y, x, 0)$, il suo rotore è $(0,0,0)$: non si ha alcuna rotazione in quanto il momento netto è nullo.

\end{itemize}

\subsection{Relazioni}

Considerando l'integrale del rotore su una superficie chiusa e applicandovi il th della divergenza

\[\oiint\limits_{S} (\mathrm{rot} \vec{f}) \cdot \dd[•]{\vec{S}} = \iiint\limits_{V(S)} \mathrm{div} (\mathrm{rot} \vec{f}) \dd[]{V} = \iiint\limits_{V(S)} \mathrm{div} (\mathrm{rot} \vec{f}) \dd[]{V} = 0\]

se la superficie è stata costruita incollando due superfici con bordo in comune ed invertendo l'orientamento di una affinché sia esterno su tutta $S$ si ha la dimostrazione dell'indipendenza dell'integrale di superficie nel teorema di Stokes dalla s. considerata, a patto che il bordo sia il medesimo. Se infatti $\vec{f}$ soddisfa le ipotesi del teorema di Schwarz 
\[\pdv[•]{f_x}{z}{y} = \pdv[•]{f_x}{y}{z} \qquad \pdv[•]{f_y}{x}{z} = \pdv[•]{f_y}{z}{x} \qquad \pdv[•]{f_z}{y}{x} = \pdv[•]{f_z}{x}{y}\]
e dunque

\[\vec{\nabla} \cdot (\vec{\nabla} \wedge \vec{f})  = \pdv[•]{f_z}{y}{x} - \pdv[•]{f_y}{z}{x} + \pdv[•]{f_x}{z}{y} - \pdv[•]{f_z}{x}{y} + \pdv[•]{f_y}{x}{z} - \pdv[•]{f_x}{y}{z} = 0\]

Si può comprendere il significato intuitivo della relazione considerando un volumetto cubico con lati lungo i tre assi cartesiani. Supponendo che la componente $z$ del rotore abbia divergenza non nulla, si ha che la 'rotazione' del campo attorno a tale asse, e dunque sul piano $xy$, aumenti con la quota su $z$. Ma ciò implica che sulle facce orientate parallelamente ai piani $xz$ e $yz$ si abbia uno squilibrio tra il lato superiore ed inferiore, e dunque un'ulteriore rotazione del campo su tali facce, ovvero un rotore sugli assi $x$ e $y$. Dato che la rotazione su facce opposte avviene \textit{in senso opposto}, si ha una divergenza non nulla anche per queste componenti del rotore. Complessivamente il loro contributo \textbf{elide quello del rotore su $z$} (sotto l'assunto di sufficiente regolarità del c.v., ovvero equivalentemente il soddisfacimento delle ipotesi di Schwarz).

\section{Conservatività}
Un c.v. è conservativo se esiste una funzione potenziale di cui è il gradiente o \textit{equivalentemente} il suo integrale lungo una qualsiasi linea chiusa è nullo.
\\Considerato un campo $\vec{f}$ e una linea chiusa $\Gamma$, applicando Stokes:

\[\oint_\Gamma \vec{f} \cdot \dd[•]{\vec{l}} = \iint\limits_{S(\Gamma)} (\vec{\nabla} \wedge \vec{f}) \cdot \dd[•]{\vec{S}}\]

dunque il campo è conservativo se e solo se l'integrale del rotore si annulla \textit{per qualsiasi superficie} con medesimo bordo $\Gamma$. Valendo per ogni curva, ciò si traduce in una condizione \textit{locale}: il campo è conservativo se e solo se il suo rotore è nullo, ovvero il campo è \textbf{irrotazionale}:

\[\vec{\nabla} \wedge \vec{f} = \vec{0}\]

\section{Identità e teorema di Green}

Si derivano due identità di grande utilità in analisi vettoriale. Considerando un campo vettoriale generico $\vec{A}$ e un volume $V$ delimitato da una superficie $S$ orientata con normale esterna si ha per il teorema della divergenza

\[\iiint\limits_V (\vec{\nabla} \cdot \vec{A}) \dd[]{V} = \oiint\limits_{S(V)} \vec{A} \cdot \dd[•]{\vec{S}}\]

se ora $\vec{A}$ può essere espresso come prodotto di un campo scalare per il gradiente di un secondo campo scalare, ovvero $\ds \vec{A} = \varphi \vec{\nabla} \psi$ si ha

\[\vec{\nabla} \cdot (\varphi \vec{\nabla} \psi) = \varphi \laplacian \psi + \vec{\nabla} \varphi \cdot \vec{\nabla} \psi\]

\[\varphi \vec{\nabla} \psi \cdot \hat{n} = \varphi \pdv[•]{\psi}{\hat{n}}\]

da cui segue la 

\lawbox{Prima identità di Green}{\iiint\limits_V (\varphi \laplacian \psi + \vec{\nabla} \varphi \cdot \vec{\nabla} \psi) \dd[]{V} = \oiint\limits_{S(V)} \varphi \pdv[•]{\psi}{\hat{n}} \dd[•]{S}}

Definendo un altro campo $\vec{B} = \psi \vec{\nabla} \varphi$ e operando il medesimo procedimento si ottiene la prima identità con i campi scambiati

\[\iiint\limits_V (\psi \laplacian \varphi + \vec{\nabla} \psi \cdot \vec{\nabla} \varphi) \dd[]{V} = \oiint\limits_{S(V)} \psi \pdv[•]{\varphi}{\hat{n}} \dd[•]{S}\]

sottraendo ora la seconda equazione alla prima, considerando la simmetria del prodotto scalare, si ottiene la 

\lawbox{Seconda identità di Green (Teorema di Green)}{
\iiint\limits_V (\varphi \laplacian \psi - \psi \laplacian \varphi) \dd[]{V} = \oiint\limits_{S(V)} \big[\varphi \pdv[•]{\psi}{\hat{n}} - \psi \pdv[•]{\varphi}{\hat{n}} \big]\dd[•]{S}
}

\section{Delta di Dirac}

Per descrivere e.g. distribuzioni discrete di carica tramite funzioni si introduce un utile strumento matematico.
\\Nel caso unidimensionale, si definisce la \textbf{delta di Dirac} secondo

\[\delta (x) = \begin{cases}
0 & se \quad x \neq 0\\
\infty & se \quad x = 0
\end{cases} \qquad \qquad \int_{-\infty}^{+\infty} \delta(x) \dd[•]{x} = 1\]

non è propriamente una funzione, quanto una \textbf{distribuzione} (anche se ha un picco infinito in un punto il suo integrale è finito e normalizzato) o funzione generalizzata, e può essere costruita come limite di una successione di funzioni. 
\\Considerando ora una generica funzione continua $f$ si ha

\[\int_{-\infty}^{+\infty} f(x) \delta(x) \dd[]{x} = f(0)\]

(il risultato è analogo anche restringendo l'intervallo di integrazione). Chiaramente per spostare il picco della delta è sufficiente operare il cambio di variabile $x = x' - a$

\[\delta (x-a) = \begin{cases}
0 & se \quad x \neq a\\
\infty & se \quad x = a
\end{cases} \qquad \qquad \int_{-\infty}^{+\infty} \delta(x-a) \dd[•]{x} = 1\]

e quindi

\[\int_{-\infty}^{+\infty} f(x) \delta(x-a) \dd[]{x} = f(a)\]

Se si restringono gli intervalli di integrazione in tutti i casi visti si avrà $0$ se il punto $x=0$ o $x=a$ non vi è incluso, in quanto su di essi la delta si riduce alla funzione identicamente nulla.
\\~\\Si può facilmente generalizzare la delta al caso pluridimensionale, ad esempio allo spazio 3-dim:

\[\delta^3(x,y,z) = \delta(x) \delta(y) \delta(z) = \begin{cases}
0 & se \quad (x,y,z) \neq (0,0,0)\\
\infty & se \quad (x,y,z) = (0,0,0)
\end{cases} \qquad \qquad \int\limits_{-\infty}^{+\infty}\int\limits_{-\infty}^{+\infty}\int\limits_{-\infty}^{+\infty} \delta^3(x,y,z) \dd[•]{x} \dd[•]{y} \dd[•]{z} = 1\]

e per un generico punto $\vec{a} = (a, b, c)$

\[\delta^3(x-a,y-b,z-c) = \delta^3(\vec{r} - \vec{a}) = \delta(x-a) \delta(y-b) \delta(z-c) = \begin{cases}
0 & se \quad (x,y,z) \neq (a,b,c)\\
\infty & se \quad (x,y,z) = (a,b,c)
\end{cases}\]
\[\int\limits_{-\infty}^{+\infty}\int\limits_{-\infty}^{+\infty}\int\limits_{-\infty}^{+\infty} \delta^3(\vec{r} - \vec{a}) \dd[•]{x} \dd[•]{y} \dd[•]{z} = 1\]

Con analoghe considerazioni in caso di restrizione del dominio di integrazione. Per un campo scalare $f$ si ha

\[\int\limits_{-\infty}^{+\infty}\int\limits_{-\infty}^{+\infty}\int\limits_{-\infty}^{+\infty} f(x,y,z) \delta^3(x,y,z) \dd[•]{x} \dd[•]{y} \dd[•]{z} = f(0,0,0)\]

\[\int\limits_{-\infty}^{+\infty}\int\limits_{-\infty}^{+\infty}\int\limits_{-\infty}^{+\infty} \delta^3(\vec{r} - \vec{a}) \dd[•]{x} \dd[•]{y} \dd[•]{z} = f(a,b,c)\]

\subsection{La divergenza del campo di una carica puntiforme}

Considerando l'espressione del campo $\vec{E}$ generato da una carica puntiforme e applicandovi l'operatore divergenza:

\[\vec{\nabla} \cdot \vec{E} = \frac{q}{4 \pi \varepsilon_0} \vec{\nabla} \cdot \big(\frac{\vec{r}}{r^3}\big)\]

si ottiene per $r \neq 0$, come dimostrato nella sezione \textbf{1.11} in coordinate cartesiane:

\[\vec{\nabla} \cdot \vec{E} = \pdv[•]{E_x}{x} + \pdv[•]{E_y}{y} + \pdv[•]{E_z}{z} = \frac{Q}{4 \pi \varepsilon_0 r^6} \big[r^3 - 3x^2 r + r^3 - 3y^2 r + r^3 - 3z^2 r\big] = \frac{3 Q}{4 \pi \varepsilon_0 r^3} (1-1) = 0\]

alternativamente utilizzando la divergenza in coordinate polari sferiche, con le derivate rispetto a $\theta$ e $\phi$ nulle in quanto il campo ha solo dipendenza radiale:

\[\vec{\nabla} \cdot \vec{E} = \frac{Q}{4 \pi \varepsilon_0}\frac{1}{r^2} \pdv[•]{•}{r} \big(r^2 \frac{1}{r^2}\big) = 0\]

sempre per $r \neq 0$. Tuttavia considerando una superficie sferica di raggio $R > 0$ centrata nella carica ed orientata con normale esterna il flusso attraverso di essa vale

\[\iint\limits_{S} \vec{E} \cdot \dd[•]{\vec{S}} = \frac{Q}{4 \pi \varepsilon_0 R^2} 4 \pi R^2 = \frac{Q}{\varepsilon_0} = 4 \pi Q k\]

e dunque per il teorema della divergenza chiaramente \textbf{$\vec{\nabla} \cdot \vec{E}$ non può annullarsi ovunque all'interno}.
\\La chiave risiede evidentemente nel punto $r = 0$, ovvero in corrispondenza della carica. Sia nell'espressione ottenuta per la divergenza in coordinate cartesiane che in polari si ha una potenza di $r$ al denominatore, e si è dunque dovuto assumere che il raggio fosse non nullo per ottenere il risultato di divergenza nulla. Per raggio tendente a $0$ si ha infatti una \textbf{divergenza divergente} a $\pm\infty$ (si perdoni il gioco lessicale) a seconda del segno della carica. 
\\Se si pensa la divergenza come limite del flusso attraverso una superficie sferica centrata nel punto normalizzato sul volume da questa racchiuso, si ha infatti che essendo il campo uniforme sulla sfera $\Phi \propto r^2$ e $V \propto r^3$, dunque $\vec{\nabla} \cdot \vec{E} \propto \frac{1}{r}$. Chiaramente se si calcola il limite in un punto in cui non è presente alcuna carica non si avrà tale andamento in quanto si avrà un raggio massimo non nullo per cui non è inclusa alcuna carica nella superficie e dunque il flusso è nullo.
\\~\\L'introduzione della delta di Dirac permette di risolvere il paradosso: infatti

\[\vec{\nabla} \cdot \big(\frac{\vec{r}}{r^3}\big) = 4 \pi \delta^3(\vec{r})\]

che si può generalizzare se l'origine non è posta nella carica a 

\[\vec{\nabla} \cdot \big(\frac{\vec{r} - \vec{a}}{\|\vec{r} - \vec{a}\|^3}\big) = 4 \pi \delta^3(\vec{r} - \vec{a})\]

Dunque per descrivere il campo di una carica in posizione $\vec{a}$

\[\rho(\vec{r}) = Q \delta^3(\vec{r} - \vec{a}) \implies \vec{\nabla} \cdot \vec{E} = \frac{Q}{\varepsilon_0} \delta^3(\vec{r} - \vec{a})\]

e generalizzando ad una distribuzione discreta di $N$ cariche $Q_i$ in posizioni $\vec{a_i}$:

\[\rho(\vec{r}) = \sum\limits_{i=1}^{N} Q_i \delta^3(\vec{r} - \vec{a_i}) \implies \vec{\nabla} \cdot \vec{E} = \frac{1}{\varepsilon_0} \sum\limits_{i=1}^{N} Q_i \delta^3(\vec{r} - \vec{a_i})\]

Si osserva infine che

\[\pdv[•]{•}{x} \big(\frac{1}{r}\big) = \pdv[•]{•}{x} \big(\frac{1}{\sqrt{x^2 + y^2 + z^2}}\big) = - \frac{x}{(x^2 + y^2 + z^2)^{3/2}} = - \frac{1}{r^2} \big(\frac{x}{r}\big)\]
e analogamente

\[\pdv[•]{•}{y} \big(\frac{1}{r}\big) = - \frac{1}{r^2} \big(\frac{y}{r}\big) \qquad \qquad \pdv[•]{•}{z} \big(\frac{1}{r}\big) = - \frac{1}{r^2} \big(\frac{z}{r}\big)\]
da cui

\[\vec{\nabla} \big(\frac{1}{r}\big) = - \frac{\hat{r}}{r^2}\]

Applicando la divergenza si ottiene così l'equazione di Poisson per il potenziale elettrostatico in presenza di una distribuzione discreta di cariche:

\[\laplacian \big(\frac{1}{r}\big) = - 4 \pi \delta^3(\vec{r}) \implies \laplacian V = - \frac{1}{\varepsilon_0} \sum\limits_{i=1}^{N} Q_i \delta^3(\vec{r} - \vec{a_i})\]









