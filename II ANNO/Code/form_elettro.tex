\documentclass[10pt, oneside]{article}
\usepackage[utf8]{inputenc}
\usepackage{graphicx} % Required for inserting images
\usepackage{amsmath}
\usepackage{esint}
\usepackage{amssymb}
\usepackage[a4paper,left=2.1cm, right=2.1cm, top=2cm, bottom=2cm]{geometry}
\usepackage{verbatim}
\usepackage[english]{babel}
\usepackage{hyperref}
\usepackage{tikz}
\usetikzlibrary{decorations.pathreplacing,calligraphy}
\usetikzlibrary { decorations.pathmorphing, decorations.shapes, }
\usepackage[cal=dutchcal]{mathalfa}
\usepackage{wrapfig}
\usepackage{physics}
\usepackage{listings}
\usepackage{comment}
\renewcommand{\rmdefault}{cmss}

\lstdefinestyle{mystyle}{	basicstyle=\ttfamily\footnotesize
}

\lstset{style=mystyle}

\newcommand{\limit}[2]{\lim\limits_{#1 \rightarrow #2}}

\newcommand{\infobox}[2]{\vspace{0.5cm}~\\ \textbf{#1} \hrulefill \vspace{0.2cm}\\#2 {}\,\\\hrule \vspace{0.5cm}}
\newcommand{\lawbox}[2]{\begin{center}
\framebox{
\parbox{\linewidth}{
\vspace{0.3cm}
\textbf{#1} \hfill $\displaystyle #2$
\vspace{0.3cm}
}
}
\end{center}}

\newcommand{\lawboxtext}[2]{\begin{center}
\framebox{
\parbox{\linewidth}{
\vspace{0.3cm}
\textbf{#1} \vspace{0.1cm} \\#2
\vspace{0.3cm}
}
}
\end{center}}


\title{Zoccoli, Spighi, Rinaldi - Classical Electrodynamics}
\author{Pocket reference for 2st year course - BSc Physics, Unibo}
\date{2023}

\makeindex

\begin{document}

\maketitle

\section{Elettrostatica nel vuoto}
\begin{description}

\item[Esperimento di Coulomb - bilancia di torsione] \hfill $\displaystyle | \vec{F} | = \frac{2c \theta}{L \sin \varphi}$ con $|\vec{\mathcal{M}}| = c \theta$

\end{description}

\lawboxtext{Legge di Coulomb}{\hfill $\displaystyle \vec{F}_e = k \frac{Q \cdot q}{R_{AB}^2} \hat{R}_{AB}$ \quad in MKS: $\displaystyle \vec{F}_e = \frac{1}{4 \pi \varepsilon_0} \frac{Q \cdot q}{R_{AB}^2} \hat{R}_{AB}$}

\begin{description}
\item[Principio di sovrapposizione (discreto)] \hfill $\displaystyle \vec{F}_T = \frac{1}{4 \pi \varepsilon_0} Q_P \sum\limits_{i=1}^{N} \frac{q_i}{R_{iP}^2}\hat{R}_{iP}$

\item[Densità di carica] \,
\begin{itemize}
\item[Volumetrica] \hfill $\displaystyle \rho(\vec{r}) = \limit{\Delta \tau}{0} \frac{\Delta q}{\Delta \tau} = \dv[•]{q}{\tau}$
\item[Superficiale] \hfill $\displaystyle \sigma(\vec{r}) = \limit{\Delta S}{0} \frac{\Delta q}{\Delta S} = \dv[•]{q}{S}$
\item[Lineare] \hfill $\displaystyle \lambda(\vec{r}) = \limit{\Delta l}{0} \frac{\Delta q}{\Delta l} = \dv[•]{q}{l}$
\end{itemize}

\item[Sovrapposizione in forma continua] \hfill $\displaystyle \vec{F}_T = \frac{1}{4 \pi \varepsilon_0} Q_P \int_\tau \rho(\vec{r}) \frac{\dd[•]{\tau}}{\Delta R^2} \hat{\Delta R}$ \quad
con $\vec{\Delta R} = \vec{r}_P - \vec{r}$

\item[Campo elettrostatico generato da una carica] \hfill $\vec{E}(\vec{R}) = \frac{1}{4 \pi \varepsilon_0} \frac{Q}{R^2} \hat{R}$

\item[Sovrapposizione (discreto e continuo)] \hfill $\displaystyle \vec{E}_T = \frac{1}{4 \pi \varepsilon_0} \sum\limits_{i=1}^{N} \frac{Q_i}{r_i^2}\hat{r}_i$ \quad | \quad $\vec{E}_T = \frac{1}{4 \pi \varepsilon_0} \int_\tau \rho(\vec{r}) \frac{\dd[•]{\tau}}{\Delta R^2} \hat{\Delta R}$

\item[Momento di dipolo elettrico] \hfill $\displaystyle \vec{P} = q d \hat{k}$ (diretto verso la carica +)

\item[Campo di un dipolo lungo l'asse] \hfill $\displaystyle \vec{E} = - \frac{1}{4 \pi \varepsilon_0} \frac{\vec{P}}{r^3}$

\item[Energia potenziale di Coulomb] \hfill $\displaystyle U(\vec{r}) = \frac{1}{4 \pi \varepsilon_0} \frac{Q q}{r} + cost$

\item[Potenziale elettrostatico] \hfill $\displaystyle V(\vec{r}) \equiv \frac{U(\vec{r})}{q} = \frac{1}{4 \pi \varepsilon_0} \frac{Q}{r} + cost$

\item[Sovrapposizione (distribuzione discreta e continua)] \hfill $\displaystyle V(P) = \frac{1}{4 \pi \varepsilon_0} \sum\limits_{i=1}^{N} \frac{Q_i}{r_i}$ \quad | \quad $\displaystyle V(\vec{r}_P) = \frac{1}{4 \pi \varepsilon_0} \int \frac{\rho \dd[•]{\tau}}{\Delta r}$

\end{description}

\lawbox{Legge di Gauss per il campo elettrico (I eq di Maxwell) in forma integrale}{\oiint\limits_\Sigma \vec{E} \cdot \vec{\dd[•]{S}} = \frac{Q_{int}}{\varepsilon_0}}

\begin{description}

\item[Per distribuzioni continue] \hfill $\displaystyle \oiint\limits_\Sigma \vec{E} \cdot \vec{\dd[•]{S}} = \frac{1}{\varepsilon_0} \iiint\limits_{V(\Sigma)} \rho \dd[•]{\tau}$

\item[Applicazione] $\Sigma$ con campo di \textbf{modulo ($|\vec{E}|$) e orientazione relativa ($\theta$) costanti} \hfill $\displaystyle |\vec{E}| = \frac{Q_T}{\cos \theta \varepsilon_0 \oiint\limits_\Sigma \vec{E} \cdot \vec{\dd[•]{S}}}$


\end{description}


\lawbox{Legge di Gauss per $\vec{E}$ in forma differenziale}{\vec{\nabla} \cdot \vec{E} = \frac{\rho}{\varepsilon_0}}

\begin{description}
\item[Distribuzione lineare indefinita] ($\lambda$ costante) \hfill $\displaystyle \vec{E} = \frac{1}{2 \pi \varepsilon_0} \frac{\lambda}{R} \hat{n}$

\item[Distribuzione piana indefinita] ($\sigma$ costante) \hfill $\displaystyle \vec{E} = \frac{\sigma}{2 \varepsilon_0} \hat{n}$

\item[Discontinuità del campo attraverso superficie carica] \hfill $\displaystyle \Delta \vec{E}_n = \frac{\sigma}{\varepsilon_0} \hat{n}$ \quad | \quad $\displaystyle \Delta \vec{E}_\tau = 0 \hat{\tau}$

\item[Proprietà dei conduttori immersi in campo elettrico esterno] (all'equilibrio)
\begin{enumerate}
\item Campo totale all'interno \textbf{nullo}; cariche disposte in modo da \textbf{schermare} campo esterno
\item In ogni punto interno la carica è \textbf{nulla}
\item Spostamento per schermare si risolve in riarrangiamento cariche \textbf{solo superficiali}
\item Immersione del conduttore nel campo ne altera le linee a seconda della propria geometria; sulla superficie il campo è normale e vale $\displaystyle \frac{\sigma}{\varepsilon_0}$
\item La superficie del conduttore è equipotenziale, così come il suo interno
\item Il campo all'interno di una cavità nel conduttore è nullo e non vi sono cariche indotte sulla superficie di questa
\end{enumerate} 
Analoghe per conduttore carico

\item[Dispersione delle punte] \hfill $\displaystyle Q_i \propto R_i \, \implies \, \sigma_i \propto \frac{1}{R_i}$

\item[Capacità di un condensatore] \hfill $\displaystyle C \equiv \frac{1}{\int_{\hspace{-0.45cm} L_i \hspace{0.22cm} A}^B \frac{\dd[•]{r_i}}{\varepsilon_0 \dd[•]{S}_i}}$
\item[Relazione tra carica e ddp] (proprietà fondamentale condensatori) \hfill $\displaystyle Q = C \Delta V$

\item[Campo in condensatore piano a facce parallele] \hfill $\displaystyle |\vec{E}_{int}| = \frac{\sigma}{\varepsilon_0}$ \quad | \quad $\displaystyle |\vec{E}_{est}| = 0$ con $\sigma = \frac{Q}{S}$

\item[Capacità cond. piano a facce p.] \hfill $\displaystyle C = \frac{\varepsilon_0 S}{d}$

\item[Condensatori in parallelo] (stessa ddp) \hfill $\displaystyle C_{eq}^{par} = \sum_i C_i$

\item[Condensatori in serie] (facce di carica opposta a due collegate) \hfill $\displaystyle \frac{1}{C_{eq}} = \sum_i \frac{1}{C_i}$

\item[Energia elettrostatica sistema di cariche] \hfill $\displaystyle U_T = \frac{1}{2} \sum\limits_{i,j=1}^{N} U_{ij} = \frac{1}{2} \sum\limits_{i=1}^{N} q_i V_{i}^{T}$ \quad | \quad $\displaystyle U_T = \frac{1}{2} \iiint_\tau \rho V_T \dd[•]{\tau}$

\item[Energia immagazzinata nel condensatore] \hfill $\displaystyle U_{E} = \frac{1}{2} C \Delta V^2 = \frac{1}{2} \frac{Q^2}{C}$

\item[Densità energetica nello spazio (vuoto)] \hfill $\displaystyle \mathcal{u}_E = \frac{1}{2} \varepsilon_0 \vec{E} \cdot \vec{E} = \frac{1}{2} \varepsilon_0 |\vec{E}|^2$

\item[Potenziale del dipolo] (origine a metà della distanza) \hfill $\displaystyle V(P) = \frac{1}{4 \pi \varepsilon_0} \frac{\vec{\mathrm{P}} \cdot \hat{r}}{r^2}$

\item[Campo del dipolo] \hfill $\displaystyle \vec{E}(\vec{r}) = \frac{3}{4 \pi \varepsilon_0} \frac{\mathrm{P}}{r^3} \big(\cos^2 \theta - \frac{1}{3}, \cos \theta \sin \theta \big)$ \qquad con $\displaystyle \theta = \arccos\big(\frac{\vec{\mathrm{P}} \cdot \vec{r}}{\mathrm{P} r}\big)$

\item[Energia e momento dipolo in campo esterno] \hfill $\displaystyle U_{\mathrm{P}} = - \vec{\mathrm{P}} \cdot \vec{E}$ \quad | \quad $\displaystyle \vec{F}_T = \vec{0}$ \quad | \quad $\displaystyle \vec{\mathcal{M}}_T = \vec{\mathrm{P}} \wedge \vec{E}$

\item[Primi termini sviluppo in multipolo] \hfill $\displaystyle V(P) = V_0 + V_{DIP} = \frac{1}{4 \pi \varepsilon_0} \frac{Q}{R} + \frac{1}{4 \pi \varepsilon_0} \frac{\vec{\mathrm{P}} \cdot \hat{R}}{R^2}$ \qquad con $\vec{\mathrm{P}} = Q ( \vec{d}_{+} - \vec{d}_{-}) = Q \vec{\delta}$

\end{description}

\section{Elettrostatica nella materia}

\begin{description}

\item[Conduttore nel condensatore] \hfill $\displaystyle \Delta V_c = E (d - D_c)$

\item[Costante dielettrica relativa e assoluta del mezzo] \hfill $\displaystyle k \equiv \frac{\Delta V_0}{\Delta V_k} > 1$ \qquad $\displaystyle \varepsilon = k \varepsilon_0$

\item[Capacità del condensatore riempito di dielettrico] \hfill $\displaystyle C_k = k C_0 = \frac{\varepsilon S}{d} = \frac{k \varepsilon_0 S}{d}$

\item[Carica di polarizzazione superficiale] indotta \hfill $\displaystyle \sigma_{\mathbb{P}} = \sigma - \sigma_k$ \qquad $\displaystyle \sigma_{\mathbb{P}} = \big(\frac{k-1}{k}\big) \sigma$

\item[Vettore polarizzazione] \hfill $\displaystyle \vec{\mathbb{P}} = n \vec{\mathrm{P}} = nq \vec{\delta}$ \qquad $\sigma_{\mathbb{P}} = \vec{\mathbb{P}}  \cdot \hat{n}$ \qquad $\displaystyle \mathbb{P} = \varepsilon_0 (E_0 - E) = \varepsilon_0 (k-1) \vec{E}$ 

\item[Suscettività] \hfill $\displaystyle \vec{\mathbb{P}} = \varepsilon_0 (k-1) \vec{E} = \varepsilon_0 \chi \vec{E}$ \qquad generale (anche anisotropi) con tensore di suscettività $\chi_{ij}$

\item[Relazione con carica di polarizzazione] \hfill $\displaystyle \div \vec{\mathbb{P}} = - \rho_{\mathbb{P}}$

\item[Spostamento elettrico] definizione e relazioni \hfill $\displaystyle \vec{D} \equiv \varepsilon_0 \vec{E} - \vec{\mathbb{P}}$ \qquad $\displaystyle \div \vec{D} = \rho_L \, \Leftrightarrow \, \oiint\limits_{\Sigma} \vec{D} \cdot \vec{\dd[•]{S}} = Q_L$ 

\item[Ulteriori relazioni] \hfill per isotropi $\displaystyle \vec{\mathbb{P}} = \frac{k-1}{k} \vec{D}$ \qquad per omogenei $\displaystyle \div \vec{\mathbb{P}} = \frac{k-1}{k} \div \vec{D}$

\item[Superficie tra dielettrici] \hfill $\displaystyle \Delta D_n = 0$ \qquad $\displaystyle \frac{\tan \theta_1}{\tan \theta_2} = \frac{k_1}{k_2}$

\item[Energia del condensatore con dielettrico] \hfill $\displaystyle \mathcal{u}_E^k = \frac{1}{2} \varepsilon E^2$ \qquad generalmente anche per anisotropi $\displaystyle \mathcal{u}_E = \frac{1}{2} \vec{E} \cdot \vec{D}$

\end{description}

\section{Correnti e circuiti}

\begin{description}
\item[Corrente elettrica] \hfill $\displaystyle i \equiv \limit{\Delta t}{0} \frac{\Delta Q}{\Delta t} = \dv[•]{q}{t}$

\item[Densità di corrente] \hfill $\displaystyle \vec{j} \equiv n e \vec{v}_d$ \qquad $\displaystyle i = \iint\limits_\Sigma \vec{j} \cdot \vec{\dd[•]{S}} = \Phi_\Sigma(\vec{j})$

\item[Stima velocità di deriva] per sezione sferica \hfill $\displaystyle |\vec{v}_d| = \frac{i}{n e \pi R^2}$ 

\end{description}

\lawbox{Equazione di continuità della carica}{i = \oiint\limits_\Sigma \vec{j} \cdot \vec{\dd[•]{S}} = - \dv[•]{q_{int}}{t}}

\lawbox{Equazione di continuità della corrente elettrica}{\vec{\nabla} \cdot \vec{j} = - \pdv[•]{\rho}{t}}

\begin{description}


\item[Legge di Ohm] con $\sigma$ conduttività \hfill $\displaystyle \vec{j} = \frac{n e^2 \tau_c}{m} \vec{E} = \sigma \vec{E}$ \qquad per semiconduttori $\vec{j} = ne^2 \big(\frac{\tau_{+}}{m_{+}} + \frac{\tau_{-}}{m_{-}}\big) \vec{E}$

\item[Resistività] \hfill $\displaystyle \rho \equiv \frac{1}{\sigma} = \frac{m}{n e^2 \tau_c}$ \qquad $\vec{E} = \rho \vec{j}$

\item[Potenza per unità di volume] \hfill $\displaystyle P_\tau = \vec{j} \cdot \vec{E} = \sigma |\vec{E}|^2 = \rho |\vec{j}|^2$

\item[Resistenza elettrica e legge di Ohm per conduttori metallici] \hfill $\displaystyle R \equiv \frac{\rho l}{S}$ \qquad | \qquad $\displaystyle V = R \cdot i$

\item[Conduttanza] \hfill $\displaystyle G \equiv \frac{1}{R} = \frac{\sigma S}{l}$ \qquad | \qquad $i = G \cdot V$

\item[Agitazione termica] \hfill $\displaystyle \rho = \rho_{20} \big(1 + \alpha \Delta T\big)$

\item[Potenza dissipata] \hfill $\displaystyle P_{diss} = R \cdot i^2 = i \cdot V$

\item[Resistori (resistenze) in serie] stessa corrente\hfill $\displaystyle R_{eq} = \sum_i R_i$ \qquad $\displaystyle P = \big(\sum_i R_i\big) i^2$

\item[Resistori in parallelo] stessa ddp ai capi \hfill $\displaystyle \frac{1}{R_{eq}} = \sum_i \frac{1}{R_i}$ \qquad $\displaystyle \frac{1}{\sum_i 1\big/R_i}$

\item[Forza elettromotrice] \hfill $\displaystyle \varepsilon = \oint_\Gamma \vec{E} \cdot \vec{\dd[•]{\gamma}}$

\item[Generatore] con resistenza interna \hfill $\displaystyle \varepsilon = (R + r) i$

\item[Campo elettromotore] non conservativo \hfill $\displaystyle \oint_\Gamma \vec{E} \cdot \vec{\dd[•]{\gamma}} = \int\limits_B^A \vec{E^\ast} \cdot \vec{\dd[•]{\gamma}}$ \qquad $\displaystyle \varepsilon - r \cdot i = V_A - V_B$

\end{description}

\lawbox{Legge di Ohm generalizzata}{V_A - V_B + \sum_k \varepsilon_k = i \cdot \sum_j R_j}

\lawbox{I Legge di Kirchhoff - legge dei nodi}{\sum\limits_A \pm i_j = \sum\limits_{entranti} i_j - \sum\limits_{uscenti} i_k = 0}

\lawbox{II Legge di Kirchhoff - legge delle maglie}{\sum_k \varepsilon_k = \sum i_j R_j}

\begin{description}
\item[RC Carica] \hfill $\displaystyle q(t) = \mathcal{E}C \big(1 - e^{-t/RC}\big)$ \quad $\displaystyle i(t) = \frac{\mathcal{E}}{R} e^{-t/RC}$ \quad $\displaystyle \Delta V_C = \mathcal{E} \big(1 - e^{-t/RC}\big)$ \quad $\displaystyle \Delta V_R = \mathcal{E}e^{-t/RC}$
\item[Frazione di carica] \hfill $\displaystyle t_f = RC \ln \frac{1}{1-f}$ \qquad con $0 \leq f < 1$
\item[Potenza e energia] \hfill $\displaystyle P_{gen} = \frac{\mathcal{E}^2}{R} e^{-t/RC}$ \qquad $\displaystyle P_R = \frac{\mathcal{E}^2}{R} e^{-2t/RC}$ \quad | \quad $\displaystyle L_{gen} = \mathcal{E}^2 C$ \qquad $\displaystyle L_{diss} = U_C = \frac{1}{2} L_{gen}$ \quad per $t_f \rightarrow \infty$
\end{description}

\section{Magnetostatica nel vuoto}

\lawbox{Legge di Gauss per $\vec{B}$ (II eq di Maxwell)}{\oiint\limits_\Sigma \vec{B} \cdot \vec{\dd{S}} = 0 \qquad \vec{\nabla} \cdot \vec{B} = 0}

\lawbox{II Legge di Laplace}{\dd{\vec{F}} = i \dd{\vec{l}} \wedge \vec{B}}

\begin{description}
\item[Forza di Lorentz] \hfill $\displaystyle \vec{F} = q \vec{v} \wedge \vec{B}$

\item[Moto di ciclotrone] \hfill $\displaystyle R = \frac{mv}{qB}$ \qquad $\displaystyle \vec{\omega} = - \frac{q}{m} \vec{B}$ \qquad $\displaystyle p = 2 \pi \frac{mv}{qB} \cos \theta$

\item[Effetto Hall] \hfill sezione $S$, spessore $h$ \qquad $\displaystyle B = n q S \frac{\Delta V_H}{i h B}$

\item[Spettrometro di massa] \hfill tipo 1) $\displaystyle R = \sqrt{\frac{2 \Delta V }{B^2} \big(\frac{m}{q}\big)}$ \quad | \quad tipo 2) $\displaystyle R = \big(\frac{m}{q}\big) \frac{E}{B_0 \cdot B}$

\item[Spira in $\vec{B}$ esterno] uniforme \hfill $\displaystyle \sum \vec{F} = \vec{0}$ \qquad $\displaystyle \vec{\mathcal{M}} = i \vec{S} \wedge \vec{B} = \vec{m} \wedge \vec{B}$ \qquad $\displaystyle U_p = - \vec{m} \cdot \vec{B} \quad \vec{\mathcal{M}} = - \dv[•]{U_p}{\theta}$
\\in regime di piccole osc. $\displaystyle T = 2 \pi \sqrt{\frac{I}{mB}}$
\end{description}

\lawbox{I Legge di Laplace (L. di Biot-Savart)}{\dd[•]{\vec{B}} = \frac{\mu_0 i}{4 \pi} \frac{\dd{\vec{l}} \wedge \vec{R}}{R^3}}

\begin{description}
\item[In condizioni stazionarie] \hfill $\displaystyle \vec{B} = \frac{\mu_0 i}{4 \pi} \int\limits_{\textrm{filo}} \frac{\dd{\vec{l}} \wedge \vec{R}}{R^3}$

\item[Singola carica in movimento] \hfill $\displaystyle \dd[•]{\vec{B}} = \frac{\mu_0 q}{4 \pi}  \frac{\vec{v} \wedge \vec{R}}{R^3}$

\item[Relazione tra campi] \hfill $\displaystyle \vec{B} = \mu_0 \varepsilon_0 \vec{v} \wedge \vec{E} = \frac{\vec{v}}{c^2} \wedge \vec{E}$

\item[Filo indefinito] \hfill $\displaystyle \vec{B}(r) = \frac{\mu_0 i}{2 \pi r} \hat{u_\varphi}$

\item[Esperienza di Ampère] \hfill per unità di lunghezza $\displaystyle \dv[•]{F_ij}{l_j} = \frac{\mu_0 i_1 i_2}{2 \pi D}$ \quad $(i,j) = (1,2) , (2,1)$

\end{description}

\lawbox{Legge di Ampère}{\oint\limits_\Gamma \vec{B} \cdot \dd[•]{\vec{r}} = \mu_0 \iint\limits_{\Sigma(\Gamma)} \vec{j} \cdot \dd[•]{\vec{S}} \qquad \vec{\nabla} \wedge \vec{B} = \mu_0 \vec{j}}

\lawbox{Legge di Ampère-Maxwell (IV eq di Maxwell)}{\oint\limits_\Gamma \vec{B} \cdot \dd[•]{\vec{r}} = \mu_0 \iint\limits_{\Sigma(\Gamma)} \vec{j} \cdot \dd[•]{\vec{S}} + \mu_0 \varepsilon_0 \dv[•]{•}{t} \big(\iint\limits_{\Sigma(\Gamma)} \vec{E} \cdot \dd[•]{\vec{S}}\big)}

\lawbox{Ampère-Maxwell in forma differenziale}{\vec{\nabla} \wedge \vec{B} = \mu_0 \vec{j} + \mu_0 \varepsilon_0 \pdv[•]{\vec{E}}{t}}

\begin{description}

\item[Campo della spira] \hfill sull'asse \quad $\displaystyle \vec{B} = \frac{\mu_0 i }{2} \frac{R_s^2}{(x^2 + R_s^2)^{3/2}} \hat{u_n}$ \qquad per $x \gg R$ \quad $\displaystyle \vec{B} = \frac{\mu_0}{2\pi} \frac{\vec{m}}{x^3}$ \newline fuori \quad $\displaystyle \vec{B} = \frac{\mu_0 i}{4 \pi r^3} \big[3 (\vec{m} \cdot \hat{u_r}) \hat{u_r} - \vec{m}\big]$

\item[Solenoide rettilineo] \hfill $\displaystyle \vec{B} = \frac{\mu_0 i n}{2} \big[\cos \phi_1 + \cos \phi_2 \big] \hat{u_n} = \frac{\mu_0 i n}{2} \bigg[\frac{d + 2x}{\sqrt[•]{4R^2 + (d + 2x)^2}} + \frac{d - 2x}{\sqrt[•]{4R^2 + (d - 2x)^2}}\bigg] \hat{u_n}$

\item[Solenoide rettilineo ideale] \hfill $\displaystyle \vec{B}(r < R) = \mu_0 i n \hat{u_n}$ \qquad $\vec{B}(r > R) = \vec{0}$

\item[Solenoide toroidale] \hfill $\displaystyle \vec{B} = \frac{\mu_0 N i}{2 \pi R} \hat{u_\varphi}$

\item[Distribuzione piana di densità di corrente] $\displaystyle \vec{B} = \frac{\mu_0}{2} \vec{J_l} \wedge \hat{u_n}$ \qquad \textbf{discontinuità} \quad $\displaystyle \vec{\Delta B} = \mu_0 \vec{J_l} \wedge \hat{u_n}$

\item[Condizioni al contorno (superfici)] \hfill $\displaystyle \Delta B_\perp = 0$ \quad $\displaystyle \Delta B_\parallel = \mu_0 J_l$

\end{description}

\section{Magnetismo nella materia}

\begin{description}

\item[Campo nel mezzo] \hfill $\displaystyle \vec{B} = k_m \vec{B_0}$ \quad | \quad nel solenoide ideale \quad $B = \mu_0 k_m i n = \mu i n$

\item[Per circuito in mezzo omogeneo] \hfill $\displaystyle \vec{B} = \frac{\mu i}{4 \pi} \oint_\Gamma \frac{\dd[•]{\vec{l}} \wedge \hat{r}}{r^2}$

\item[Suscettività e permeabilità relativa] \hfill $\vec{B} - \vec{B_0} = \vec{\Delta B} = (k_m - 1) \vec{B_0} = \chi_m \vec{B_0}$ $\longrightarrow$ $\displaystyle \chi_m = \frac{\Delta B}{B_0}$

\item[Legge di Curie] (non in regime ferromagnetico) \hfill $\chi_m = \mathcal{C} \frac{\rho}{T}$

\item[Momento magnetico orbitale] per elettrone \hfill $\displaystyle \vec{m}_0 = - \frac{e}{2 m_e} \vec{L}$

\item[Precessione di Larmor] \hfill $\displaystyle \vec{\omega}_L = \frac{e}{2 m_e} \vec{B}$ \qquad $\displaystyle \vec{m}_L = - \frac{e^2 r^2}{6 m_e} \vec{B}$

\item[Per atomi polielettronici] huhu parte da finire qui

\item[Magnetizzazione] \hfill $\displaystyle \vec{M} = \frac{\Delta \vec{m}}{\Delta \tau} = \langle \vec{m} \rangle \Delta n_\tau$

\item[Densità lineare di magnetizzazione] \hfill $\displaystyle \vec{J_{lm}} = \vec{M} \wedge \hat{u_n}$ 

\item[Densità superficiale di corrente amperiana] \hfill $\displaystyle \vec{J_M} = \vec{\nabla} \wedge \vec{M}$ \qquad $\displaystyle \oint_\Gamma \vec{M} \cdot \dd[•]{\vec{l}} = i_M$

\item[Campo $\vec{H}$] \hfill $\displaystyle \vec{H} \equiv \frac{\vec{B}}{\mu_0} - \vec{M}$ \qquad $\displaystyle \oint\limits_\Gamma \vec{H} \cdot \dd[•]{\vec{l}} = i_c$ \qquad $\displaystyle \vec{\nabla} \wedge \vec{H} = \vec{J}_c$ \quad $\displaystyle \vec{\nabla} \cdot \vec{H} = - \vec{\nabla} \cdot \vec{M}$

\item[Relazione caratteristica del mezzo magnetizzato] \hfill $\displaystyle \vec{M} = \chi_m \vec{H}$

\item[Altre relazioni] valide con campi variabili solo per mezzi lineari \hfill $\displaystyle \vec{B} = \mu_0 k_m \vec{H} = \mu \vec{H}$ \qquad $\displaystyle \vec{M} = \frac{\chi_m}{\mu} \vec{B}$

\item[Superfici di contatto tra mezzi] \hfill $\displaystyle \Delta H_\parallel = 0$ \quad \textbf{Legge di rifrazione} $\displaystyle \frac{\tan \theta_1}{\tan \theta_2} = \frac{k_1}{k_2}$

\item[Ferromagnetici] \hfill permeabilità differenziale \, $\displaystyle  \mu_d \equiv \dv[•]{B}{H}$

\end{description}

\section{Trasformazioni relativistiche}

da fare

\section{Induzione}

\begin{description}

\item[Coefficiente di mutua induzione] \hfill $\displaystyle M_{12} \equiv \frac{\mu_0}{4 \pi} \iint\limits_{\Sigma(\Gamma_2)} \big[\oint\limits_{\Gamma_1} \frac{\dd[•]{\vec{l_1}} \wedge \vec{R}}{R^3} \big] \cdot \dd[•]{\vec{S_2}}$
\end{description}

\lawbox{Legge di Faraday-Neumann (III eq di Maxwell)}{\mathcal{E}_{ind} = \oint\limits_\Gamma \vec{E} \cdot \dd[•]{\vec{r}} =  - \dv[•]{\Phi_\Sigma(\vec{B})}{t} = - \dv[•]{•}{t} \big(\iint\limits_{\Sigma(\Gamma)} \vec{B} \cdot \dd[•]{\vec{S}}\big)}

\lawbox{Faraday-Neumann in forma differenziale}{\vec{\nabla} \wedge \vec{E} = - \pdv[•]{\vec{B}}{t}}

\begin{description}

\item[Autoinduzione] \hfill $\displaystyle \mathcal{E}_{ind} = - \dv[•]{•}{t}(Li)$

\item[Induttanza del solenoide ideale] \hfill $\displaystyle L = \mu_0 n^2 V = \mu_0 n N S$

\item[RL in accensione] \hfill $\displaystyle i(t) = \frac{\mathcal{E}}{R}\big(1 - e^{-Rt/L}\big)$ \qquad $\displaystyle \mathcal{E}_{ind} = - \mathcal{E} e^{-Rt/L}$

\item[Potenza ed energia] \hfill $\displaystyle \dd[•]{U_L} = L i \dd[•]{i}$ \qquad $\displaystyle U_L(t) = \frac{1}{2} L i(t)^2$ \quad $\displaystyle U(\infty) = \frac{1}{2} L \big(\frac{\mathcal{E}}{R}\big)^2$

\item[Densità di energia magnetica] \hfill $\displaystyle u_B = \frac{1}{2 \mu_0} B^2$
\end{description}

\section{Teorema di Poynting e onde elettromagnetiche}

\lawbox{Teorema di Poynting}{W = \dv[•]{L}{t} = \iiint\limits_\tau \vec{j} \cdot \vec{E} \dd[•]{\tau} = - \dv[•]{•}{t} \big[\iiint\limits_\tau \big( \frac{1}{2 \mu_0} B^2 + \frac{\varepsilon_0}{2}E^2 \big) \dd[•]{\tau}\big] - \oiint\limits_{\Sigma(\tau)} \big(\frac{1}{\mu_0} \vec{E} \wedge \vec{B}\big) \cdot \dd[•]{\vec{S}}}

\begin{description}

\item[Densità volumetrica di energia elettromagnetica] \hfill $\displaystyle u = \frac{1}{2 \mu_0} B^2 + \frac{1}{2} \varepsilon_0 E^2$

\item[Vettore di Poynting] \hfill $\displaystyle \vec{S'} = \frac{1}{\mu_0} \vec{E} \wedge \vec{B}$

\item[Equazioni delle onde EM] \hfill $\displaystyle \laplacian \vec{E} = \mu_0 \varepsilon_0 \pdv[2]{\vec{E}}{t}$ \qquad $\displaystyle \laplacian \vec{B} = \mu_0 \varepsilon_0 \pdv[2]{\vec{B}}{t}$ $\quad \implies \quad$ $\displaystyle c = \frac{1}{\sqrt{\varepsilon_0 \mu_0}}$


\end{description}


















\section{Costanti e proprietà}
\subsection{Costanti}
\begin{itemize}
\item Costante dielettrica del vuoto $\varepsilon_0$ \dotfill $\displaystyle 8.85 \times 10^{-12} \, \mathrm{C^2 N^{-1} m^{-2}}$
\item Carica dell'elettrone $e$ \dotfill $\displaystyle 1.6021766208 \times 10^{-19} \, \mathrm{C}$
\item Permeabilità magnetica del vuoto $\mu_o$ \dotfill $\displaystyle 4 \pi \times 10^{-7} \, \mathrm{T m A^{-1}}$





\end{itemize}




\end{document}