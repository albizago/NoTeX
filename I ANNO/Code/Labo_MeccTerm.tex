\documentclass[10pt, oneside]{book}
\usepackage[utf8]{inputenc}
\usepackage{amsmath}
\usepackage{amsthm}
\usepackage{amssymb}
\usepackage{enumitem}
\usepackage{mdframed}
\usepackage{hyperref}
\usepackage{multirow}
\usepackage{systeme}
%\usepackage{moresize}
\usepackage{relsize}
\usepackage{comment}
\usepackage{cancel}
\usepackage[a4paper,left=2.3cm, right=2.3cm, top=2.1cm, bottom=2.1cm]{geometry}
\usepackage[italian]{babel}
\usepackage{ragged2e}
\usepackage{fancyhdr}
\usepackage[Lenny]{fncychap}
%\ChTitleVar{\raggedleft\Huge\bfseries\fontfamily{cmss}\selectfont}
\usepackage{tikz}
\usepackage{pgfplots}
\usetikzlibrary{decorations.pathreplacing,calligraphy}
\usetikzlibrary { decorations.pathmorphing, decorations.shapes, }
\usepackage[cal=dutchcal]{mathalfa}
\usepackage{wrapfig}
\usepackage{physics}
\usepackage{listings}
\usepackage{graphicx}
\usepackage{eso-pic, transparent}
\usetikzlibrary{calc}
\usetikzlibrary{positioning}
\pgfplotsset{width=10cm,compat= newest}
\usepgfplotslibrary{external}
%\tikzexternalize
\usepackage{bbold}
\swapnumbers
\hypersetup{
    colorlinks=true,
    linkcolor=blue,
}
 
\renewcommand{\arraystretch}{1.3}
\setlength{\tabcolsep}{0.5cm}

\renewcommand{\qedsymbol}{\hfill \large $\blacksquare$}
%\renewcommand{\descriptionlabel}{$\ast$}
\newcommand{\celsius}{\, \mathrm{{}^\circ C}}
\newcommand{\evolt}{\, \mathrm{eV}}
\newcommand{\kelvin}[1]{\, \mathrm{K^{#1}}}
\newcommand{\joule}[1]{\, \mathrm{J^{#1}}}
\newcommand{\pascal}[1]{\, \mathrm{Pa^{#1}}}
\newcommand{\molvol}{\mathrm{v}}
\newcommand{\molms}{\mathcal{M}}
\newcommand{\angstrom}{\, \mathrm{\AA}}
\newcommand{\meters}[2]{\, \mathrm{#1 m^{#2}}}
\newcommand{\grams}[1]{\, \mathrm{g^{#1}}}
\newcommand{\mols}[1]{\, \mathrm{mol^{#1}}}
\newcommand{\limit}[2]{\lim\limits_{#1 \rightarrow #2}}
\newcommand{\deltas}[1]{\Delta S_{#1}}
\newcommand{\clausius}[2]{\frac{\delta Q_{#1}}{T_{#2}}}
\newcommand{\mean}[1]{\langle #1 \rangle}
\newcommand{\infobox}[2]{\vspace{0.5cm}~\\ \textbf{#1} \hrulefill \vspace{0.2cm}\\#2 {}\,\\\hrule \vspace{0.5cm}}
\newcommand{\lawbox}[2]{\begin{center}
\framebox{
\parbox{\linewidth}{
\vspace{0.3cm}
\textbf{#1} \hfill $\displaystyle #2$
\vspace{0.3cm}
}
}
\end{center}}

\newcommand{\lawboxtext}[2]{\begin{center}
\framebox{
\parbox{\linewidth}{
\vspace{0.3cm}
\textbf{#1} \vspace{0.1cm} \\#2
\vspace{0.3cm}
}
}
\end{center}}

\newcommand{\ds}{\displaystyle}
\newcommand{\tendsto}[2]{\xrightarrow[#1 \rightarrow #2]{}}
\newcommand{\integral}[4]{\int_{#1}^{#2} #3 \, \mathrm{d}#4}
\newcommand{\molhtv}{\mathcal{c}_V }
\newcommand{\molhtp}{\mathcal{c}_p }
\newcommand{\setel}[1]{ \{ #1\} }
\newcommand{\setprop}[3]{ \{ #1 \, \in \, #2 \, : \, #3 \}}
\def\upint{\mathchoice%
    {\mkern13mu\overline{\vphantom{\intop}\mkern7mu}\mkern-20mu}%
    {\mkern7mu\overline{\vphantom{\intop}\mkern7mu}\mkern-14mu}%
    {\mkern7mu\overline{\vphantom{\intop}\mkern7mu}\mkern-14mu}%
    {\mkern7mu\overline{\vphantom{\intop}\mkern7mu}\mkern-14mu}%
  \int}
\def\lowint{\mkern3mu\underline{\vphantom{\intop}\mkern7mu}\mkern-10mu\int}
%\def\derx{\frac{\textrm{d}^2}{\textrm{d}x^2}}

\title{SIOLI'S LECTURES ON THERMODYNAMICS}
\author{MAXIMILIANO SIOLI}
\date{a.a. 2022-2023}

\begin{document}
\makeatletter
\begin{titlepage}
\vspace{-2.1cm}
\AddToShipoutPictureBG*{%
  \AtPageLowerLeft{%
    \transparent{0.6}\includegraphics[width=\paperwidth,height=\paperheight]{cover2.jpg}%
  }%
}
\hspace{0cm}
\vfill
\, \\\larger[20]\textsf{\textbf{Appunti di teoria \\per l'esame di Laboratorio di \\Meccanica e Termodinamica}}
\\\smaller[2]Alberto Zaghini
\\a.a. 2022-2023
\\~\\ \larger[20]\,\,
\\~\\ \,\,

\vfill
\hspace{0cm}
\end{titlepage}
\makeatother

\tableofcontents
\newpage

\chapter{Grandezze fisiche, dimensioni, cifre significative}
Scopo dell'attività scientifica = \textbf{comprensione dei fenomeni naturali}. Alla base di ciò si hanno \textbf{osservazioni} (serie di) $\rightarrow$ riconoscimento aspetti ricorrenti (caratteristici).\\
\textbf{Caratteristiche} \textbf{Osservabili} (analisi preliminare \textit{qualitativa}) $\longrightarrow$ \textbf{misurabili} (analisi \textit{quantitativa}) = \textbf{ESPERIMENTI}.\\
\paragraph{Metodo scientifico} galileiano
\begin{enumerate}
\item Osservazione acritica
\item Riduzione delle osservazioni $\rightarrow$ selezione delle informazioni \textbf{(*)}
\item Formulazione delle leggi
\item $\displaystyle \textrm{verifica sperimentale} : \begin{cases} \displaystyle NO & \rightarrow 1. \\ \\ \displaystyle SI' & \rightarrow \textrm{Legge convalidata}

\end{cases}$
\end{enumerate}
La verifica sperimentale avviene in condizioni \textit{privilegiate} \textbf{(*)}.\\
\begin{description}
\item[1,2] introduzione di nuove quantità caratterizzanti $\rightarrow$ migliore descrizione dei fenomeni
\item[3] modello interpretativo
\item[4] rafforzamento / falsificazione
\end{description}
\textbf{Leggi} = relazioni funzionali tra quantità \textbf{misurabili} $\rightarrow$ \textbf{Grandezze fisiche} = proprietà quantificabili\\
Ogni grandezza $\in$ \textbf{classe = dimensione} ($\mathrm{[1]}$ = adimensionali, \textbf{come gli argomenti di tutte le funzioni trascendenti}) -> ogni classe contiene tutte le grandezze \textbf{omogenee} che sole sono confrontabili e sommabili/sottraibili.\\
Per tutte le grandezze fisiche possibile stabilire tramite un parametro d'ordine ($>$ $=$ $<$) relazione d'ordine transitiva. Grandezze per cui è definita anche regola di composizione (somma, differenza) sono \textbf{additive}, non additive in caso contrario.\\
Stessa classe = stesso significato ma differente procedura:
\begin{itemize}
\item Misura diretta
\item Misura \textbf{indiretta} (derivazione attraverso procedura matematica)
\end{itemize}

\subsubsection*{Angolo piano}
\[\alpha = \frac{\overset{\large\frown}{AB}}{r} \qquad 0 \leq \alpha \leq 2 \pi \textrm{ in radianti (rad)}\]
\subsubsection*{Angolo solido}
\[\Omega = \frac{A}{r^2} \qquad 0 \leq \Omega \leq 4 \pi \textrm{ in steradianti (sr)}\]

\subsection{Analisi dimensionale}
\[G = c A^\alpha B^\beta C^\gamma \, \implies \, [G] = [A]^\alpha [B]^\beta [C]^\gamma\]

\section{Misura}
Grandezza fisica $\rightarrow$ processo di \textbf{misurazione (o misura)} $\rightarrow$ assegnazione di un \textbf{valore numerico} (misura). 3 elementi:
\begin{enumerate}
\item Materiale / oggetto (o sistema di oggetti)
\item Grandezza caratterizzante materiale/sistema
\item Procedura utilizzata per misura
\end{enumerate}
1,2,3 $\rightarrow$ \textbf{precisa descrizione}.\\
Valore numerico permette confronto con \textbf{campione di riferimento (= unità di misura)} = grandezza omogenea $\rightarrow$ rapporto numerico quantitativo.

\subsection{Sistema internazionale}
Sistema di unità di misura = insieme di 
\begin{itemize}
\item \textbf{Definizioni} delle unità fondamentali / di base misurate (solitamente) in modo \textbf{diretto}
\item \textbf{Regole} per la definizione di grandezze \textbf{derivate} : il coefficiente moltiplicativo nell'espressione come funzione di grandezze fondamentali è \textbf{sempre 1} (\textbf{coerenza} sistema di udm)
\end{itemize}
Definizioni devono essere quanto più
\begin{enumerate}
\item Indipendenti da tempo e luogo (stabili)
\item Precise
\item Riproducibili (legato a 1.)
\end{enumerate}
$\Rightarrow$ \textbf{Aggiornamento}.

\paragraph{7 grandezze fondamentali} definite sulla base di 7 \textbf{costanti fisiche} fondamentali. La loro scelta garantisce
\begin{itemize}
\item Indipendenza reciproca
\item Completezza (sufficienti per ottenere tutte le derivate)
\end{itemize}
I multipli ed i sottomultipli devono essere \textbf{decimali}.

\paragraph{Bureau International des Poids et Mesures} stabilisce unità e costanti e presiede all'aggiornamento delle definizioni e dei campioni di riferimento

\paragraph{Unità e costanti}
\begin{table}[h!]
\centering
\begin{tabular}{c|c|c}
\textbf{Grandezza} & \textbf{udm} & \textbf{Costante}\\\hline
tempo & secondo s & frequenza di transizione iperfine Cesio\\
lunghezza & metro m & velocità della luce nel vuoto\\
massa & chilogrammo kg & costante di Planck\\
temperatura termodinamica & kelvin K & carica elementare\\
q.tà di sostanza & mole mol & numero di Avogadro\\
intensità luminosa & candela cd & efficacia luminosa\\
intensità di corrente & ampère A & carica elementare
\end{tabular}
\end{table}

\subsubsection*{Convenzioni d'uso}
\begin{itemize}
\item nomi comuni: \textbf{iniziale minuscola}
\item invarianti al plurale
\item se
\begin{itemize}
\item \textbf{non} accompagnate da numero: scritte per esteso (\textit{Il metro})
\item accompagnate da numero: simbolo (maiuscola) \textbf{non in corsivo, no seguito da punto, dopo} il valore
\end{itemize}
\end{itemize}
Esponenti derivate $\in \, \mathbb{Z}$

\paragraph{Multipli e sottomultipli} solo un prefisso! (per kg si fa riferimento a g)
\begin{table}[h!]
\centering
\begin{tabular}{c c | c c}
$10^{18}$ & exa (E) & $10^{-1}$ & deci (d)\\
$10^{15}$ & peta (P) & $10^{-2}$ & centi (c)\\
$10^{12}$ & tera (T) & $10^{-3}$ & milli (m)\\
$10^{9}$ & giga (G) & $10^{-6}$ & micro $\mathrm{\mu}$ \\
$10^{6}$ & mega (M) & $10^{-9}$ & nano (n)\\
$10^{3}$ & kilo (k) & $10^{-12}$ & pico (p)\\
$10^{2}$ & etto (h) & $10^{-15}$ & femto (f)\\
$10^{}$ & deca (da) & $10^{-18}$ & atto (a)\\
\end{tabular}
\end{table}

\subsection{Cifre significative}
= tutte le cifre di un numero a partire dalla prima \textbf{diversa da 0}, lette da sx a dx.\\
Per evitare ambiguità con \textbf{zeri non significativi}, si utilizza \textbf{notazione scientifica} (gli zeri in mezzo od in fondo \textbf{sono significativi}, in quanto le c.s. descrivono la precisione di una misura)
\[A \times 10^n \qquad 1 \leq A < 10 \quad n \, \in \, \mathbb{Z}\]
ove $A$ non termina con 0.

\subsubsection*{Operazioni e CS}
Regola generale:
\begin{quote}
numero di cifre significative nel risultato di operazioni condotte su due o più misure di grandezze fisiche = numero di cifre significative della misura meno accurata
\end{quote}
\paragraph{Arrotondamento}
\begin{itemize}
\item prima cifra eliminata $\geq 5$ : si aumenta ultima c.s. di 1
\item se $< 5$ (\textbf{escluso}) : ultima c.s. invariata
\end{itemize}
\paragraph{Addizione e sottrazione} ultima c.s. = ultima ottenuta da somma o differenza c.s. delle due misure iniziali
\paragraph{Prodotto e quoziente} numero di c.s. = minimo numero di c.s. tra le misure iniziali
\paragraph{Numeri esatti} $\rightarrow$ considerati come n. con \textit{infinite} c.s.
\paragraph{Ordine di grandezza} = prima cifra a sx $\neq 0$ (dopo arrotondamento) moltiplicata per opportuna potenza di $10$\\Utilizzate in confronto grandezze omogenee.

\chapter{Incertezze}
\begin{quote}
\textbf{ad ogni misura è associata un'incertezza}
\end{quote}
Non è \textbf{mai} eliminabile del tutto (per quanto riducibile) x due ragioni:
\begin{enumerate}
\item Sensibilità strumentale limitata (soglia di risoluzione, sotto cui impossibile distinguere grandezze)
\item Inevitabilità errori nell'effettuazione dell'operazione di misura
\end{enumerate}
\paragraph{Processo di misura} = confronto grandezze con udm $\rightarrow$ determina \textbf{Intervallo di valori}:
\[n_0 u + \frac{n_1}{10} u + ... + \frac{n_k}{10^k} u < G < n_0 u + \frac{n_1}{10} u + ... + \frac{(n_k+1)}{10^k} u\]
\begin{itemize}
\item \textbf{Limite di riproducibilità} di scala
\item \textbf{Soglia di riproducibilità} : condizioni del sistema o dell'ambiente che pregiudicano riproducibilità operazione di misura
\end{itemize}
\paragraph{Perché ridurre l'incertezza?}
\begin{itemize}
\item Evidenziare fenomeni precedentemente ignorati / nascosti
\item Permettere confronto tra misure omogenee: valutazione compatibilità reciproca e con grandezze di riferimento
\end{itemize}

\section{Misure dirette}
\begin{quote}
incertezza = risoluzione strumento = più piccola variazione di grandezza che str. riesce ad apprezzare
\end{quote}
misura = localizzazione punto su scala graduata / display digitale
\begin{itemize}
\item Strumenti analogici: risoluzione = metà minima distanza tra tacche
\item Str. digitali: ris = mezza unità digit meno significativo
\end{itemize}
Ogni strumento è dotato di \textbf{data sheet} che specifica
\begin{enumerate}
\item Range (portata)
\item Risoluzione 
\item Condizioni ambientali adatte all'utilizzo
\end{enumerate}
Per digitali: risoluzione diminuisce aumentando numero di bit in uscita.

\subsection{Incertezza assoluta}
Per una grandezza $x$ è $\Delta x$ (chiaramente le due sono grandezze omogenee)\\
L'esito di una misura è espresso secondo
\[x = (x_{best} \pm \Delta x) \, \mathrm{u.m.}\]
inc si rappresenta graficamente tramite \textbf{barra di errore}; si suppone il \textit{valor vero} della grandezza da misurare ricada nell'intervallo di misura così definito.
\paragraph{Regole}
\begin{enumerate}
\item L'incertezza va arrotondata a \textbf{una sola} c.s.
\item L'ultima c.s. della $x_{best}$ deve essere \textbf{dello stesso ordine di grandezza} dell'ultima c.s. di $\Delta x$ (ovvero stessa posizione decimale se espresse in notazione scientifica con stessa potenza di 10)
\end{enumerate}
Incertezze \textbf{strumentali} o \textbf{di lettura} = sinonimi; entrambe si riferiscono a incertezze \textbf{massime}.

\paragraph{Precisione} = \textbf{incertezza relativa} = rapporto inc assoluta / valore ottimale
\[\frac{\Delta x}{x_b}\]
\begin{enumerate}
\item Può essere espressa in percentuale
\item \'E adimensionale (rapporto omogenee) $\rightarrow$ permette confronto tra misure di grandezze non omogenee
\end{enumerate}

\subsection{Errore}
\textbf{Non} è sinonimo di incertezza!\\
Corrisponde a \textbf{differenza tra \textit{valore reale} grandezza (che si ipotizza esista) e valore \textit{best} trovato}
\[E = \big| X - x_b\big|\]
Non si misura! L'incertezza ne è la \textbf{miglior stima}

\section{Discrepanza}
= differenza tra due valori misurati della stessa grandezza (omogenei $\rightarrow$ confrontabili). Se entrambi espressi come 
\[x_i = (x_b^i \pm \Delta x_i)\]
allora
\[discr = \big| x_b^2 - x_b^1\big|\]
Può essere
\begin{itemize}
\item \textbf{Significativa} se \textbf{non} $\exists$ valori compatibili con entrambi gli intervalli di misura (non si sovrappongono)
\[ [x_1 \pm \Delta x_1] \, \cap \, [x_2 \pm \Delta x_2] = \emptyset \]
\item \textbf{Non s.} se si ha almeno un punto di sovrapposizione
\end{itemize}
Data una grandezza con valore accettato $X$ noto e misure $x_i$, se la discrepanza $\displaystyle \big| X - x_i \big|$ è
\begin{itemize}
\item Significativa: la misura $x_i$ è \textbf{incompatibile} con $X$
\item Non significativa: la misura $x_i$ è \textbf{compatibile} con $X$
\end{itemize}

\section{Incertezze sistematiche}
= dovute a fattori non controllati (ma controllabili) \textbf{insiti nell'apparato di misura o dovuti ad operazioni errate di misura}. \textbf{Non} includono sbagli occasionali (e.g. letture errate)\\
Portano a \textbf{sottostima / sovrastima} \textit{sistematica} (ovvero sempre nel medesimo verso e - circa - della stessa entità).
\paragraph{Possibili sorgenti}
\begin{itemize}
\item Calibrazione errata (\textit{offset}) di uno strumento o riproduzione errata di u.m. nella scala di uno strumento
\item Condizioni ambientali differenti da quelle prescritte per la procedura di misura (che possono dare e.g. effetti termici non considerati)
\item Presenza di fattori che influiscono sulla grandezza stessa che si va a misurare (e.g. mancato isolamento termico o presenza di fondo radioattivo)
\end{itemize}
\paragraph{Come rilevarli?} si ripete misura in condizioni sperimentali differenti, cercando di ottenere sistematico trascurabile rispetto a risoluzione e quindi incertezza. Generalmente \textbf{non} si hanno \textbf{regole fisse} (non sono computabili e riducibili come incertezze massima









\end{document}