\documentclass[10pt, oneside]{Book}
\usepackage[utf8]{inputenc}
\usepackage{graphicx} % Required for inserting images
\usepackage{amsmath}
\usepackage{amsthm}
\usepackage{amssymb}
\usepackage[a4paper,left=2.1cm, right=2.1cm, top=2cm, bottom=2cm]{geometry}
\usepackage{verbatim}
\usepackage{hyperref}
\renewcommand{\rmdefault}{cmss}

\title{Jaco's Fantastic Adventures}
\author{A C++ Referenza}
\date{2023}

\begin{document}

\maketitle

\tableofcontents

\chapter{Introduzione}

\section{Siti utili}
\begin{itemize}
\item \boxed{\texttt{https://godbolt.org}} Compiler Explorer
\item \boxed{\texttt{https://explainshell.com/}} per comandi shell unix
\item \boxed{\texttt{https://en.cppreference.com/w/}} Reference
\item \boxed{\texttt{https://github.com/Programmazione-per-la-Fisica}} repository lezioni + tutorial + labo
\item \boxed{\texttt{https://hackingcpp.com/}} Risorse varie utili
\end{itemize}

\section{Shell}
\begin{itemize}
\item \boxed{\texttt{<command> --help}} guida rapida

\item \boxed{\texttt{cd <nome directory>}} \boxed{\texttt{cd /<directory path>}} \\(per nomi con spazi \texttt{Pinco\ Pallino} o \texttt{'Pinco Pallino'})
\\\boxed{\texttt{cd ..}} cartella parente, \boxed{\texttt{cd .}} cartella attuale, \boxed{\texttt{cd $\thicksim$}} directory home

\item \boxed{\texttt{pwd}} posizione directory

\item \boxed{\texttt{ls}} \boxed{\texttt{ls /<dir path>}} elenca contenuti directory \\\boxed{\texttt{ls -l}} lunga,  \boxed{\texttt{ls -a}} con file nascosti, \boxed{\texttt{ls -a}} \boxed{\texttt{ls -la}} con file che iniziano per '.',  \boxed{\texttt{ls -A}} senza file impliciti con '.' e '..'

\item \boxed{\texttt{mkdir <dir name>}} crea cartella, \boxed{\texttt{mkdir -p <dir1>/<dir2>/<dir3>}} crea cartelle nidificate \boxed{\texttt{rmdir <dir name>}} rimuove dir (se vuota!)

\item \boxed{\texttt{rm <path>}} rimuove file / cartella(e), \boxed{\texttt{rm -f <...>}} forza rimozione (ignora file e argomenti non esistenti e PROTEZIONI FILE) \footnote{NON USARE MAI \boxed{\texttt{rm -rf}} sulla cartella di root: si perde tutto!!}, \boxed{\texttt{rm -d <...>}} per directory vuote, \boxed{\texttt{rm -r <...>}} \boxed{\texttt{rm -R <...>}} ricorsivo: per eliminare cartelle, \boxed{\texttt{rm -i <..>}} chiede di confermare ogni rimozione, \boxed{\texttt{rm -I <...>}} chiede solo una volta per $\geq3$ argomenti, \boxed{\texttt{rm -v <...>}} 'verbose' : restituisce info su cosa si elimina

\item \boxed{\texttt{file <file path>}} tipo file

\item \boxed{\texttt{touch <filename>}} creare file vuoto

\item \boxed{\texttt{cat <filename>}} stampare su shell contenuto file

\item \boxed{\texttt{cp <source> <destination>}} copia file sorgente specificato nella destinazione: questa può essere un file (nome della copia) o una cartella (vi crea copia con nome del source) \boxed{\texttt{cp -r <...> <...>}} per copiare cartelle

\item \boxed{\texttt{mv <source> <destination>}} sposta file (o cartella) : sintassi analoga alla copia. Può essere usato per \textbf{rinominare} spostando in stessa posizione di memoria ma con nome diverso file/cartella di destinazione

\item \textbf{Aprire VS Code} \boxed{\texttt{code .}}

\item \textbf{Modificare con editor nano} \boxed{\texttt{nano <filename>}} o \boxed{\texttt{nano <file path>}}
\\\boxed{Ctrl} + \boxed{O} = salva; \boxed{Ctrl} + \boxed{X} = chiudi


\item \textbf{Compilare : } 
\\\boxed{\texttt{g++ -Wall -Wextra name.cpp - o <compiledname>}} $\rightarrow$ \boxed{\texttt{./<compiledname>}}
\\Per usare sanitizer (controllare memory leak) \boxed{\texttt{g++ -Wall -Wextra -fsanitize=address <...> -o <...>}}
\\Per produrre file senza linking \boxed{\texttt{g++ ... -c <name>.cpp}}
\\Per salvare tutti i file intermedi generati nel processo di compilazione \boxed{\texttt{g++ ... -save-temps <name>.cpp }}

\item \textbf{Formattazione : }
\\\boxed{\texttt{clang-format --dump-config -style=google > .clang-format}} genera file di form.
\\\boxed{\texttt{clang-format -i <file name>}} formatta file (\texttt{-i} impone modifiche direttamente sul file). 
\\Oppure da VS Code combinazione di tasti: \boxed{Alt} + \boxed{\texttt{$\uparrow$ \quad}} + \boxed{\texttt{F}} 

\item \textbf{Scaricare file} \boxed{\texttt{curl <link> -o <filename>}} 
\item \textbf{Scaricare repository da GitHub} \boxed{\texttt{git clone <link file .git>}}
\item \textbf{Installare pacchetti} \boxed{\texttt{sudo apt install <name(s)>}}

\item \textbf{Aggiornamenti} verificare versione installata \boxed{\texttt{cat /etc/os-release}}, aggiornare distro \boxed{\texttt{sudo do-release-upgrade}}. Per aggiornamento periodicamente catalogo pacchetti \boxed{\texttt{sudo apt update \&\& sudo apt upgrade}} 

\item \textbf{Problemi con utenti} se Ubuntu si apre come utente root, aprire Powershell Windows e digitare 
\\\boxed{\texttt{ubuntu.exe config --default-user <USERNAME>}}
\end{itemize}

\paragraph{Accesso a cartelle: WSL \& Windows} \textbf{da Windows a WSL} digitare nella barra indirizzi \begin{verbatim}
\\wsl$\Ubuntu\
\end{verbatim}
oppure \boxed{\texttt{explorer.exe /home/user}} \boxed{\texttt{explorer.exe .}}
\\\textbf{da WSL a Windows} digitare nella shell \boxed{\texttt{cd /mnt/C/}} (disco)

\section{Commenti}
\begin{verbatim}
int main() // commento singola linea
{ 
     /* commento
     multilinea */
}
\end{verbatim}

\chapter{Basics}

\section{Variabili}
Identificatori che assegnano nome a oggetti
\\\begin{verbatim}
int i; // dichiarazione
i = 23; // assegnazione
int j{40}; // dichiarazione e inizializzazione

// inizializzazione con braces {} : universale ma previene narrowing (conversione implicita):

int l{1.}; // error!

int const m{23}; // non modificabile

auto h{...}; // compiler deduces type from initializer (expression)
             // preserves const-ness of references (see later)!
\end{verbatim}

\subsection{Stringhe}
Tipo non primitivo, da Standard Library
\begin{verbatim}
std::string ciao{"Letterale"};
ciao = ciao + " diverso"; // concatenazione
ciao.size(); // capacity
ciao.empty() // boolean: true if string is empty, otherwise false
ciao = "Pesce";
bool b{ciao == "Domani" || ciao != "Qui" || ciao > "a" || ciao < "a" || ciao <= "a" || ciao >= "a" }; 
// comparison
char c = ciao[2]; char d = ciao.back(); char e = ciao.front() // access to carachter
ciao.insert(); ciao.append(), ciao.erase() // insertion, removal
ciao.find(); // search

// PER CHAR:
#include <cctype>
...
std::isalpha(<char>); // verifica se carattere è alfabetico: 0 se non lo è,
                      // =/= 0 altrimenti
std::tolower(<char>); // converte in minuscolo
std::toupper(<char>); // converte in maiuscolo
\end{verbatim}

\section{Operatori}
\subsection{Aritmetici}
\begin{verbatim}
+a; -a; a + b; a - b; a * b; a / b; a % b; ~a; a & b; a | b; a ^ b; a << b; a >> b;
\end{verbatim}
\subsection{Logici}
\begin{verbatim}
!a; // not 
a && b; // and
a || b; // or
\end{verbatim}
\subsection{Confronto}
\begin{verbatim}
a == b; a != b; a < b; a > b; a >= b; a <= b;
\end{verbatim}
\subsection{Incremento}
\begin{verbatim}
++a; --a; a++; a--;
\end{verbatim}
\subsection{Assegnazione}
\begin{verbatim}
a = b; a += b; a -= b; a *= b; a /= b; 
a %= b; // resto
a &= b; a |= b; a ^= b; a <<= b; a >>= b;
\end{verbatim}
\subsection{Accesso}
\begin{verbatim}
a[b]; // subscript
*a; // dereference
&a; // address-of
a->b; // structure dereference
a.b; // access member
\end{verbatim}
\subsection{Altri}
\begin{verbatim}
a(...); a, b; ? : ;
\end{verbatim}

\chapter{Flow control}

\section{Statements / enunciati}
\begin{verbatim}
a + 4; // expression statement
; // empty statement
{int b = 3; a += b;} // compound statement (block)
\end{verbatim}

\section{If, then, else}
\begin{verbatim}
if ( <boolean condition> ) { 
   ... // cond true
} else { 
   ... // statement
} // else is optional
if (i < 1) <statement>; else <statement2>; // senza graffe
\end{verbatim}

\section{While}
\begin{verbatim}
while ( <boolean condition> ) <statement>
// executes repeatedly until condition becomes false
\end{verbatim}

\section{For}
\begin{verbatim}
for (<initial statement> ; <condition-expression> ; <expression>) <statement>
// alla fine di ogni iterazione: expression eseguita, poi condition-expr valutata

int i{1}; // inizializzata fuori dal loop
for ( ; i < 20; ++i) ; // senza graffe
\end{verbatim}
\subsection{Range-for}
\begin{verbatim}
for( <range-declaration> : <range-expression> ) <statement>
// iterate on all elements of <range-expression>
// <range-declaration> declares variable of SAME TYPE of elements of range : CONST REFERENCE
\end{verbatim}

\section{Break, continue}
\begin{verbatim}
for(...){
    ...
    continue; // jump to end of current iteration
    ...
    break; // terminate the loop
}
\end{verbatim}

\section{Numeri e input}
\begin{verbatim}
#include <iomanip>
... 
4. ; -1.4e7; 13.25E-2; // double
4.f ; -1.4e7f; 13.25E-2F; // float
int n
std::cout << std::setprecision(n) << 32.; // imposta precisione numero (inclusione header richiesta)
\end{verbatim}

\section{Funzioni matematiche standard}
\begin{verbatim}
#include <cmath>
· · ·
double x{· · ·};
std::sqrt(x);
std::pow(x, .5); // (<base>, <esponente>)
std::sin(x);
std::log(x);
std::abs(x);
\end{verbatim}

\section{Conversione}
\begin{verbatim}
int a = 1 + 2.34; // implicita
a += static_cast<int>(2.34 + 3.21); // esplicita
\end{verbatim}

\chapter{Functions}
\begin{verbatim}
<return-type> function-name( <parameter1>, ...); // declaration

<return-type> function-name( <parameter1>, ...){
    ... // statement
    ... function-name( ... ); // recursion: function can call itself
    return <expression>;      // expr must be convertible to ret type
                              //more return stat. in block (all of same type!!)
} // definition
// parameter type must be specified, name is optional

void ciao(){ // nothing returned + no parameter
    ...
    return;
}

int func(...);
auto func(...) -> int; // equivalenti
\end{verbatim}
\subsection{Overloading}
\begin{verbatim}
int func(int);
int func(char);

func('a');
func(23);

// segnatura determinata SOLO DA PARAMETRI (no return-type)
\end{verbatim}

\subsection{Main}
\begin{verbatim}
int main(){
	return 0; // = successo (opzionale)
}
// altre sintassi: 
#include<cstdlib>
...
    return EXIT_SUCCESS; 
    return EXIT_FAILURE;
\end{verbatim}
Per valore restituito, da shell \boxed{\texttt{\$?}}

\section{Doctest}
\begin{verbatim}
#define DOCTEST_CONFIG_IMPLEMENT_WITH_MAIN
#include "doctest.h"
...
// no main!!

// dopo funzioni:
TEST_CASE("Testing isqrt") {
CHECK(isqrt(0) == 0);
CHECK(isqrt(9) == 3);
CHECK(isqrt(10) == 3);
CHECK(isqrt(-1) == 0);
· · ·
}
\end{verbatim}
\subsection{Per avere subcases}
Per ogni subcase il TEST CASE è eseguito da capo.

\begin{verbatim}
TEST_CASE("vectors can be sized and resized") {
    std::vector<int> v(5);

    REQUIRE(v.size() == 5);
    REQUIRE(v.capacity() >= 5);
    // opzionale, necessario se si impongono requirements comuni per tutti i subcase

    SUBCASE("adding to the vector increases its size") {
        v.push_back(1);

        CHECK(v.size() == 6);
        CHECK(v.capacity() >= 6);
    }
    SUBCASE("reserving increases just the capacity") {
        v.reserve(6);

        CHECK(v.size() == 5);
        CHECK(v.capacity() >= 6);
    }
}
\end{verbatim}

\section{Conditionary expressions}
\begin{verbatim}
<condition-expression> ? <expression-true> : <expression-false>;
// condition-expr must be (convertible to) bool
// <expr-true> and <expr-false> have to be of the same type 
\end{verbatim}

\chapter{Pointers \& References}

\section{Pointers}
\begin{verbatim}
int i{5};
int* p{&i}; // <type>* = pointer to type (different object type!), & = address-of operator
int** pp{&p}; // <type>** = pointer to pointer to type

int k{*p}; // * = dereference operator : restituisce reference, non oggetto!
q = nullptr; // null pointer
auto a = *q; // ERRORE!

struct S{
  int n;
  ...
  void f();
}
S ogg{...};
S* pt = &ogg;
pt->n;
pt->f();
// structure-dereference operator - equivale a (*<pointer>).<member>
\end{verbatim}
\subsection{Pass by pointer}
\begin{verbatim}
int conta(std::string* s){
    ...
    ... *s ...;
    ...
}
std::string ciao{"bu"};
conta(&ciao);
\end{verbatim}

\section{References}
\begin{verbatim}
int i{25};
int& ri{i}; // <type>& = reference to type (different type!)
int& rx; // ERRORE: dichiarazione sempre con inizializzazione!

int p{3};
ri{p}; // ERRORE: no rebinding (riassociazione)
\end{verbatim}
\subsection{Pass by reference}
\textbf{Parametri input:} consigliata per tipi non primitivi, altrimenti by value. \textbf{Output: } returnare valore oppure passare by non const reference . \textbf{I/O: } non-const ref. Occhio con il \texttt{const} !
\begin{verbatim}
void func(int& n){
    ...
    ++n;
    ...
}

void func(std::string const& ciao){
    // cannot modify ciao (see later)
}
\end{verbatim}
\subsection{Const ref}
\begin{verbatim}
std::string text{· · ·};
std::string& rtext{text}; // can read/modify text via rtext
std::string const& crtext{text}; // cannot modify, read-only!

std::string const text{· · ·};
std::string& rtext{text}; // ERROR: else could modify text via rtext
std::string const& crtext{text}; // ok, can only read text via crtext
\end{verbatim}
\subsection{Returning references}
\textbf{Mai} per variabili locali della funzione o in generale oggetti che non sopravvivono alla fine dell'enunciato (finisce scope: ERRORE)

\section{Enumeration}
Tipo distinto con costanti dette enumerators (underlying type = int di default - modificabile). Tutti i valori dell'underlying type sono validi per l'enumeration.
\begin{verbatim}
enum class Operator { Plus, Minus, Multiplies, Divides };
auto op{Operator::Plus}; // op is of type Operator (:: scope resolution operator)
// valore di default = precedente en. + 1 (valore del primo è 0)
enum class Operator { Plus = -2, Minus, Multiplies = 42, Divides }; 
// possono essere assegnati valori

Operator op{55};

enum class byte : unsigned char { }; // select different underlying type
auto i{static_cast<int>(Operator::Plus)}; // conversioni all'und. type DEVONO ESSERE ESPLICITE
\end{verbatim}
\subsection{Unscoped enum}
No need to use \texttt{Operator::}, conversion to the underlying type is implicit. (SCONSIGLIATO)
\begin{verbatim}
enum Operator { Plus, Minus, Multiplies, Divides }; // NB no class
\end{verbatim}

\section{Switch}
Trasferisce controllo a uno tra enunciati multipli secondo valore espressione
\begin{verbatim}
double compute(char op, double left, double right)
{
    double result;
    switch (op) { // condition = only integral (int, char, bool) or enumeration value
      case '+':
        result = left + right;
        break;
      · · ·
      case '/':
        result = (right != 0.) ? left / right : 0.;
        break; // if not inserted, control 'falls through' next instruction!
      default: // at most one (not necessarily at the end!) - handles all other cases
        result = 0.;
    }
    
    return result;
}
\end{verbatim}
In caso di \texttt{fall through} segnalazione del compilatore: se si intende silenziare warning aggiungere come attributo \boxed{\texttt{[[fallthrough]]}}.

\chapter{Data abstraction}
\section{Struct}
Default = \texttt{public} (can be omitted)
\begin{verbatim}
struct <newtype> {
    <type> <name>; // data member, instance variable
}; // occhio!

<newtype> ad{};
auto b = ad.<name>; // operatore di accesso ai membri
\end{verbatim}


\section{Class}
default = \texttt{private} (can be omitted)
\begin{verbatim}
class <newtype> {
  private:
      <type1> <name1>;
      ...
      <typeN> <nameN>;
  public:
    <newtype>(<type1> x, ..., <typeN> z) : <name1>{x}, ..., <nameN>{z} {}
      // CONSTRUCTOR : inizialization order is important!
    <newtype>(<type1> x) : <name1>{x}, ..., <nameN>{<default-valueN>} {}
      // multiple constructors!
    <newtype>(<type1> x, <type2> y) : <newtype>{x, <def-value2>, ..., <def-valueN>} {} // delegating
    <newtype>() : <name1>{<def-value1>}, ..., <nameN>{<def-valueN>} {}
    // DEFAULT CONSTRUCTOR (MUST be declared if there are other constr, otherwise automatically 
    // generated)
    void func(){
      ... <name1> ...;      
    } // member functions - methods (only f. with access to private memb.)
      // OVERLOADING is possible
    auto name1 const {
      return <name1>;    
    } // methods that don't modify object MUST BE DECLARED CONST
};

<newtype>& manipulate(<newtype>& a){
    ... a.name1 ... ; // no access to private
} // implemented as FREE FUNCTION - suggested (better for manteniance)


class Complex {
    double r_{0.};
    double i_{0.};
  public:
    Complex(double x, double y) : r_{x}, i_{y} {} 
    Complex() = default; // default implementation generated by compiler:
                         // (r_{0.} i_{0.}) - do not use {} ! 
};
\end{verbatim}
\subsection{Costruttori}
\begin{verbatim}
Complex(double x = 0., y = 0.) : r_{x}, i_{y} {} 
// default arguments: assumed if corresponding arg is absent (works for any function!)

explicit Complex(...) : ... {} // explicit constructor : prevents implicit conversion in
// initialization + impl. construction from list between braces: 
double norm(Complex consy& c) {...}
norm(2.); // ERROR!
Complex foo(...){
  ...
  return {a, b};
} // ERROR!
\end{verbatim}
\subsection{This}
\begin{verbatim}
struct T{
   void func(){
     ... this ...;   
   } // POINTER of type T* to object for which method called
   void fit() const{
     ... this ...; // in const methods, is of type T const*
   }
};
\end{verbatim}
\section{Overloading}
\begin{verbatim}
class <newtype>{
  ...
};

<newtype> operator@(<newtype> const& c, <newtype> const& d){ 
    ...
}
<newtype> ad{};
<newtype> bc{};

ad @ bc; // overloading di operatori (NB: l'associatività non può essere modificata)
\end{verbatim}
Tipicamente si implementa \texttt{operator@} in termini di \texttt{operator@=}. Quest'ultimo restituisce (di solito) \textbf{reference} all'oggetto su cui si opera. Metodo non const, ma passare secondo addendo come const ref:
\begin{verbatim}
class <newtype>{
   ...
  public:
   ...
   <newtype>& operator+=(<newtype> const& rhu){
     ...
     return *this;
   }
};
\end{verbatim}

\subsection{Classi nidificate}
Nested class \textbf{può accedere ai membri privati dell'outer c.} (no viceversa)

\begin{verbatim}
class Regression {
  · · ·
  public: 
  // can also be private, but in case cannot be named outside (e.g. one must use 'auto')
  class Result; // can be forward declared and later defined
  class Result { · · · }; // nested class definition
  Result fit() const { · · · }
};

Regression reg;
· · ·
Regression::Result result{ reg.fit() }; // or auto result{· · ·}
// necessario scope operator ::

class Tizio{
  ...
  class Caio; // can be declared inside and defined outside
};

class Tizio::Caio{ // don't forget scope operator
  using tiz = Tizio; // can use type alias for outer class
  tiz k_;
  ...
};
\end{verbatim}

\section{Assert}
Espressione booleana il cui soddisfacimento è verificato \textbf{al runtime}: se fallisce il programma muore. \textbf{Abbondare nel codice!}.
\begin{verbatim}
#include <cassert>
...
assert(<boolean-cond>);
\end{verbatim}
Per disabilitare: quando si compila da shell \boxed{\texttt{g++ -DNDEBUG ...}}
\\Possibile anche dalla Standard Library, per verificare condizione \textbf{AL COMPILE TIME}:
\begin{verbatim}
static_assert ( <bool-constexpr> , <message> );
// messaggio opzionale: DEVE ESSERE string literal
// (neanche una constexpr valutata al compile!!)

static_assert ( <bool-constexpr> );

// bool-constexpr deve essere costante bool o constexpr valutata al compile che
// viene convertita in bool
\end{verbatim}

\section{Eccezioni}
Sono oggetti. Utilizzare \textbf{solo in contesti specifici} dove non è possibile comunicare errore in altro modo.
\begin{verbatim}
struct E{};

auto function3() {
   ... // executed
   throw E{}; // sollevare (lanciare) eccezione
   ... // NOT executed : flow control is transferred
}

auto function2(){
   ... // executed
   function3();
   ... // not executed
}

auto function1(){
   try{
     ... // executed
     function2();
     ... // not executed
   } catch (E const& e) { // HANDLER
     ... e ... // use e
   }
}
\end{verbatim}
Si sollevano \textbf{by value} e si catchano \textbf{by reference (o const ref)}.
\\Eccezione si propaga fino a handler (\texttt{catch}) adatto (suitable, compatibile con il tipo di eccezione lanciata). Se non viene trovato programma viene \texttt{terminated}.
\paragraph{Catch multipli}
Viene scelto il primo che corrisponde a tipo eccezione sollevata: \textbf{l'ordine conta!}
\begin{verbatim}
auto read_from(std::filesystem::path const& p) {
  std::ifstream is(p);
  if (!is) {
    throw std::filesystem::filesystem_error{
    "read_from", p, std::make_error_code(std::errc::invalid_argument)
    };
  }
  · · ·
}

auto g() {
  try {
    read_from("/tmp/data");
    · · ·
  } catch (std::filesystem::filesystem_error const& e) {
    std::cerr << e.path1();
  } catch (std::exception const& e) {
    std::cerr << e.what();
  } catch (...) { // it's really three dots!!! (specific syntax)
    std::cerr << "unknown exception";
  }
}
\end{verbatim}

\subsection{Eccezioni nei costruttori} utili nel caso non sia possibile inizializzazione corretta (impossibile stabilire / soddisfare invariante di classe).
\\Possibile utilizzare un tipo di eccezione dalla Standard:
\begin{verbatim}
#include <stdexept>
. . .
class Rational {
  · · ·
  Rational(int num = 0, int den = 1) : n{num}, d{den} { // constructor
    if (d == 0) { 
      throw std::runtime_error{"denominator is zero"};
      // costruito con stringa o string literal
    }
    · · ·
  }
};

auto do_computation() {
  · · ·
  Rational{n,m} // m here happens to be 0
  · · ·
}

try {
  do_computation();
  · · ·
} catch (std::runtime_error const& e) {
  std::cerr << e.what() << '\n';
  // metodo what() per recuperare stringa
}
\end{verbatim}
\textbf{Nota:} eccezioni nella standard implementate tramite gerarchia polimorfica. E.g. classe derivata (indirettamente) da \texttt{std::runtime\_error} (a sua volta derivata dalla base \texttt{std::exception}) è \texttt{std::filesystem::filesystem\_error}, che si lancia in caso di problemi nella gestione dei \texttt{path}. Il catch permette di recuperare i percorsi in memoria coinvolti.


\section{Verificare tipo}
Funzioni con return type booleano. Vedi C++ Ref per specifici
\begin{verbatim}
#include <type_traits>

is_void(...);
...
is_floating_point(...);
...
is_pointer(...);
...
is_fundamental(...);
...
is_const(...);
...
is_signed(...); is_unsigned(...);
...
is_polymorphic(...);
...

\end{verbatim}


\section{Type aliases}
Nome alternativo per tipi esistenti. \textbf{NON} introduce nuovo tipo (nessun overloading per funzioni!)
\begin{verbatim}
using Length = double; // using = keyword!
typedef double Length; // equivalent, old alternative
\end{verbatim}
Aliases utilizzati dentro classi per dichiarare tipi 
\begin{verbatim}
class FitResult { · · · };

class Regression {
    · · ·
  public:
    using Result = FitResult;
    Result fit() const { · · · }
};

Regression::Result result{ reg.fit() }; // result is of type FitResult
\end{verbatim}

\section{Structured binding}
Dichiarare più variabili inizializzandole a valori di membri di una struct, secondo \textbf{ordine!}
\\Anche come const reference ai membri:
\begin{verbatim}
struct Point {
double x;
double y;
};
Point p{1.,2.};
auto [a, b] = p;
std::cout << a << ' ' << b; // print 1 2

// COME REFERENCE
Point p{1.,2.};
auto& [a, b] = p; // a is a ref to p.x, b is a ref to p.y
a = 3.;
b = 4.;
std::cout << p.x << ' ' << p.y; // print 3 4
\end{verbatim}

\chapter{Templates}
Modello di classe o funzione con tipi (e non solo) come parametri.
\\Concetto di base nell'approccio definito \textbf{generic programming}
\\\texttt{Template} è keyword.
\section{Class Template}
\textbf{NON} costituisce type di per sè!
\begin{verbatim}
// NB: < ... > sonp sintassi specifica! 

template<typename FP> // or, template<class FP>
class Complex {
   static_assert(std::is_floating_point_v<FP>); // (*)
   FP r;
   FP i;
  public:
   Complex(FP x = FP{}, FP y = FP{}) : r{x}, i{y} {}
   FP real() const { return r; }
   FP imag() const { return i; }
};

// (*) compile-time check + type introspection
// possibile definire tipi accettabili come parametri di template

Complex<double> d; // instantiation of a Complex<double> type
Complex<float> f;  // DIFFERENT TYPE than d
d + f;             // possible ERROR
\end{verbatim}
Per impostare valori di default dei parametri:
\begin{verbatim}
template<typename FP = <default type> >
\end{verbatim}
\subsection{Type aliases as templates}
Possibile ad esempio bloccare parametri di template:
\begin{verbatim}
// array of 3 T’s
template<class T> using Array3 = std::array<T, 3>;

Array3<double> a; // std::array<double, 3>
\end{verbatim}

\section{Function template}
\begin{verbatim}
template<typename FP> // or template<class FP>
auto norm2(Complex<FP> const& c) {
   return c.real() * c.real() + c.imag() * c.imag();
}
auto nf = norm2(f); // nf is of type float
auto nd = norm2(d); // nd is of type double
\end{verbatim}
Possibile anche
\begin{verbatim}
template<typename FP>
auto norm2(Complex<FP> const& c)
{ return c.real() * c.real() + c.imag() * c.imag(); }
\end{verbatim}
Per l'istanziazione del template devono essere noti \textbf{tutti gli argomenti}. Possono essere dedotti \textbf{dagli argomenti della chiamata della funzione}:
\begin{verbatim}
template<class F> F norm2(Complex<F> const& c) {· · ·}
Complex<float> f;
norm2(f); // no need to specify norm2<float>(f)
norm2<float>(f); // ok, be explicit
\end{verbatim}
L'importante è che nel caso si possano dedurre tutti quelli necessari!
\\Analogo per template di classe con la chiamata del constructor:
\begin{verbatim}
template<class FP> class Complex {· · ·};

Complex d; // error, cannot deduce FP
Complex<double> e; // ok
Complex f{1.}; // ok, Complex<double>
\end{verbatim}
Oltre ai tipi, possono esserci altri parametri. Se valori, devono essere noti \textbf{necessariamente al compile time} ed essere di tipo integral, enumeration, pointer, reference, \texttt{$std::nullptr_t$}, floating-point, literal (con specifiche caratteristiche, vd C++Ref)
\begin{verbatim}
template<class T, int N> struct array { · · · };
template<class T, T v> struct integral_constant { · · · };
// possibile anche mischiare type e non-type!
\end{verbatim}

\chapter{Standard Library}
\textit{Perché sprecare tempo a imparare quando l'ignoranza è immediata?} (Hobbes)

\section{Namespace}
Per partizionare spazio nomi e evitare conflitto tra identificatori. \textbf{Richiesto per progetto!} Si consiglia di mettere tutte le entità di un componente di software (classi, funzioni, etc.) dentro stesso namespace.
\begin{verbatim}
// in <vector>
namespace std {
template<class T> vector { · · · };
}
\end{verbatim}
Possono essere riaperti, anche in altri file (tranne il \texttt{namespace std} !) e nidificati.
\subsection{Namespace alias}
\begin{verbatim}
namespace ch = std::chrono;
auto t0 = ch::system_clock::now(); // std::chrono::system_clock::now()
// it's the same namespace, not a new one!
\end{verbatim}
\subsection{Using declaration + directive}
\begin{verbatim}
using std::string; // DECLARATION: rende visibile simbolo di un namespace           
string s; // possibile accedere a membro senza scope operator

using namespace std; // DIRECTIVE: visibili TUTTI i simboli
string s;
\end{verbatim}
\textbf{NOTA BENE} \texttt{using directive} sconsigliato: non si può chiudere e si rischia conflitto di identificatori, specie in scope globale e header file

\section{Iteratori}
Range sono semiaperti a dx e rappresentati da 2 iteratori:
\texttt{first} corrisponde al primo elemento, \texttt{last} all'estremo superiore $\notin$ range, ovvero la posizione di memoria immediatamente adiacente (successiva) all'ultimo elemento.
\\\textbf{Un range è vuoto} quanto \texttt{first == last}
\subsection{Operazioni su iteratori}
Sintatticamente analoghe a quelle sui pointer
\begin{verbatim}
std::vector<int> v {1,2,3};
auto it = v.begin();
auto its = v.end();
*its; // UNDEFINED BEHAVIOUR: no elemento del vettore!

// DEREFERENCE:
std::cout << *it; // print 1
*it = 4; // v is now {4,2,3}
// ACCESS TO MEMBERS:
struct Point {
double x;
double y;
};
...
std::vector<Point> vect {Point{1,2}, Point{3,4}};
auto its = vect.begin();
std::cout << (*itv).x; // print 1
std::cout << itv->x; // equivalent
// INCREMENTO (Spostamento di posizione):
++itv;
std::cout << itv->x; // print 3

// CONFRONTO
auto itn = vect.begin();
itv == itn; // verifica se puntano a stesso elemento
// NOTA: funziona anche per iteratori su diversi vettori con stesso tipo di elementi 
// (diverso tipo: ERRORE), ovviamente sempre falso
\end{verbatim}
\textbf{Nota:} possibile creare vettore di stringhe (implementate nella standard come container di caratteri) e accedere a metodi analoghi
\begin{verbatim}
std::vector<std::string> v {"hello", "world"};
auto itv = v.begin(); // itv points to the first string in the vector
auto its = itv->begin(); // its points to the first character
// of the first string ('h');
// a string is a container of characters
\end{verbatim}
\subsection{Gerarchia degli iteratori, operazioni aggiuntive}
In ordine decrescente di potere (operazioni supportate, versatilità) con operazioni \texttt{Lettura / Accesso / Scrittura / Iterazione / Confronto}
\begin{enumerate}
\item[1] RandomAccessIterator
\begin{verbatim}
=*p / -> [] / *p= / ++ -- + - += -= / == != < > <= >=
\end{verbatim}
\item[2] BidirectionalIterator
\begin{verbatim}
=*p / -> / *p= / ++ -- / == !=
\end{verbatim}
\item[3] ForwardIterator
\begin{verbatim}
=*p / -> / *p= / ++ / == !=
\end{verbatim}
\item[4a] OutputIterator
\begin{verbatim}
=*p / -> / / ++ / == !=
\end{verbatim}
\item[4b] InputIterator
\begin{verbatim}
 / / *p= / ++ /
\end{verbatim}
\end{enumerate}


\section{Vector}
Contenitore dinamico: dimensione varia al runtime, layout di memoria contiguo, \textbf{da utilizzare come default}
\begin{verbatim}
#include <vector>

std::vector<int> a; // empty vector of ints
std::vector<int> b{2}; // one element, initialized to 2
std::vector<int> c(2); // two elements (!), value-initialized (to default, 0 for int)
std::vector<int> d{2,1}; // two elements, initialized to 2 and 1
std::vector<int> e(2,1); // two elements, both initialized to 1

auto f = b; // crea copia del vettore
f == b; // operatore supportato: verifica corrispondano gli elementi (qui true)
\end{verbatim}
Occhio alla sintassi nell'inizializzazione: se vi è nel programma un costruttore che accetta \begin{verbatim} std::initializer_list
\end{verbatim} (class template corrispondente ad array di oggetti \texttt{const}) verrebbe chiamato inizializzando con lista di oggetti tra \texttt{} !
\subsection{Operazioni / metodi}
\begin{verbatim}
std::vector<int> vec{};

vec.size(); // dimensione (NB è unsigned int!)
            // per farci operazioni bene forzare prima conversione implicita in int
vec.empty(); // verifica se è vuoto o no (booleano)
// NB diverso da vettore con tutti valori di default!
vec[<n>]; // accesso all' n-esimo elemento (int)
vec[0]; // primo elemento: IL CONTEGGIO PARTE DA 0
vec[vec.size()]; // ERRORE: l'ultimo elemento corrisponde a vec.size()-1

vec.push_back(5); // aggiunge elemento alla fine

vec.begin(); // iteratore corrispondente al first
vec.end(); // iteratore corr. al last 
// (vd. dopo)

vec.insert( <iterator>, <value>); // inserisce elemento (inizializzato a <value>) nella posizione 
                                  // PRECEDENTE a quella indicata da <iterator>
vec.insert( vec.end(), <value> ); // analogo a push_back
vec.insert( <it>, <count>, <value>); // inserisce <count> elementi con valore <value> prima di <it>
                                     // NOTA: <count> è size_type (unsigned)
vec.insert( <it>, <it_first>, <it_last>); // ins. RANGE di elementi da <it_first> a <it_last> 
                                          // prima di <it> NOTA BENE: first e last non possono 
                                          // essere iteratori su vec!
std::initializer_list<int> l{1,2,3,4};
vec.insert( <it>, l); // inserisce elementi della lista nel vettore
                      // NB tipo el. deve essere lo stesso di vec

vec.erase( <iterator> ); // rimuove elemento *<iterator>
vec.erase( <it_first>, <it_last> ); // rimuove range: <it_first> a primo elemento, <it_last> a
                                    // estremo superiore
                                    
vec.capacity(); // capacità complessiva allocata per il vettore
                // (non solo slot inizializzati)
vec.reserve(size_type <newcap>); // se <newcap> > capacità, rialloca spazio
                                 // NB nel caso INVALIDA tutti ITERATORI!
                                 // (non modifica size)
vec.resize(size_type <count>); // se <count> = size, nulla
                               // se <count> < size, colma differenza
                               // inserendo di default elementi 
                               // se <count> > size, riduce ai primi
                               // <count> elementi
vec.resize(size_type <count>, const int <value>); // analogo, ma nel secondo
                                                  // caso colma inserisce <value>

\end{verbatim}
\textbf{Nota bene: }dopo \texttt{erase}, iteratori a elementi rimossi non sono più validi: utilizzo o operazioni su di essi danno \texttt{undefined behaviour}. Nel caso dei \texttt{vector}, \textbf{anche iteratori a elementi successivi, fino a \texttt{end}} sono invalidati.
\\Analogamente nel caso di riallocazione di memoria, \textbf{tutti gli iteratori} che puntano al vettore sono invalidati!

\section{Array}
Dimensione stabilita al compile, layout contiguo
\begin{verbatim}
#include <array>

std::array< <Type>, <Size> > arr{}; // Size deve essere specificato al compile!
std::array<int,2> c{}; // 2 ints, value-initialized (0 for int)
std::array<int,2> d{1}; // 2 ints, initialized to 1 and 0 
                        // (following order!)
                        
auto e = d; // copia tutto l'array, tipo di e: std::array<int,2>

d.at( <n> ); // n è size_type: restituisce n-esimo elemento
             // e SOLLEVA std::out_of_range se n >= size
d[ <n> ]: // n-esimo elemento, senza bound checking

d.begin(); d.end(); d.empty(); // analogo a vettori, per empty sempre falso se N > 0
\end{verbatim}

\section{Algoritmi}
\textbf{Si consiglia utilizzo:} efficienti,
Funzioni generiche che operano su range, implementati come template ('duale' dei container nella STL). Necessario
\begin{verbatim}
#include <algorithm>
\end{verbatim}
\textbf{Nota:} algoritmi che trovano / modificano specifici elementi restituiscono iteratori, non valori!
\\Varie categorie:
\subsection{Non modifying}
\begin{verbatim}
all_of
any_of
none_of
// verifica se predicato vero per tutti, almeno uno o nessun elemento di range

for_each
for_each_n
// applica funzione a tutti / primi n elem di range

count
count_if
// numero elem. che soddisfano criteri

mismatch
// prima posizione in cui differiscono 2 range

find
find_if
find_if_not
// primo 

find_end
// ultima occorrenza di determinata sequenza in un range

find_first_of
// occorrenza qualsiasi di data sequenza in range

adjacent_find
// primi due elementi adiacenti che soddisfano predicato (default: sono uguali)

search
// cerca prima occorrenza di dato range in altro range

search_n
// cerca n copie consecutive di elemento in range
\end{verbatim}
\subsection{Modifying}
\begin{verbatim}
copy
copy_if
// copies a range of elements to a new location

copy_n
// copies a number of elements to a new location

copy_backward
// copies a range of elements in backwards order

move
// moves a range of elements to a new location

move_backward
// moves a range of elements to a new location in backwards order

fill
// copy-assigns the given value to every element in a range

fill_n
// copy-assigns the given value to N elements in a range

transform
// applies a function to a range of elements, storing results in a destination range

generate
// assigns the results of successive function calls to every element 
// in a range

generate_n
// assigns the results of successive function calls to N elements 
// in a range

remove
remove_if
// removes elements satisfying specific criteria

remove_copy
remove_copy_if 
// copies a range of elements omitting those that satisfy specific criteria

replace
replace_if
// replaces all values satisfying specific criteria with another value

replace_copy
replace_copy_if
//copies a range, replacing elements satisfying 
//specific criteria with another value

swap
// swaps the values of two objects

swap_ranges
// swaps two ranges of elements

iter_swap
// swaps the elements pointed to by two iterators

reverse
// reverses the order of elements in a range

reverse_copy
// creates a copy of a range that is reversed

rotate
// rotates the order of elements in a range

rotate_copy
// copies and rotate a range of elements

shift_left
shift_right
// shifts elements in a range

random_shuffle
shuffle
// randomly re-orders elements in a range

sample
// selects n random elements from a sequence

unique
// removes consecutive duplicate elements in a range

unique_copy
// creates a copy of some range of elements that contains 
// no consecutive duplicates
\end{verbatim}
\subsection{Partitioning}
\begin{verbatim}
is_partitioned
// determines if the range is partitioned by the given predicate

partition
// divides a range of elements into two groups

partition_copy
// copies a range dividing the elements into two groups

stable_partition
// divides elements into two groups while preserving their relative order

partition_point
// locates the partition point of a partitioned range
\end{verbatim}
\subsection{Sorting}
\begin{verbatim}

is_sorted
// checks whether a range is sorted into ascending order

is_sorted_until
// finds the largest sorted subrange

sort
// sorts a range into ascending order

partial_sort
// sorts the first N elements of a range

partial_sort_copy
// copies and partially sorts a range of elements

stable_sort
// sorts a range of elements while preserving order between equal elements

nth_element
// partially sorts the given range making sure that it is partitioned by 
// the given element
\end{verbatim}
\boxed{\texttt{Per range ordinati:}}
\subsection{Binary search}
\begin{verbatim}
lower_bound
// returns an iterator to the first element not less than the given value

upper_bound
// returns an iterator to the first element greater than a certain value

binary_search
// determines if an element exists in a partially-ordered range

equal_range
// returns range of elements matching a specific key
\end{verbatim}
\subsection{Set}
\begin{verbatim}
includes
// returns true if one sequence is a subsequence of another

set_difference
// computes the difference between two sets

set_intersection
// computes the intersection of two sets

set_symmetric_difference
// computes the symmetric difference between two sets

set_union
// computes the union of two sets
\end{verbatim}
\subsection{Altro}
\begin{verbatim}
merge
// merges two sorted ranges

inplace_merge
// merges two ordered ranges in-place: second one attached to the end of
// the first
\end{verbatim}
\boxed{\texttt{Senza requisiti di ordine:}}
\subsection{Min / Max}
\begin{verbatim}
max
// returns the greater of the given values

max_element
// returns the largest element in a range

min
// returns the smaller of the given values

min_element
// returns the smallest element in a range

minmax
// returns the smaller and larger of two elements

minmax_element
// returns the smallest and the largest elements in a range

clamp
// clamps a value between a pair of boundary values: returns upper or lower
// bound if out of the interval, otherwise returns value
\end{verbatim}
\subsection{Comparison}
\begin{verbatim}
equal
// determines if two sets of elements are the same

lexicographical_compare
// returns true if one range is lexicographically less than another

lexicographical_compare_three_way 
// compares two ranges using three-way comparison: < = >
\end{verbatim}
\subsection{Numeric}

\paragraph{Parallelismo} per esecuzione in parallelo (con più cores processore)
\begin{verbatim}
#include <execution>
...
std::<alg1-name>( std::execution::par, <alg1 arguments> );
std::<alg2-name>( std::execution::par, <alg2 arguments> );
\end{verbatim}

\section{Concepts}
Insieme di requisiti che un tipo deve soddisfare al compile, limitano tipi utilizzabili come argomenti di template; introdotti esplicitamente da C++20.
\begin{verbatim}
template<class T>
concept Incrementable = requires(T t) { ++t; };

template<Incrementable T>
auto advance(T& t) { ++t; }

int i {42};
advance(i); // ok, int is a model of Incrementable

struct S {};
S s;
advance(s); // error, S is not a model of Incrementable
\end{verbatim}

\section{Function objects}
Possibile costruire oggetti funzione tramite overloading dell'operatore di chiamata, da passare come argomenti ad algoritmi.
\begin{verbatim}
struct LessThan42 {
  auto operator()(int n) const
  {
    return n < 42;
  }
};

LessThan42 lt42{};
// or: auto lt42 = LessThan42{};
auto b = lt42(32); // true

std::vector v {61,32,51};
auto it = std::find_if(
    v.begin(), v.end(),
    lt42 // or directly: LessThan42{}
); // *it == 32
\end{verbatim}
Possibile passare parametri come variabili private di classe (o pubbliche di struct)
\begin{verbatim}
class LessThan {
   int m_;
  public:
   explicit LessThan(int m) : m_{m} {}
   auto operator()(int n) const {
     return n < m_;
   }
};

LessThan lt42 {42};
auto b1 = lt42(32); // true
// or: auto b1 = LessThan{42}(32);
\end{verbatim}

\section{Lambda expression}
Metodo rapido per creare function object senza nome (usa e getta), definiti \texttt{closures}.
\begin{verbatim}
[ <captures> ]( <params> ) <specs requires> { <body> }
\end{verbatim}
\texttt{<captures>} è il prefisso: possibili sintassi:
\begin{verbatim}
std::find_if(..., [](int n) {
                     return n < 42;
                  }
); // non cattura nulla

... [=] ... // cattura tutte le variabili necessarie (tra quelle dichiarate 
            // e inizializzate prima) BY VALUE

auto m = ...;     
... [m] ... // cattura m (in genere tutte le var. indicate) BY VALUE

... [m = <value> ] ... // inizializza e cattura m BY VALUE 

... [&] ... // cattura tutte le variabili necessarie (tra quelle dichiarate 
            // e inizializzate prima) BY REFERENCE
            
auto m = ...;     
... [&m] ... // cattura m (in genere tutte le var. indicate) BY REFERENCE

... [=, &k] ... // cattura tutto il necessario BY VALUE e k BY REFERENCE

// NB se il primo termine nella cattura è = (&), eventuali successivi non possono essere
// catturati by value (by ref) !

... [&, k] ... // tutto BY REF e k BY VALUE
\end{verbatim}
\textbf{Nota 1:} le variabili globali sono disponibili senza cattura!
\\\textbf{Note 2:} possibile catturare \texttt{current object}: (possibili \texttt{[\&, ..., this, ...]} , \texttt{[=, ..., this, ...]} (da C++20), \texttt{[=, ..., *this, ...]} (da C++17))
\\\textbf{Nota 3:} catturare solo variabili che si utilizzano (altrimenti viene comunque allocato spazio dal costruttore)
\\~\\\texttt{<params>} sono i parametri dell'operatore () sovraccaricato e implementato come template (parametri = tipi argomenti), che vengono passati alla lambda nell'esecuzione dell'algoritmo.
\\Ogni lambda crea implicitamente una classe differente con metodo overloaded e variabili catturate come private.
\\Per \texttt{<specs required>} le sintassi:
\begin{verbatim}
[...](...) {...} // nulla: operator() implementato come const: 
                 // non può modificare variabili catturate (!)

[...](...) mutable {...} // può modificare variabili catturate by value
// NOTA BENE: catturate by reference sono modificabili DI DEFAULT,
// impossibile catturare by const& !

[...](...) -> <type> {...} // specifica return type (default: auto - 'generic lambda')

[...](...) mutable -> <type> {...} // ordine se presenti entrambi
\end{verbatim}

\subsection{Std::function}
Function wrapper polimorfico che può prendere qualsiasi entità chiamabile con determinata segnatura (funzioni, metodi ma anche oggetti funzione).
\begin{verbatim}
#include <functional>
...
using Function = std::function<int(int,int)>; // signature
\end{verbatim}

\chapter{Compilation model}
One Definition Rule: ogni entità può essere definita una volta sola per translation unit! Class, templates e funzioni e variabili \textit{inline} possono essere definite in più TU \textbf{se le definizioni sono identiche} (token-by-token nel source code)
\\\textbf{Nota bene:} violazioni ODR non sempre diagnosticate da compiler ma possono causare malfunzionamenti eseguibile.
\\\paragraph{Strutturazione componente software}
\begin{itemize}
\item \textit{Header file / file di intestazione, interfaccia} (\texttt{.hpp}) \textbf{dichiarazioni} delle free functions, \textbf{definizioni} di classi con \textbf{dichiarazioni} di metodi e \textbf{definizioni }template
\item \textit{Source file / file di implementazione} (\texttt{.cpp}) \textbf{definizioni} di free functions e metodi e di qualsiasi altra entità necessaria per l'implementazione 
\item \textit{File di test} (\texttt{.cpp})
\end{itemize}

\subsection{Header file}
L'header è incluso tramite \texttt{\#include "name.hpp"} nel file di implementazione e in tutti gli altri file di progetto che abbiano necessità di accedere a funzioni e classi ivi presenti.
\\Ogni funzione definita in un header file deve essere \texttt{inline}:
\begin{itemize}
\item per le free functions deve essere esplicitato
\item per i metodi delle classi e i template è implicito
\end{itemize}
Metodi possono essere \textbf{dichiarati} nella classe e \textbf{definiti} nel source file, utilizzando l'operatore di scope.
\\\textbf{Se il metodo è definito fuori dalla definizione della classe ma è in un header file, va dichiarato \texttt{inline} !}
\begin{verbatim}
// in statistics.hpp

class Sample{
   ...
  public:
   void add(double); // declaration
   ...
};
...
inline void Sample::add(double d){
  ... // definition
}
\end{verbatim}

\subsection{Include guards}
Per evitare che un header file venga incluso più volte nella stessa translation unit (e.g. se viene incluso anche un altro h.f. che a sua volta include il primo).
\begin{verbatim}
// at the beginning of EVERY header file!

// e.g. need to include result.hpp and sample.hpp with the latter 
// including the former

// file result.hpp
#ifndef RESULT_HPP
#define RESULT_HPP
· · ·
// classes, functions, templates, ...
· · ·
#endif
\end{verbatim}

\section{Build systems (CMAKE)}
File di configurazione \texttt{CMakeLists.txt}. Esempio dal mock project del prof (commenti secondo sintassi)
\begin{verbatim}
cmake_minimum_required(VERSION 3.16)
project(mandelbrot_sfml VERSION 0.1.0)
# nome e versione progetto

# abilita il supporto per i test, tra cui l'opzione BUILD_TESTING usata sotto
include(CTest)

set(CMAKE_EXPORT_COMPILE_COMMANDS ON)

# richiedi l'uso di C++17, senza estensioni non-standard offerte dal compilatore usato 
set(CMAKE_CXX_STANDARD 17)
set(CMAKE_CXX_STANDARD_REQUIRED ON)
set(CMAKE_CXX_EXTENSIONS ON)

# abilita warning
string(APPEND CMAKE_CXX_FLAGS
      " -Wall -Wextra -Wpedantic -Wconversion -Wsign-conversion -Wcast-qual -Wformat=2"
      " -Wundef -Wshadow -Wcast-align -Wunused -Wnull-dereference"
      " -Wdouble-promotion -Wimplicit-fallthrough -Wextra-semi -Woverloaded-virtual"
      " -Wnon-virtual-dtor -Wold-style-cast")

# abilita l'address sanitizer e l'undefined-behaviour sanitizer in debug mode
string(APPEND CMAKE_CXX_FLAGS_DEBUG " -fsanitize=address,undefined -fno-omit-frame-pointer")
string(APPEND CMAKE_EXE_LINKER_FLAGS_DEBUG " -fsanitize=address,undefined -fno-omit-frame-pointer")

# richiedi il componente graphics della libreria SFML, versione 2.5
# le dipendenze vengono identificate automaticamente
find_package(SFML 2.5 COMPONENTS graphics REQUIRED)

add_executable(mandelbrot_sfml main.cpp)
target_link_libraries(mandelbrot_sfml PRIVATE sfml-graphics)

# se il testing e' abilitato...
# per disabilitare il testing, passare -DBUILD_TESTING=OFF a cmake durante la fase di configurazione
if (BUILD_TESTING)

  # aggiungi l'eseguibile all.t
  add_executable(all.t all.t.cpp complex.t.cpp)
  # aggiungi l'eseguibile all.t alla lista dei test
  add_test(NAME all.t COMMAND all.t)

endif()
\end{verbatim}

\paragraph{Comandi da shell}
\begin{itemize}
\item \textbf{Installazione (con SFML)} \boxed{\texttt{sudo apt install cmake libsfml-dev}}

\item \textbf{Configurazione area di compilazione} all'interno di cartella con file progetto eseguire \boxed{\texttt{ cmake -S . -B <newdir>}} : viene creata sottocartella \texttt{<newdir>} (nome a scelta, di solito \texttt{build}) dove avverrà compilazione. Nota: se il file \texttt{CMakeLists.txt} non viene modificato, non è necessario eseguire questo passaggio più di una volta.

\item \textbf{Conf. in debug} \boxed{\texttt{cmake -S . -B <newdir>/debug -DCMAKE\_BUILD\_TYPE=Debug}}

\item \textbf{Conf. in release mode} \boxed{\texttt{ cmake -S . -B <newdir>/debug -DCMAKE\_BUILD\_TYPE=Release}} 

\item \textbf{Compilazione} \boxed{\texttt{cmake --build <newdir>}} 
\\Se fallisce \textbf{basta} modificare file sorgente .hpp e .cpp (senza riconfigurazione)

\item \textbf{Compilare con test} \boxed{\texttt{cmake --build <newdir> --target test}}
\end{itemize}

Possibile, nel caso si aggiunga il testing, eseguire direttamente l'eseguibile dei test (vd. file \texttt{.txt})

\chapter{Explicit Memory Management}
Costruire oggetti sullo \texttt{heap}. Gestione esplicita: costruzione implica responsabilità su gestione e \textbf{distruzione}! (Sullo \texttt{stack} è implicita, automatica)
\begin{verbatim}
new []; // creates array: allocates memory + runs constructor
delete []; // delete array: run destructor + deallocate memory

int p* = new int{5}; // new restituisce pointer
delete p; // chiamato sul pointer
// p non modificato: UNICA possibile azione è riassegnazione
p = ...;

int* q = nullptr;
delete q; // well defined: does nothing

void func(){
   int* pt = new int{45};
   ... // no delete
} // MEMORY LEAK : pt's scope ended (implicitly destructed)

// DOUBLE DELETE : trying to call destructor more than one time on same
// object! 
\end{verbatim}
\textbf{Nota bene:} pointer è unico riferimento a oggetto creato sullo \textit{heap}: se il suo scope finisce prima della distruzione dell'oggetto creato questo è perduto!
\begin{verbatim}
Sample* create()
{
  auto rc = new Sample{};
  rc->add(· · ·);
  · · ·
  return rc;
}

auto use()
{
  auto ru = create();
  ru->stats();
  · · ·
  delete ru;
} // ok!
\end{verbatim}
Sconsigliato utilizzo raw pointer in caso di eccezioni: se porzione di codice con \texttt{delete} non viene eseguita, si rischia leak!

\section{Array nativi}
Sequenza contigua di oggetti in memoria. Usare molta precauzione nell'utilizzo!
\begin{verbatim}
int a[3] = {123, 456, 789}; // int[3], the size must be a constant
                            // and can be deduced from the initializer
++a[0];
a[3]; // undefined behavior (from 0 to size-1)

// "arrays decay to pointers at the slightest provocation" (!!!)
auto b = a; // int*, size information lost
            // restituisce pointer al primo elemento !
            // anche quando passato come parametro funzione!
            
++b; // increase by sizeof(int) - incrementabile
assert(b == &a[1]); // true

*b = 654; // dereferenziabile
b += 2; // increase by 2 * sizeof(int)
if (b == a + 3) { ... } // ok, but not more than this
\end{verbatim}
Allocazione dinamica array: quando dimensione nota solo al runtime.
\begin{verbatim}
int*p = new int[3] {14, 56, 144};
...
delete [] p; // sintassi specifica: altrimenti per decadimento pointer
             // non elimina array!
\end{verbatim}
Difficoltà gestione: per passare correttamente a funzione necessario
\begin{itemize}
\item Passare separatamente anche dimensione
\item Da C++20 utilizzare \texttt{std::span} (header \texttt{<span>}). \'E template di classe che contiene tipo elementi e estensione (funziona con qualsiasi sequenza di oggetti contigua in memoria)
\end{itemize}
Cmq \textbf{sconsigliato}.

\section{Null Terminated Byte Strings (NTBS)}
Array di caratteri non nulli seguiti dal carattere nullo (\texttt{char{0}} oppure \texttt{0}), noto anche come \textit{C-Strings}. \'E di tipo \texttt{char[]} / \texttt{char*} se modificabili o \texttt{char const[]} / \texttt{char const*} se non. 
\\Per la lunghezza \texttt{std::strlen} (legge array fino a trovare il null carachter)
\\Il tipo degli \texttt{string literals} è \texttt{char const[N]} con \texttt{N} costante. 
\\\'E possibile inizializzare una \texttt{std::string} da una NTBS; inoltre tramite il metodo \texttt{c\_str} si può ottenere la rappresentazione NTBS da una stringa. Returna un pointer all'array (\texttt{char const*}), che viene invalidato se si chiamano metodi \texttt{non const} sulla stringa di partenza. \textbf{Permette di estrarre i singoli caratteri}.
\begin{verbatim}
std::string stan{"Domani"};
char const* pt2 = stan.c_str();
std::cout << pt2[0]; // prints D
\end{verbatim}

\section{Main}
Può avere due forme
\begin{verbatim}
int main() {
   ...
   return 0; // successo (implicito)
             // qualsiasi altro valore = failure
}

int main(int argc, char* argv[]) {...}
\end{verbatim}
\texttt{argc} corrisponde al numero di argomenti della riga di comando, \texttt{argv[]} è un array di C-stringhe (\texttt{char*} in quanto C-stringhe sono a loro volta array di caratteri) che rappresentano gli argomenti.
\begin{itemize}
\item \texttt{argv[0]} è il nome del programma (di solito)
\item \texttt{argv[argc]} è \texttt{nullptr}
\end{itemize}
Esistono librerie per gestire e interpretare command line.

\section{Allocazione stile C}
Strumenti ereditati da C, \textbf{SCONSIGLIATISSIMI}: gestiscono memoria non inizializzata!
\begin{verbatim}
#include <cstdlib>
...
void* malloc(size_t s); // alloca s byte
void* calloc(size_t n, size_t s); // alloca n volte s byte inizializzati a 0

void free(void* p); // libera spazio allocato in precedenza
void* realloc(void* p, size_t s); // cambia dimensione array puntato da p,
                                  // oppure crea copia allocando s byte e
                                  // poi libera p
\end{verbatim}
\texttt{size\_t} è tipo standard per dimensioni (unsigned int)

\section{Considerazioni globali su allocazione dinamica}
\textbf{Quando non necessario, non utilizzarla (usare stack)!}
\\Maggiore responsabilità, utilizzo di tempo e spazio in memoria: oltre a quanto visto necessario allocare su stack per pointer + per conformazione \textit{heap} ('groviera') tempo non determinabile mappatura disponibilità di memoria da parte degli allocatori.
\\Dove sono necessarie \textbf{garanzie sul tempo di esecuzione} è proibito allocare dinamicamente!
\\Per Address Sanitizer (ASan) vedi istruzioni compilazione, \textbf{NB} non trova sempre tutti i problemi di memoria!

\section{Destructor}
\begin{itemize}
\item \textbf{Ordine di distruzione è opposto a quello di costruzione / definizione} (da tenere a mente quando si ordinano variabili private di classe!)
\item Sotto-oggetti distrutti ricorsivamente
\end{itemize}
\begin{verbatim}
{
  S s{· · ·};
  · · ·
  T t{s};
  · · ·
} // t is destroyed first, then s
\end{verbatim}
Una classe può avere \textbf{un solo distruttore}
\begin{verbatim}
class Array {
    int* m_data{};
    int m_n{};
   public:
    Array(int n) : m_data{new int[n]}, m_n{n} {} 
    //non è possibile sostituire m_data{new int[m_n]}: 
    //conta l'ordine di costruzione per il compiler 
    //stabilito dalla private e non dal constructor
    ~Array() {  // destructor, declared as ~ClassName()
        delete[] m_data;
    }
    int& operator[](int i){
        assert(i >= 0);
        assert(i < m_n);
        return m_data[i];
    }
};
\end{verbatim}
\textbf{RAII} = Resource Acquisition Is Initialization

\section{Controlling copying}
\begin{verbatim}
DynamicArray original{· · ·};

// COPY CONSTRUCTOR: (new object created as copy)
auto copy{original}; // auto copy = original

DynamicArray other{...};
// COPY ASSIGNMENT OPERATOR: (change value of existing obj as copy)
other = original;
\end{verbatim}
Per gestire operazioni in caso di risorse da gestire (e.g. allocate dinamicamente):
\begin{verbatim}
class DynamicArray {
   ...
  public:
   ...
   DynamicArray(DynamicArray const& other) : ... {...} // copy constructor
                // takes object of same class by const reference
   DynamicArray& operator=(DynamicArray const& other) {
                // copy assignment operator: takes const ref, returns ref
                // (*this modified)
     if (this != &other) { // checks for auto-assignment
       ...
     }
     return *this;
   }
   ...
};
\end{verbatim}

\section{Smart pointer}
Si comporta come pointer ma gestisce il lifetime dell'oggetto cui punta. Ogni volta che si alloca dinamicamente passare raw pointer a smart p. \textbf{il prima possibile!}
\\Esempio di implementazione:
\begin{verbatim}
template<typename Pointee>
class SmartPointer {
   Pointee* m_p;
  public:
   explicit SmartPointer(Pointee* p): m_p{p} {}
   ~SmartPointer() { delete m_p; }
   Pointee* operator->() { return m_p; }
   Pointee& operator*() { return *m_p; }
};

class Sample { · · · };

{
  SmartPointer<Sample> sp{new Sample{}};
  sp->add(· · ·);
  (*sp).stats();
} // sp destroyed at the end of the scope!
\end{verbatim}
Possibile anche (anzi) utilizzare smart pointers forniti dalla standard tramite
\begin{verbatim}
#include <memory>
\end{verbatim}
\begin{itemize}
\item \textbf{\texttt{std::unique\_ptr<T>}}
\end{itemize}
\begin{verbatim}
class Sample { · · · };

void take(std::unique_ptr<Sample> q); // by value

std::unique_ptr<Sample> p{new Sample{}}; // explicit new
auto p = std::make_unique<Sample>(); // better (*)
auto r = p; // error, non-copyable
take(p); // error, non-copyable
auto r = std::move(p); // ok, movable
take(std::move(r)); // ownership is moved (no copy!)
\end{verbatim}
\texttt{(*)} costruisce un nuovo oggetto Sample e restituisce valore puntatore ad esso; argomenti passati a \texttt{make\_unique} vengono inoltrati al costruttore di Sample!
\\\textbf{Consigliato} salvo quando necessario:
\begin{itemize}

\item \textbf{\texttt{std::shared\_ptr<T>}} Permette shared ownership (contando anche reference). Per condividere own. con altri \texttt{shared\_ptr} necessario copy construction o copy assignment: se si fa invece tramite raw pointer sottostante si ha UB!
\begin{verbatim}
void take(std::shared_ptr<Sample> q); // by value

std::shared_ptr<Sample> p{new Sample{}}; // explicit new
auto p = std::make_shared<Sample>(); // better (*)
auto s = p; // ok, copyable
take(p); // ok, copyable
auto s = std::move(p); // ok, movable
take(std::move(s)); // ok, movable
\end{verbatim}
\end{itemize}
\textbf{Nota 1: }sempre possibile passare da \texttt{unique} a \texttt{shared}, ma non viceversa!
\\\textbf{Nota 2:} per accedere al raw pointer:
\begin{verbatim}
<smart>_ptr<T>::get(); // any kind of smart ptr, returns non-owning T*
unique_ptr<T>::release(); // returns owning T* : must be explicitly
                          // managed!!
\end{verbatim}
\textbf{Nota 3:} array sono supportati
\paragraph{Passare a funzioni} passare smart ptr solo se funzione necessita di utilizzare lo smart pt stesso! Passando \texttt{unique} by value possibile trasferire ownership, \texttt{shared} by value per 'mantenere in vita' risorsa puntata, by const reference per chiamare metodi / utilizzare il valore dello s.p. stesso.
\paragraph{Returnare} se funzione passa al chiamante qualcosa allocato dinamicamente, returnare smart ptr!

\subsection{Disabilitare operazioni di copia}
Per sopprimere le operazioni di copia si segnano il copy constructor e il copy assignment operator con \texttt{= delete}:
\begin{verbatim}
template<typename Pointee>
class UniquePtr {
   Pointee* m_p;
  public:
   explicit UniquePtr(Pointee* p): m_p{p} {}
   ~UniquePtr() { delete m_p; }
   UniquePtr(UniquePtr const&) = delete;
   UniquePtr& operator=(UniquePtr const&) = delete;
   Pointee* operator->() { return m_p; }
   Pointee& operator*() { return *m_p; }
};
\end{verbatim}
\textbf{Nota:} il copy constructor, anche con \texttt{= delete} resta un costruttore, dunque impedisce lo stesso la generazione del default constructor!
\\In genere \texttt{= delete} può applicarsi a qualsiasi funzione, a patto che sia specificato nella \textbf{prima dichiarazione} della funzione in una translation unit. L'utilizzo di una funzione \textit{deleted} causa errore di compilazione

\section{Special Member Functions}
Per ogni classe (in aggiunta al costruttore di default). Si seguono:
\begin{itemize}
\item \textbf{RULE OF ZERO} se non necessario, non definirne nessuna (generate automaticamente dal compilatore, comportamento dipende da quello dei data members)
\item \textbf{RULE OF FIVE} se si definisce almeno una, meglio definirle tutte: nel caso possibile anche usare \texttt{= default} e \texttt{= delete}
\end{itemize}
\begin{verbatim}
class MyClass {
   MyClass(MyClass const&); // copy constructor
   MyClass& operator=(MyClass const&); // copy assignment
   MyClass(MyClass&&); // move constructor
   MyClass& operator=(MyClass&&); // move assignment
   ~MyClass(); // destructor
};
\end{verbatim}

\section{Tassonomia Container della STL}
\begin{description}
\item[Sequence] client stabilisce dove posizionare elementi (\texttt{array, deque, forward\_list, list, vector})
\item[Associative] decide il container:
\begin{itemize}
\item \textbf{Ordered} ordine sulla base di \textit{key} (funzione)
\\coppie key-valore: \texttt{map, multimap} // valori ordinati: \texttt{set, multiset}
\item \textbf{Unordered} posizione determinata da \textit{hash} della \textit{key} dell'elemento, ovvero dal valore numerico di una funzione che prende in input la \textit{key}
\\\texttt{unordered\_map, unordered\_multimap, unordered\_set, unordered\_multiset}
\end{itemize}
\end{description}

\subsection{Std::list}
Dimensione dinamica, layout di memoria \textbf{non contiguo}, stabilità degli iteratori \textbf{[aggiunta, rimozione e spostamento elementi non invalida iteratori o reference - solo con delete!]}.
\\Implementata come double-linked list: ogni nodo comprende elemento, puntatore al precedente ed al successivo. Consente dunque iterazione bidirezionale, a differenza di \texttt{std::forward\_list} (solo in avanti) al prezzo di efficienza di spazio.
\begin{verbatim}
#include <list>
...
std::list<int> lista = {7,6,8,9};

lista.front(); // primo elemento (TIPO int)
lista.back(); // ultimo el (TIPO int)
lista.begin(); // consueto (TIPO int*)
lista.end(); // consueto (TIPO int*)
lista.empty();
lista.size();
lista.max_size(); // limite per ragioni di sistema o implementazione

lista.erase(<iterator>);
lista.push_back(...);
lista.push_front(...); // inserisce all'inizio
                       // NON INVALIDA ITERATORI
lista.insert(<iterator>, <value>);

lista.sort(); // 6,7,8,9
std::list<int> lista2 = {10,11,13,12};
lista2.sort(); // both must be ordered
lista.merge(lista2); // lista2 becomes empty, lista iterators NOT altered
std::cout << lista 1; // 6 7 8 9 10 11 12 13
                     // for equivalent elements, the one from lista precedes
                     // the other from lista2 !
\end{verbatim}
Per iteratori: supporta solo \texttt{++it, --it} (no random access)

\chapter{Static data and functions}
Alcuni oggetti possono essere creati fuori da qualsiasi blocco di funzione (incluso \texttt{main}), come \textit{globali}. Hanno \texttt{static} storage duration, ovvero lifetime corrisponde all'intera durata del programma: sono inizializzati prima della chiamata di \texttt{main} e distrutti dopo la fine della sua esecuzione. Sono posizionati nel segmento di memoria indicato con \textit{Static Data}.
\begin{itemize}
\item Possono essere inizializzati a valore costante al compile o dinamicamente (e.g. con chiamata di funzione)
\item L'ordine di inizializzazione e distruzione è \textbf{deterministico} solo nella medesima Translation Unit: in caso di dipendenza tra oggetti in varie TU possibili problemi!
\item \textbf{Meglio evitare di utilizzarli, specie se non costanti}
\end{itemize}
Applicazione: definire costanti (e.g. fisiche), possibilmente dentro un namespace che non sia quello globale
\begin{verbatim}
namespace std::numbers {
  inline constexpr double e = · · · ;
  inline constexpr double pi = · · · ;
  · · ·
} // specify inline if, as probable, definitions are in a header file
\end{verbatim}
\paragraph{Constexpr} garantisce che l'inizializzazione può essere effettuata al compile time. Utilizzato nella dichiarazione di un oggetto o di un metodo non \texttt{static} implica \textbf{\texttt{const}}, in quella di una funzione o di uno \texttt{static data member} implica \texttt{inline}. 
\\Una variabile \texttt{constexpr} deve essere \textit{LiteralType}, essere immediatamente inizializzata e l'espressione che la inizializza (compresa di chiamate a costruttori e conversioni implicite) deve essere una \texttt{constant expression} valutabile al compile time.

\section{Membri static}
(è keyword) Un membro di una classe dichiarato \texttt{static} \textbf{non è parte} di alcun oggetto della classe. Dunque
\begin{itemize}
\item Esiste anche se $\nexists$ oggetti della classe
\item Ha static storage duration
\item \'E dichiarato dentro la classe \textbf{ma definito fuori}, fatto salvo sia dichiarato \texttt{inline} o \texttt{constexpr} (vd. sopra per implicazione) o ancora sia \texttt{const} integral type.
\end{itemize}
Per accedervi si utilizza scope operator; tuttavia se sono stati dichiarati oggetti del tipo si può anche usare operatore di accesso ai metodi (\texttt{.}).
\begin{verbatim}
struct X {
  static int n; // declaration
};
// probably in a .cpp file
int X::n = 42; // definition

//Alternativamente:
struct X {
  ...
  inline static int n = 42; // or constexpr static ...
};

X xx;
std::cout << xx.n; // prints 42
\end{verbatim}
I metodi dichiarati \texttt{static} possono accedere agli altri membri della classe, \textbf{ma solo quelli \texttt{static}}. Non possono inoltre essere dichiarati \texttt{const} (si applica solo a metodi non \texttt{static})
\paragraph{Sospendere processo}
\begin{verbatim}
#include <chrono>
...
using namespace std::chrono_literals;
std::this_thread::sleep_until(std::chrono::system_clock::now() + 15ms);
// now() è uno static member di tipo std::chrono::time_point
\end{verbatim}

\chapter{Dynamic polymorphism}
Concetto centrale nell'\textbf{Object-Oriented Programming}, permette di implementare un'unica interfaccia (polimorfica) per entità di diversi tipi. Dinamico = al runtime!

\section{Inheritance}
Una classe può essere dichiarata \texttt{derived} da una o più classi \texttt{base} (iterando $\rightarrow$ gerarchia).
\begin{itemize}
\item Membri della classe base lo sono anche delle derivate. 
\item La costruzione di un oggetto \texttt{Derived} comporta (anche solo implicitamente) quella di un sotto-oggetto \texttt{Base}
\item Puntatori \texttt{Derived*} possono essere convertiti implicitamente in \texttt{Base*}
\item \'E possibile legare una referenza \texttt{Base\&} a un oggetto \texttt{Derived}
\end{itemize}
\begin{verbatim}
struct Base {
  int a;
  Base(int a) : a{a} {}
  void f();
  int operator()() const;
};
struct Derived : Base {
  double d;
  Derived(int i, double d)
    : Base{i+1}, d{d} {}
  int h() const;
};

Derived de{42, 3.14};
de.d;
de.h();
de.a; // Base::a
de.f(); // Base::f
de(); // Base::operator()
Base* b1 = &de;
Base& b2 = de;
\end{verbatim}

\section{Abstract classes \& Virtual functions}
Permette di realizzare interfaccia comune per classi derivate e loro metodi. \textbf{Non esiste di per sè:} non è possibile creare oggetti di tipo \texttt{Base}!
\begin{verbatim}
struct Shape { // abstract base class
  virtual ~Shape(); // virtual destructor, no '= 0' here
  virtual Point where() const = 0; // pure virtual function
};

struct Circle : Shape { // derived
  Point c;
  int r;
  ~Circle();
  Point where() const override; 
};

struct Rectangle : Shape {
  Point ul;
  Point lr;
  ~Rectangle();
  Point where() const override;
};

std::unique_ptr<Shape> create_shape(); // use a smart pointer *
auto s = create_shape();
s->where(); // call redirected to corresponding function at RUNTIME
// * automatically deleted at end of scope
\end{verbatim}
Un metodo non \texttt{static} è una funzione virtuale se è dichiarata per la prima volta con la keyword o se compie \texttt{override} di una funzione virtuale di una classe base. Una classe è \textbf{polymorphic} se ha almeno un metodo virtuale.
\\\textbf{Funzioni overridden devono avere la stessa segnatura!} altrimenti si ha \textbf{HIDING} : sufficiente varino tipi argomenti oppure si aggiunga attributo \texttt{const}
\\Una funzione è \textbf{pura} se la dichiarazione termina con \texttt{= 0}; salvo eccezione (vd. dopo) \textbf{non può essere definita}. Una classe è \textbf{abstract} se ha almeno un metodo virtuale puro; nel caso non ne possono essere creati oggetti.
\\Possibile anche la classe base non sia astratta:
\begin{verbatim}
struct Shape {
  Point p;
  Shape(Point p) : p{p} {}
  virtual ~Shape();
  virtual Point where() const { return p; } // not pure, has default implementation
};

struct Circle : Shape {
  int r;
  Circle(Point p, int d) : Shape{p}, r{d} {}
  ~Circle();
  // where() is inherited from Shape
};

struct Rectangle : Shape {
  Point lr;
  Rectangle(Point p1, Point p2) : Shape{p1}, lr{p2} {}
  ~Rectangle();
  Point where() const override { return (p + lr) / 2; }
  // implementata diversamente con override (p ereditato da Shape)
};
\end{verbatim}
non consigliato, specie per la gestione delle variabili private: in questo caso, ad esempio, \texttt{p} stesso rappresenta un'interfaccia e non è possibile modificarne il nome.

\section{Slicing}
In generale, per classi base non astratte \textbf{è possibile l'istanziazione e la copia di oggetti}. Nel caso è però consigliato di porre i metodi speciali di copy/move \texttt{= delete}
\\Quando si passa un oggetto di classe derivata a una funzione che prende un parametro di classe base \textbf{by value}, \textbf{VIENE PASSATO SOLO IL SOTTO-OGGETTO DI CLASSE BASE}!
\\In generale si consiglia comunque di \textbf{dichiarare il distruttore come pure virtual} ma di definirlo (in modo che le classi derivate siano distrutte in modo adeguato)
\begin{verbatim}
struct Shape {
  Point ul;
  Shape(Point p): p{p} {}
  virtual ~Shape() = 0;
  virtual Point where() const; // non-pure virtual function
};

inline Shape::~Shape() = default; // or any other implementation
\end{verbatim}
In generale \textbf{Si consiglia di tenere le classi base astratte}

\section{Final}
Una funzione virtuale può avere uno solo degli attributi \texttt{virtual, override, final}. Una funzione \texttt{final} non può essere overridden in classi derivate, mentre una \textbf{classe \texttt{final}} non può avere derivate!
\begin{verbatim}
struct Derived final { · · · };
\end{verbatim}

\section{Access control}
Un membro di classe può essere
\begin{itemize}
\item \textbf{\texttt{public}} nome può essere utilizzato ovunque
\item \textbf{\texttt{private}} solo dai membri e dai friend della classe
\item \textbf{\texttt{protected}} solo dai membri, dai friend della classe \textbf{e dalle derivate da tale classe, oltre che i loro friend}
\end{itemize}
Ordine dentro classe: \texttt{private:} $\rightarrow$ \texttt{protected:} $\rightarrow$ \texttt{public:}
\\Anche la derivazione in sè può avere tali attributi
\begin{verbatim}
class Derived : public|private|protected Base {};
\end{verbatim}
\textbf{Si sconsigliano gli ultimi due}! Con derivazione \texttt{public}:
\begin{itemize}
\item \texttt{public} in Base $\rightarrow$ \texttt{public} in Derived
\item \texttt{protected} in Base $\rightarrow$ \texttt{protected} in Derived
\end{itemize}
\textit{Sub-typing}(relazione \textit{is-a} (\textit{è una}))
\\\textbf{CONSIGLIO:} tenere i data members comunque in \texttt{private}!

\section{Distruttore e copy/move}
In caso di inheritance polimorfica, il distruttore di classe base \textbf{deve essere}
\begin{itemize}
\item \texttt{public} \& \texttt{virtual} oppure
\item \texttt{protected} \& \texttt{non-virtual}
\end{itemize}
Le operazioni di copy/move per la classe base devono essere accessibili alle derivate ma non al public: \textbf{si dichiarano \texttt{protected}}!
\begin{verbatim}
class Base
{
   · · · 
  protected:
   Base(Base const&);
   Base& operator=(Base const&);
   Base(Base&&);
   Base& operator=(Base&&); // FIVE RULE!
  public:
   · · · 
};
\end{verbatim}
Possibile altrimenti implementare un metodo virtuale di \textit{cloning}
\begin{verbatim}
class Base
{
  public:
   virtual Base* clone() const = 0;
};

class Derived: public Base
{
  public:
   Derived* clone() const override
   {
     return new Derived{*this}; // call the copy ctor of Derived
   }
};
\end{verbatim}

\section{I/O Streams}
La libreria Input/Output della SL è implementata tramite gerarchie di classi e basata sul concetto di \textit{streams}. \texttt{std::cin} e \texttt{std::cout} sono riferite al terminale, ma altre classi permettono lettura e scrittura da/su file, stringhe, rete (networking). 
\\Grazie alla gerarchia polimorfica oggetti di tali classi possono essere passati a funzioni implementate partendo dalla classe base.
\\Alias specifici del namespace:
\begin{verbatim}
namespace std {
  using istream = basic_istream<char>;
  using ostream = basic_ostream<char>;
  using istringstream = basic_istringstream<char>;
  using ostringstream = basic_ostringstream<char>;
  using ifstream = basic_ifstream<char>;
  using ofstream = basic_ofstream<char>;
  · · · 
  istream cin;
  ostream cout;
}
\end{verbatim}
\paragraph{Leggere e scrivere file}
\begin{verbatim}
#include <fstream>
...
std::ifstream is{"/tmp/in"}; // open /tmp/in for reading
std::ofstream os{"/tmp/out"}; // open /tmp/out for writing
if (!is || !os) { // supporta conversione a booleano: good = true
                  // fail = false (nell'apertura del file)
// manage error
}
double d;
while (is >> d) {
  os << std::setw(12) << d * 2 << '\n';
}
\end{verbatim}
Il costruttore e il distruttore di \texttt{fstream} aprono e chiudono implicitamente il file. Vi sono anche metodi espliciti \texttt{open()} e \texttt{close()} (sconsigliati)

\paragraph{Leggere e scrivere stringhe}
\begin{verbatim}
#include <sstream>
...
std::istringstream is{"12. 14. -42."};
std::ostringstream os;
double d;
while (is >> d) {
os << std::setw(12) << d * 2 << '\n';

std::cout << os.str(); // member func str() allows to get & set string contents
// " 24\n 28\n -84\n"
\end{verbatim}
Per stampare su stringa da qualsiasi stream in ingresso:
\begin{verbatim}
std::getline(<istream>, <string>);
std::getline(<istream>, <string>, <char>); // stampa fino a quando
                                           // incontra <char>
\end{verbatim}
Con \boxed{\texttt{./a.out < testo.txt}} da shell possibile dare file di testo in input all'eseguibile (ovviamente dopo compilazione).
\subsection{Operatore minore-minore}
\texttt{operator<<} implementato come \textit{free function} (oggetto da streammare è il secondo parametro, non l'oggetto di classe \texttt{*this} su cui si chiamerebbe altrimenti! - si deve stampare sullo stream)
\begin{verbatim}
class Complex {
  ...
 public:
  ...
};

inline std::ostream& operator<<(std::ostream& os, Complex const& c)
{
  os << ... ;
  return os;
}
\end{verbatim}
In generale, implementato come \textbf{template}, accetta e returna \texttt{std::basic\_ostream<...>}

\subsection{Friend functions}
Free functions implementate in modo da essere esposte in modo adeguato attraverso l'interfaccia pubblica e da accedere ai membri privati di classe. Può essere sia in parte privata che pubblica.
\begin{verbatim}
class Complex {
   double r; double i;
  public:
   friend std::ostream& operator<<(std::ostream& os, Complex const& c) {
     os << '(' << c.r << ',' << c.i << ')'; // access to private
     return os;
   }
   · · ·
};
\end{verbatim}
Essendo definite dentro la definizione di classi, le free function sono automaticamente \texttt{inline}. Possono inoltre essere solo dichiarate all'interno e definite altrove
\\Un metodo di una classe può essere \texttt{friend} di un'altra. Se ciò riguarda \textbf{tutti} i metodi della prima classe, la si può in modo sintetico indicare come \texttt{friend} della seconda

\chapter{SFML}
Stutturazione file di esempio (main!). Si nota la struttura del Game Cycle, che viene eseguito a ogni frame. 
\begin{verbatim}
...
#include <SFML/Graphics.hpp>
#include <iostream>
...

auto to_color(int k)
{
  return k < 256 ? sf::Color{static_cast<sf::Uint8>(10 * k), 0, 0}
                 : sf::Color::Black; // header include anche colori
                                     // di default già implementati
                                     // come static const
}

// NB in header file definitions are inside namespace sf
// here one must use scope operator sf:: 

int main()
{
  auto const display_width  = 600u;
  auto const display_height = 600u; // display size
  ...
  sf::RenderWindow window(sf::VideoMode(display_width, display_height),
                          "Mandelbrot Set");
  window.setFramerateLimit(60);

  sf::Image image; 
  image.create(window.getSize().x, window.getSize().y);
  // getSize() retrieves size of rendering region as a Vector2u
  // (type defined in library!)

  for (auto row = 0u; row != display_height; ++row) {
    for (auto column = 0u; column != display_width; ++column) {
      auto k = mandelbrot(top_left + complex{delta_x * column, delta_y * row});
      image.setPixel(column, row, to_color(k)); // change color of a pixel
    }
  }

  sf::Texture texture;
  texture.loadFromImage(image);
  sf::Sprite sprite;
  sprite.setTexture(texture);

  while (window.isOpen()) { // handles events happening during a frame
    sf::Event event;
    while (window.pollEvent(event)) {
      if (event.type == sf::Event::Closed) { // gestisce chiusura finestra
        window.close(); 
      }
    } // NB: tutti gli eventi che avvengono nella durata di un frame
      // (click, etc.) vengono raccolti indiscriminatamente

    window.clear(); // 'stampa' sfondo (colore)

    window.draw(sprite); // da chiamare per ogni forma etc. da visualizzare
                         // ANCHE ITERANDO SU VETTORE DI FORME

    window.display(); // termina frame
    
    using namespace std::chrono_literals; // possibile limitare framerate
    std::this_thread::sleep_for(15ms);
  }
}
\end{verbatim}
Per lavorare con la classe \texttt{Shape} e le sue derivate, \texttt{CircleShape}, \texttt{RectangleShape}, \texttt{ConvexShape}:
\begin{verbatim}
sf::CircleShape circle;
circle.setFillColor(sf::Color::Blue); // or sf::Color( R,G, B);

circle.setPosition( <float_x>, <float_y>);
circle.setPosition( <const sf::Vector2f& >); // vettore posizione nello schermo

//NB l'asse x è orientato da sx a dx, y DALL'ALTO AL BASSO:
// il punto {0,0} è l'angolo in alto a sx

//ALTI METODI:
circle.setRotation( <angolo float>);
circle.setScale( <float_x>, <float_y>); // analogamente con Vector2f
circle.setOrigin(x, y); // imposta origine locale dell'oggetto

circle.move(x,y); circle.move( vec );
circle.rotate(x);
circle.scale(x,y); circle.scale( vec );

circle.getTransform(); // returns the combined tranform of the object
                       // (all the trans. done together) as const Transform&
circle.getInverseTransform(); // analogo ma per la transform combinata inversa

// PER TUTTI I METODI set E' IMPLEMENTATO ANCHE IL CORRISPONDENTE get 
// (return type è l'argomento del set)
\end{verbatim}
Le trasformazioni sono ottenute tramite matrici tridimensionali (trasformazioni affini, non solo lineari!), secondo metodi implementati nella classe base \texttt{Transformable}, da cui deriva la citata \texttt{Shape} ma anche \texttt{Sprite} e \texttt{Text}.
\\Lista colori di default:
\\\texttt{Black, White, Red, Green, Blue, Yellow, Magenta, Cyan, Transparent}
\\~\\\boxed{\textbf{NOTA BENE: VI SONO MEMORY LEAKS INSITI IN SFML!}} \\\boxed{\textbf{NELLA RELAZIONE DEL PROGETTO VANNO INDICATE EVENTUALI SEGNALAZIONI DEL SANITIZER}}

\chapter{Git}
Strumento per il \textit{version control}.

\begin{description}
\item[Creare repo] \boxed{\texttt{git init}}
\item[Stato repo] \boxed{\texttt{git status}}
\item[Aggiungere modifiche] (aggiunta e rimozione di file + modifiche) \boxed{\texttt{git add <filename>}}, anche file multipli o intera \textbf{working directory} - per quest'ultima \boxed{\texttt{git add .}} (\textbf{SCONSIGLIATO})
\item[Aggiungere commit con messaggio] \boxed{\texttt{git commit -m "< messaggio >"}}\textbf{consigliato mettere sempre e descrivere modifiche fatte}
\item[Impostare utente globale] \boxed{\texttt{git --global user.name <Nome>}} \boxed{\texttt{git --global user.mail <indirizzo> }} (vale per tutta la macchina, assunto di default nella creazione di ogni altro repo)
\item[Cronologia repo] \boxed{\texttt{git log}} $\rightarrow$ ogni commit ha \textbf{codice identificativo alfanumerico univoco}, \texttt{Head ->} indica branch corrente
\\\boxed{\texttt{git log --oneline}} per versione compatta
\item[Mostrare differenze tra commit] \boxed{\texttt{git diff}} = dall'inizio all'ultimo
\\\boxed{\texttt{git diff --staged}} = tra ultimo commit e cambiamenti nella staging area
\\\boxed{\texttt{git diff <id>}} = cambiamenti tra ultimo commit e commit indicato (\textbf{SUFFICIENTI PRIMI 5 CARATTERI DELL'ID}); 
\\\boxed{\texttt{git diff <id1> <id2>}} = tra due commit indicati
\item[Togliere da staging] \boxed{\texttt{git restore --staged <filename>}} (non consigliato, meglio stare attenti prima)
\item[Branching] \boxed{\texttt{git branch}} : il corrente è preceduto da \boxed{\texttt{\*}}
\\\textbf{Aggiungere branch} \boxed{\texttt{git branch <branch-name>}}
\\\textbf{Switchare branch} \boxed{\texttt{git checkout <branch-name>}}
\item[Merging] \boxed{1} prima fare checkout nel branch di destinazione (e.g. \texttt{main})
\\\boxed{2} poi \boxed{\texttt{git merge <branch-name>}}
\\\textbf{ERRORE NEL BRANCHING} messaggio in shell:
\boxed{\texttt{CONFLICT (content): Merge conflict in <filename(s)>}}
\\\boxed{\texttt{Automatic merge failed; fix conflicts and then commit the result.}}
\\Per mostrare file con conflitti \boxed{\texttt{git status}};
nel/i file vengono indicate sezioni oggetto di conflitto:
\begin{verbatim}
This content is not affected by the conflict
<<<<<<< main
this is conflicting text from main
=======
this is conflicting text from the new branch
>>>>>>> new_branch;
\end{verbatim}
Si apportano modifiche opportune \textbf{e si rimuovono avvisi}, poi \boxed{\texttt{git add <file incriminato/i>}} $\rightarrow$ \boxed{\texttt{git commit}}
\item[Rimuovere branch] \boxed{\texttt{git branch -d <branch-name>}} (\textbf{Nota:} di per sè non occupano spazio, sono solo etichette. Cmq buona prassi rimuoverli quando non servono più)
\end{description}
\section{GitHub}
\paragraph{SSH} \textit{TBD}
\begin{description}
\item[Aggiungere repository remoto] \boxed{\texttt{git remote add origin <repository_ssh_url>}} (\texttt{origin} è il branch remoto di default)
\item[Sicronizzare (pushare)] \boxed{\texttt{git push -u origin main}} imposta anche collegamento di default
\item[Aggiornare remote repo] \boxed{\texttt{git push}} pusha commit \textbf{dalla local repo} (tutto quello che non è stato committato non viene caricato)
\\Per branch \boxed{\texttt{git push <remote-branch> <local-branch>}}
\end{description}
Il \texttt{push} fallisce se local non è in pari con il remote: possibile recuperare dalla remote \textbf{senza merge con il local} con \boxed{\texttt{gi fetch}}
\\\boxed{\texttt{git pull}} Per fetchare e anche fare merge con stato corrente del workspace
\end{document}
